\documentclass[11pt,preprint, authoryear]{elsarticle}

\usepackage{lmodern}
%%%% My spacing
\usepackage{setspace}
\setstretch{1.2}
\DeclareMathSizes{12}{14}{10}{10}

% Wrap around which gives all figures included the [H] command, or places it "here". This can be tedious to code in Rmarkdown.
\usepackage{float}
\let\origfigure\figure
\let\endorigfigure\endfigure
\renewenvironment{figure}[1][2] {
    \expandafter\origfigure\expandafter[H]
} {
    \endorigfigure
}

\let\origtable\table
\let\endorigtable\endtable
\renewenvironment{table}[1][2] {
    \expandafter\origtable\expandafter[H]
} {
    \endorigtable
}


\usepackage{ifxetex,ifluatex}
\usepackage{fixltx2e} % provides \textsubscript
\ifnum 0\ifxetex 1\fi\ifluatex 1\fi=0 % if pdftex
  \usepackage[T1]{fontenc}
  \usepackage[utf8]{inputenc}
\else % if luatex or xelatex
  \ifxetex
    \usepackage{mathspec}
    \usepackage{xltxtra,xunicode}
  \else
    \usepackage{fontspec}
  \fi
  \defaultfontfeatures{Mapping=tex-text,Scale=MatchLowercase}
  \newcommand{\euro}{€}
\fi

\usepackage{amssymb, amsmath, amsthm, amsfonts}

\def\bibsection{\section*{References}} %%% Make "References" appear before bibliography


\usepackage[round]{natbib}

\usepackage{longtable}
\usepackage[margin=2.3cm,bottom=2cm,top=2.5cm, includefoot]{geometry}
\usepackage{fancyhdr}
\usepackage[bottom, hang, flushmargin]{footmisc}
\usepackage{graphicx}
\numberwithin{equation}{section}
\numberwithin{figure}{section}
\numberwithin{table}{section}
\setlength{\parindent}{0cm}
\setlength{\parskip}{1.3ex plus 0.5ex minus 0.3ex}
\usepackage{textcomp}
\renewcommand{\headrulewidth}{0.2pt}
\renewcommand{\footrulewidth}{0.3pt}

\usepackage{array}
\newcolumntype{x}[1]{>{\centering\arraybackslash\hspace{0pt}}p{#1}}

%%%%  Remove the "preprint submitted to" part. Don't worry about this either, it just looks better without it:
\makeatletter
\def\ps@pprintTitle{%
  \let\@oddhead\@empty
  \let\@evenhead\@empty
  \let\@oddfoot\@empty
  \let\@evenfoot\@oddfoot
}
\makeatother

 \def\tightlist{} % This allows for subbullets!

\usepackage{hyperref}
\hypersetup{breaklinks=true,
            bookmarks=true,
            colorlinks=true,
            citecolor=blue,
            urlcolor=blue,
            linkcolor=blue,
            pdfborder={0 0 0}}


% The following packages allow huxtable to work:
\usepackage{siunitx}
\usepackage{multirow}
\usepackage{hhline}
\usepackage{calc}
\usepackage{tabularx}
\usepackage{booktabs}
\usepackage{caption}


\newenvironment{columns}[1][]{}{}

\newenvironment{column}[1]{\begin{minipage}{#1}\ignorespaces}{%
\end{minipage}
\ifhmode\unskip\fi
\aftergroup\useignorespacesandallpars}

\def\useignorespacesandallpars#1\ignorespaces\fi{%
#1\fi\ignorespacesandallpars}

\makeatletter
\def\ignorespacesandallpars{%
  \@ifnextchar\par
    {\expandafter\ignorespacesandallpars\@gobble}%
    {}%
}
\makeatother

\newenvironment{CSLReferences}[2]{%
}

\urlstyle{same}  % don't use monospace font for urls
\setlength{\parindent}{0pt}
\setlength{\parskip}{6pt plus 2pt minus 1pt}
\setlength{\emergencystretch}{3em}  % prevent overfull lines
\setcounter{secnumdepth}{5}

%%% Use protect on footnotes to avoid problems with footnotes in titles
\let\rmarkdownfootnote\footnote%
\def\footnote{\protect\rmarkdownfootnote}
\IfFileExists{upquote.sty}{\usepackage{upquote}}{}

%%% Include extra packages specified by user

%%% Hard setting column skips for reports - this ensures greater consistency and control over the length settings in the document.
%% page layout
%% paragraphs
\setlength{\baselineskip}{12pt plus 0pt minus 0pt}
\setlength{\parskip}{12pt plus 0pt minus 0pt}
\setlength{\parindent}{0pt plus 0pt minus 0pt}
%% floats
\setlength{\floatsep}{12pt plus 0 pt minus 0pt}
\setlength{\textfloatsep}{20pt plus 0pt minus 0pt}
\setlength{\intextsep}{14pt plus 0pt minus 0pt}
\setlength{\dbltextfloatsep}{20pt plus 0pt minus 0pt}
\setlength{\dblfloatsep}{14pt plus 0pt minus 0pt}
%% maths
\setlength{\abovedisplayskip}{12pt plus 0pt minus 0pt}
\setlength{\belowdisplayskip}{12pt plus 0pt minus 0pt}
%% lists
\setlength{\topsep}{10pt plus 0pt minus 0pt}
\setlength{\partopsep}{3pt plus 0pt minus 0pt}
\setlength{\itemsep}{5pt plus 0pt minus 0pt}
\setlength{\labelsep}{8mm plus 0mm minus 0mm}
\setlength{\parsep}{\the\parskip}
\setlength{\listparindent}{\the\parindent}
%% verbatim
\setlength{\fboxsep}{5pt plus 0pt minus 0pt}



\begin{document}



\begin{frontmatter}  %

\title{Literature Review}

% Set to FALSE if wanting to remove title (for submission)




\author[Add1]{Gabriel Rambanapasi}
\ead{gabriel.rams44@gmail.com}





\address[Add1]{Stellenbosch University, Cape Town, South Africa}

\cortext[cor]{Corresponding author: Gabriel Rambanapasi}

\begin{abstract}
\small{
A discussion of past literature that has influenced the investigation
into dividend strategies. We take a novel approach in combining firms of
the same sectors but in different geographical regional. Overall we find
that the effect is much more pronounced used a forward and backward
looking indicator.
}
\end{abstract}

\vspace{1cm}


\begin{keyword}
\footnotesize{
Multivariate GARCH \sep Kalman Filter \sep Copula \\
\vspace{0.3cm}
}
\footnotesize{
\textit{JEL classification} L250 \sep L100
}
\end{keyword}



\vspace{0.5cm}

\end{frontmatter}

\setcounter{footnote}{0}



%________________________
% Header and Footers
%%%%%%%%%%%%%%%%%%%%%%%%%%%%%%%%%
\pagestyle{fancy}
\chead{}
\rhead{}
\lfoot{}
\rfoot{\footnotesize Page \thepage}
\lhead{}
%\rfoot{\footnotesize Page \thepage } % "e.g. Page 2"
\cfoot{}

%\setlength\headheight{30pt}
%%%%%%%%%%%%%%%%%%%%%%%%%%%%%%%%%
%________________________

\headsep 35pt % So that header does not go over title




\hypertarget{introduction}{%
\section{\texorpdfstring{Introduction
\label{Introduction}}{Introduction }}\label{introduction}}

\hypertarget{dividends-constitution-and-impliaction-on-stakeholder-and-firms}{%
\section{Dividends constitution and Impliaction on Stakeholder and
Firms}\label{dividends-constitution-and-impliaction-on-stakeholder-and-firms}}

Dividends are cash payouts that companies use to distribute capital to
their shareholders. They can take various forms, such as cash, stock,
liquidating, scrip, or property dividends. However, cash dividends are
the most common type. The decision to issue dividends is typically made
by the board of directors, taking into account the company's operating
needs for a given financial year. When a dividend is announced, it has
an impact on the financial statements of a company. Resultantly,
dividend announcements create a liability which impacts the balance
sheet by increasing the current liabilities and decreases shareholder
equity, specifically retained earnings, on the balance sheet. In other
words, the company incurs an obligation to pay out the dividend, and the
value of the company retained by shareholders decreases accordingly.
Various literature have shown that in some countries shareholder values
dcreases in line with dividend payments. Vijayakumar (2010), Hussainey,
Mgbame, and Chijoke-Mgbame (2011) and Masum (2014) show negative effect
on prices on ex dividend day following the payments of dividends. This
effectively means that value of a shareholder remains the same as
capital its just swapped for cash Suwanna
(\protect\hyperlink{ref-suwanna2012impacts}{2012}).

Cash dividends, although widely used, are not as tax-efficient as other
types of capital distributions, such as share buybacks. Share buybacks
have gained popularity in advanced economies, particularly in the United
States, where they reached a high of \$437 billion in 2018.
Surprisingly, their adoption has been relatively slow in South Africa.
According to a study by Wesson, Muller, and Ward in 2014, there were
only 195 open market share repurchases announced in South Africa from
1999 to 2009. In comparison, Manconi, Peyer, and Vermaelen estimated
that share repurchases constituted approximately 58\% of total
announcements in the United States, 15\% in Canada, and 11\% in Japan
over the same period. This indicates a significant disparity in the
adoption of share buybacks across the world, despite their popularity in
the United States.

\begin{itemize}
\tightlist
\item
  update argument on sharebuybacks and why you just mentioned them
  instead of others
\end{itemize}

\hypertarget{why-do-firms-pay-dividends}{%
\subsection{Why do firms pay
dividends}\label{why-do-firms-pay-dividends}}

The issue of dividend policy and its impact on shareholder wealth has
sparked debates and opposing arguments in financial literature. On one
hand, the dividend puzzle suggests that dividends reduce equity value
and make investors worse off. According to this view, dividends can be
seen as a reward for investors who bear the risk associated with their
investments. Additionally, dividends can be considered a return on
investment rather than relying solely on capital gains. Various
literature has emerged trying to solve the puzzel, either supporting
irrelevance or relevance in dividend payments.The Modigliani-Miller (MM)
theory opines that dividends are irrelevant, it argues that shareholders
are indifferent to dividend payments, implying that there is no optimal
dividend policy. They contend that all dividend policies are equally
good and thus payments of dividends could easily be reinvested in shares
and make no difference to share holder wealth. However, the MM theorem
fails to consider real-world market imperfections that may give
relevance to dividend payments. Global assets market face multiple
constraints or imperfections namely flotation costs, transaction costs
(e.g., taxes and flotation costs), information asymmetry, and
principal-agent problems.

Tax preferences play a role in the argument for dividend relevance.
Different investors may be attracted to different stocks based on their
tax treatments, thus investors choose stocks based on their individual
investment needs. However, supporters of the MM theorem argue that
changes in dividend policy should not significantly impact stock prices
due to the substitution effect. According to this effect, allocation
decisions of firms occur almost simultaneously, resulting in a net zero
effect on prices. Flotation costs refer to the opportunity costs
incurred by a firm when paying dividends. By distributing dividends,
companies forego opportunities to expand their operations using retained
earnings. In a world without flotation costs, as suggested by the MM
theorem, management would be indifferent between issuing dividends and
borrowing from the market. However, in reality, external financing comes
at a higher cost, leading to trade-offs in dividend policy decisions.

Information asymmetry between shareholders and managers is another
factor that gives relevance to dividend payments. Investors rely on
dividend announcements to assess a company's stock price. Dividend
signaling conveys information about the company's quality. Investors
compare dividend announcements to historical levels while considering
company fundamentals. However, there is a risk of manipulation by
management, making the dividend signal imperfect for determining share
prices. Extending the argument on information asymmetry leads to the
principal-agent problem, where management and shareholders may have
differing goals for the use of retained earnings, leading to conflicts.
The free cash flow hypothesis suggests that dividend payments force
management to raise capital from external sources, which increases
borrowing costs and scrutiny from capital markets. This, in turn,
reduces management's ability to make sub optimal investments.

\hypertarget{dividend-portfolios-to-signal-returns}{%
\subsection{Dividend Portfolios To Signal
Returns}\label{dividend-portfolios-to-signal-returns}}

\begin{itemize}
\tightlist
\item
  Include the argumemt that there are a proxy for value
\end{itemize}

Studies on dividend signaling for returns can be categorized into
academic return signaling studies and practitioner-oriented long-term
return studies. Academic studies, such as Fama and French (1988),
initially found a positive correlation between increasing predictive
power and longer forecast horizons. However, subsequent studies like
Bekaert and Ang (2001) and Ang and Bekaert (2006) found no evidence of
long-term predictability in stock returns when considering finite sample
influence. This suggests that dividend yield may not be a reliable
predictor of subsequent returns. One possible reason for this declining
predictive power is the increasing use of share buybacks as an
alternative means for capital distribution, which reduces the
contribution of dividend yield to total return (Robertson and Wright,
2006).

On the other hand, practitioner-oriented literature focuses on the
long-term returns of systematic dividend portfolios. One popular
strategy is the ``Dogs of the Dow (DOD),'' which involves creating a
portfolio of the top 10 highest-paying dividend stocks on the Dow Jones
Industrial Index at the beginning of the year based on the dividends
paid in the previous 12 months. The portfolio is held for 12 months, and
the process is repeated annually. Various studies have examined the DOD
strategy or similar high-yield dividend strategies in different time
periods and regions, consistently showing superior risk-adjusted returns
compared to the market index. Examples of such studies include Da Silva
(2001) in Latin America, Alles and Sheng (2008) in Australia, Visscher
and Filbeck (2003) in Canada, Kotkamp and Otte (2001) in Germany, and
Wang et al.~(2011) in China. More recently, Filbeck (2017) investigated
the performance of DOD against a high-yield portfolio of Fortune Most
Desired Companies (MAC) compared to the Dow Jones Industrial Average and
the S\&P 500. The study found significantly higher risk-adjusted returns
for the DOD strategy.

\begin{longtable}[]{@{}
  >{\raggedleft\arraybackslash}p{(\columnwidth - 8\tabcolsep) * \real{0.0060}}
  >{\raggedright\arraybackslash}p{(\columnwidth - 8\tabcolsep) * \real{0.0251}}
  >{\raggedright\arraybackslash}p{(\columnwidth - 8\tabcolsep) * \real{0.1623}}
  >{\raggedright\arraybackslash}p{(\columnwidth - 8\tabcolsep) * \real{0.0644}}
  >{\raggedright\arraybackslash}p{(\columnwidth - 8\tabcolsep) * \real{0.7422}}@{}}
\toprule()
\begin{minipage}[b]{\linewidth}\raggedleft
Year
\end{minipage} & \begin{minipage}[b]{\linewidth}\raggedright
Author
\end{minipage} & \begin{minipage}[b]{\linewidth}\raggedright
Objectives
\end{minipage} & \begin{minipage}[b]{\linewidth}\raggedright
Methodology
\end{minipage} & \begin{minipage}[b]{\linewidth}\raggedright
Findings
\end{minipage} \\
\midrule()
\endhead
1997 & Filbeck & Testing the performance of the top paying dividend
stock in the UK market index. & Optimization, equally dollar weighting
stock. & Dividend yield investment strategy in Britain was not effective
between March 1984 and February 1994. The portfolio returns exceeded the
market returns on both unadjusted and risk adjusted bases, in only four
years. The superior multiple year Top Ten portfolio performances were
primarily due to the outstanding second year returns \\
2001 & Da Silva & Investigate dividend-driven trading strategies based
on dividend yield growth effects in the Polish stock market in the years
1994--2004 & NA & NA \\
2003 & Visscher and Filbeck & Test the Dividends of the Dog Strategy in
Canada. & High dividend yield strategy to the Toronto 35 Index. & Top 10
highest performing portolios returns were high enough to compnesate for
transaction costs and taxes. The Dogs Strategy also performed well
against the Toronto 35 index. \\
2007 & Brzeszczyński & Testing the Dogs Strategy in Poland & NA &
Dividend yield growth portfolios were capable of beating the market in
the entire sample period. Their performance, however, was not consistent
over time and the highest returns were obtained during final
years.different types of portfolios demonstrate the importance of
dividends as a source of significant fundamental information items from
stock market companies. At the same time, they show that a dividend
investment strategy for the Polish stock market is most successful when
the selection of stocks for the dividend yield growth portfolios is
subject to further restrictions, most notably concerning company
size. \\
2007 & Fama and Eugene & investigate the contributions to return for
value versus growth stock & NA & Returns are higher for value than there
are for growth stock. However, that result result is only special from
the period 1964-2004. From 1927 to 1963 the contribution of dividends to
return is not different between value and growth stock. \\
2011 & Wang et al & Test the Dividends of the Dog Strategy in China. &
Optimiization & Dow Dogs portfolios significantly outperform the market
benchmark for the period of 1994 to 2009 in China‟s markets. Further
analysis indicates that (1) The fewer Dogs included in the portfolio,
the greater the portfolio abnormal returns; (2) In general, the shorter
holding period (in months), the greater the portfolio abnormal
returns \\
2011 & Rennie & Assess the performance of the Dogs of the Dow strategy
in Finland & Optimiization & The Dogs of the Dow Strategy out performs
the market index. Outperformance is more pronounced in market
downturns. \\
2015 & Lemmon & The dividend yield effect is tested in Hong Kong where
no taxes exist on either dividend income or capital gain. & NA & paper
documents a robust dividend yield effect in the Hong Kong market and
suggests that nontax reasons help to explain the yield effect. \\
2017 & Filbeck & Test the Dividends of the Dog Strategy against Most
Admired Companies and the Dow Jones Industrial Average. & Optimiization
& MAC Dogs provide significantly higher raw returns than both the DOW
and the standard DoD strategy. Moreover, the MAC Dogs provide
significantly higher raw and risk-adjusted returns against our
benchmarks. Our results remain robust when using either alternative
dividend-yield cutoff points or alternative event windows surrounding
the issue dates for Fortune \\
2017 & You & Performance of Dogs versus Market index in Taiwan. &
Optimization & DY portfolios provide significant yield and investor
appear to have a lagged reaction to dividend announcements. \\
\bottomrule()
\end{longtable}

\begin{itemize}
\tightlist
\item
  Give a summary of the indexes and methodology used in all the papers
  listed. Put this in excel and make into a table.
\item
  Look at papers that investigate dividend signalling by sector.
\item
\end{itemize}

References are to be made as follows: Fama \& French
(\protect\hyperlink{ref-fama1997}{1997: 33}) and Grinold \& Kahn
(\protect\hyperlink{ref-grinold2000}{2000}) Such authors could also be
referenced in brackets (\protect\hyperlink{ref-grinold2000}{Grinold \&
Kahn, 2000}) and together Grinold \& Kahn
(\protect\hyperlink{ref-grinold2000}{2000}). Source the reference code
from scholar.google.com by clicking on ``cite'\,' below article name.
Then select BibTeX at the bottom of the Cite window, and proceed to copy
and paste this code into your ref.bib file, located in the directory's
Tex folder. Open this file in Rstudio for ease of management, else open
it in your preferred Tex environment. Add and manage your article
details here for simplicity - once saved, it will self-adjust in your
paper.

\begin{quote}
I suggest renaming the top line after @article, as done in the template
ref.bib file, to something more intuitive for you to remember. Do not
change the rest of the code. Also, be mindful of the fact that bib
references from google scholar may at times be incorrect. Reference
Latex forums for correct bibtex notation.
\end{quote}

To reference a section, you have to set a label using
``\textbackslash label'\,' in R, and then reference it in-text as
e.g.~referencing a later section, Section \ref{Meth}.

\newpage

\hypertarget{references}{%
\section*{References}\label{references}}
\addcontentsline{toc}{section}{References}

\hypertarget{refs}{}
\begin{CSLReferences}{1}{0}
\leavevmode\vadjust pre{\hypertarget{ref-bhattacharyya2007dividend}{}}%
Bhattacharyya, N. 2007. Dividend policy: A review. \emph{Managerial
Finance}. 33(1):4--13.

\leavevmode\vadjust pre{\hypertarget{ref-black1976dividend}{}}%
Black, F. 1976. The dividend puzzle. \emph{The journal of portfolio
management}. 2(2):5--8.

\leavevmode\vadjust pre{\hypertarget{ref-cornell2014dividend}{}}%
Cornell, B. 2014. Dividend-price ratios and stock returns: International
evidence. \emph{Journal of Portfolio management}. 40(2):122.

\leavevmode\vadjust pre{\hypertarget{ref-deangelo2006}{}}%
DeAngelo, H. \& DeAngelo, L. 2006. The irrelevance of the MM dividend
irrelevance theorem. \emph{Journal of financial economics}.
79(2):293--315.

\leavevmode\vadjust pre{\hypertarget{ref-fama1997}{}}%
Fama, E.F. \& French, K.R. 1997. Industry costs of equity. \emph{Journal
of financial economics}. 43(2):153--193.

\leavevmode\vadjust pre{\hypertarget{ref-gordon1963optimal}{}}%
Gordon, M.J. 1963. Optimal investment and financing policy. \emph{The
Journal of finance}. 18(2):264--272.

\leavevmode\vadjust pre{\hypertarget{ref-grinold2000}{}}%
Grinold, R.C. \& Kahn, R.N. 2000. Active portfolio management.

\leavevmode\vadjust pre{\hypertarget{ref-jensen1976theory}{}}%
Jensen, M.C. \& Meckling, W.H. 1976. Theory of the firm: Managerial
behavior, agency costs and ownership structure. \emph{Journal of
financial economics}. 3(4):305--360.

\leavevmode\vadjust pre{\hypertarget{ref-Texevier}{}}%
Katzke, N.F. 2017. \emph{{Texevier}: {P}ackage to create elsevier
templates for rmarkdown}. Stellenbosch, South Africa: Bureau for
Economic Research.

\leavevmode\vadjust pre{\hypertarget{ref-10.2307ux2f3694818}{}}%
Koch, A.S. \& Sun, A.X. 2004. Dividend changes and the persistence of
past earnings changes. \emph{The Journal of Finance}. 59(5):2093--2116.

\leavevmode\vadjust pre{\hypertarget{ref-lintner1956distribution}{}}%
Lintner, J. 1956. Distribution of incomes of corporations among
dividends, retained earnings, and taxes. \emph{The American economic
review}. 46(2):97--113.

\leavevmode\vadjust pre{\hypertarget{ref-miller1985dividend}{}}%
Miller, M.H. \& Rock, K. 1985. Dividend policy under asymmetric
information. \emph{The Journal of finance}. 40(4):1031--1051.

\leavevmode\vadjust pre{\hypertarget{ref-rangvid2014dividend}{}}%
Rangvid, J., Schmeling, M. \& Schrimpf, A. 2014. Dividend predictability
around the world. \emph{Journal of Financial and Quantitative Analysis}.
49(5-6):1255--1277.

\leavevmode\vadjust pre{\hypertarget{ref-suwanna2012impacts}{}}%
Suwanna, T. 2012. Impacts of dividend announcement on stock return.
\emph{Procedia-Social and Behavioral Sciences}. 40:721--725.

\end{CSLReferences}

\hypertarget{appendix}{%
\section*{Appendix}\label{appendix}}
\addcontentsline{toc}{section}{Appendix}

\hypertarget{appendix-a}{%
\subsection*{Appendix A}\label{appendix-a}}
\addcontentsline{toc}{subsection}{Appendix A}

Some appendix information here

\hypertarget{appendix-b}{%
\subsection*{Appendix B}\label{appendix-b}}
\addcontentsline{toc}{subsection}{Appendix B}

\bibliography{Tex/ref}





\end{document}
