\documentclass[11pt,preprint, authoryear]{elsarticle}

\usepackage{lmodern}
%%%% My spacing
\usepackage{setspace}
\setstretch{1.2}
\DeclareMathSizes{12}{14}{10}{10}

% Wrap around which gives all figures included the [H] command, or places it "here". This can be tedious to code in Rmarkdown.
\usepackage{float}
\let\origfigure\figure
\let\endorigfigure\endfigure
\renewenvironment{figure}[1][2] {
    \expandafter\origfigure\expandafter[H]
} {
    \endorigfigure
}

\let\origtable\table
\let\endorigtable\endtable
\renewenvironment{table}[1][2] {
    \expandafter\origtable\expandafter[H]
} {
    \endorigtable
}


\usepackage{ifxetex,ifluatex}
\usepackage{fixltx2e} % provides \textsubscript
\ifnum 0\ifxetex 1\fi\ifluatex 1\fi=0 % if pdftex
  \usepackage[T1]{fontenc}
  \usepackage[utf8]{inputenc}
\else % if luatex or xelatex
  \ifxetex
    \usepackage{mathspec}
    \usepackage{xltxtra,xunicode}
  \else
    \usepackage{fontspec}
  \fi
  \defaultfontfeatures{Mapping=tex-text,Scale=MatchLowercase}
  \newcommand{\euro}{€}
\fi

\usepackage{amssymb, amsmath, amsthm, amsfonts}

\def\bibsection{\section*{References}} %%% Make "References" appear before bibliography


\usepackage[round]{natbib}

\usepackage{longtable}
\usepackage[margin=2.3cm,bottom=2cm,top=2.5cm, includefoot]{geometry}
\usepackage{fancyhdr}
\usepackage[bottom, hang, flushmargin]{footmisc}
\usepackage{graphicx}
\numberwithin{equation}{section}
\numberwithin{figure}{section}
\numberwithin{table}{section}
\setlength{\parindent}{0cm}
\setlength{\parskip}{1.3ex plus 0.5ex minus 0.3ex}
\usepackage{textcomp}
\renewcommand{\headrulewidth}{0.2pt}
\renewcommand{\footrulewidth}{0.3pt}

\usepackage{array}
\newcolumntype{x}[1]{>{\centering\arraybackslash\hspace{0pt}}p{#1}}

%%%%  Remove the "preprint submitted to" part. Don't worry about this either, it just looks better without it:
\makeatletter
\def\ps@pprintTitle{%
  \let\@oddhead\@empty
  \let\@evenhead\@empty
  \let\@oddfoot\@empty
  \let\@evenfoot\@oddfoot
}
\makeatother

 \def\tightlist{} % This allows for subbullets!

\usepackage{hyperref}
\hypersetup{breaklinks=true,
            bookmarks=true,
            colorlinks=true,
            citecolor=blue,
            urlcolor=blue,
            linkcolor=blue,
            pdfborder={0 0 0}}


% The following packages allow huxtable to work:
\usepackage{siunitx}
\usepackage{multirow}
\usepackage{hhline}
\usepackage{calc}
\usepackage{tabularx}
\usepackage{booktabs}
\usepackage{caption}


\newenvironment{columns}[1][]{}{}

\newenvironment{column}[1]{\begin{minipage}{#1}\ignorespaces}{%
\end{minipage}
\ifhmode\unskip\fi
\aftergroup\useignorespacesandallpars}

\def\useignorespacesandallpars#1\ignorespaces\fi{%
#1\fi\ignorespacesandallpars}

\makeatletter
\def\ignorespacesandallpars{%
  \@ifnextchar\par
    {\expandafter\ignorespacesandallpars\@gobble}%
    {}%
}
\makeatother

\newenvironment{CSLReferences}[2]{%
}

\urlstyle{same}  % don't use monospace font for urls
\setlength{\parindent}{0pt}
\setlength{\parskip}{6pt plus 2pt minus 1pt}
\setlength{\emergencystretch}{3em}  % prevent overfull lines
\setcounter{secnumdepth}{5}

%%% Use protect on footnotes to avoid problems with footnotes in titles
\let\rmarkdownfootnote\footnote%
\def\footnote{\protect\rmarkdownfootnote}
\IfFileExists{upquote.sty}{\usepackage{upquote}}{}

%%% Include extra packages specified by user

%%% Hard setting column skips for reports - this ensures greater consistency and control over the length settings in the document.
%% page layout
%% paragraphs
\setlength{\baselineskip}{12pt plus 0pt minus 0pt}
\setlength{\parskip}{12pt plus 0pt minus 0pt}
\setlength{\parindent}{0pt plus 0pt minus 0pt}
%% floats
\setlength{\floatsep}{12pt plus 0 pt minus 0pt}
\setlength{\textfloatsep}{20pt plus 0pt minus 0pt}
\setlength{\intextsep}{14pt plus 0pt minus 0pt}
\setlength{\dbltextfloatsep}{20pt plus 0pt minus 0pt}
\setlength{\dblfloatsep}{14pt plus 0pt minus 0pt}
%% maths
\setlength{\abovedisplayskip}{12pt plus 0pt minus 0pt}
\setlength{\belowdisplayskip}{12pt plus 0pt minus 0pt}
%% lists
\setlength{\topsep}{10pt plus 0pt minus 0pt}
\setlength{\partopsep}{3pt plus 0pt minus 0pt}
\setlength{\itemsep}{5pt plus 0pt minus 0pt}
\setlength{\labelsep}{8mm plus 0mm minus 0mm}
\setlength{\parsep}{\the\parskip}
\setlength{\listparindent}{\the\parindent}
%% verbatim
\setlength{\fboxsep}{5pt plus 0pt minus 0pt}



\begin{document}



\begin{frontmatter}  %

\title{Literature Review}

% Set to FALSE if wanting to remove title (for submission)




\author[Add1]{Gabriel Rambanapasi}
\ead{gabriel.rams44@gmail.com}





\address[Add1]{Stellenbosch University, Cape Town, South Africa}

\cortext[cor]{Corresponding author: Gabriel Rambanapasi}

\begin{abstract}
\small{
A discussion of past literature that has influenced the investigation
into dividend strategies. We take a novel approach in combining firms of
the same sectors but in different geographical regional. Overall we find
that the effect is much more pronounced used a forward and backward
looking indicator.
}
\end{abstract}

\vspace{1cm}


\begin{keyword}
\footnotesize{
Multivariate GARCH \sep Kalman Filter \sep Copula \\
\vspace{0.3cm}
}
\footnotesize{
\textit{JEL classification} L250 \sep L100
}
\end{keyword}



\vspace{0.5cm}

\end{frontmatter}

\setcounter{footnote}{0}



%________________________
% Header and Footers
%%%%%%%%%%%%%%%%%%%%%%%%%%%%%%%%%
\pagestyle{fancy}
\chead{}
\rhead{}
\lfoot{}
\rfoot{\footnotesize Page \thepage}
\lhead{}
%\rfoot{\footnotesize Page \thepage } % "e.g. Page 2"
\cfoot{}

%\setlength\headheight{30pt}
%%%%%%%%%%%%%%%%%%%%%%%%%%%%%%%%%
%________________________

\headsep 35pt % So that header does not go over title




\hypertarget{introduction}{%
\section{\texorpdfstring{Introduction
\label{Introduction}}{Introduction }}\label{introduction}}

\hypertarget{dividends-constitution-and-impliaction-on-stakeholder-and-firms}{%
\section{Dividends constitution and Impliaction on Stakeholder and
Firms}\label{dividends-constitution-and-impliaction-on-stakeholder-and-firms}}

Dividends are cash payouts that companies use to distribute capital to
their shareholders. They can take various forms, such as cash, stock,
liquidating, scrip, or property {[}baker2009understanding{]}. However,
cash dividends are the most common type. The decision to issue dividends
is typically made by the board of directors, taking into account the
company's operating needs for a given financial year. When a dividend is
announced, it has an impact on the financial statements of a company.
Resultantly, dividend announcements create a liability which impacts the
balance sheet by increasing the current liabilities and decreases
shareholder equity, specifically retained earnings, on the balance Baker
\& Powell (\protect\hyperlink{ref-baker2009understanding}{2009: 374}).
In other words, the company incurs an obligation to pay out the
dividend, and the value of the company retained by shareholders
decreases accordingly. Various literature have shown that in some
countries shareholder values dcreases in line with dividend payments.
Vijayakumar (\protect\hyperlink{ref-vijayakumar2010effect}{2010}),
Hussainey, Mgbame \& Chijoke-Mgbame
(\protect\hyperlink{ref-hussainey2011dividend}{2011}) and Masum
(\protect\hyperlink{ref-masum2014dividend}{2014}) show negative effect
on prices on ex dividend day following the payments of dividends. This
effectively means that value of a shareholder remains the same as
capital its just swapped for cash.

Cash dividends, although widely used, are not as tax-efficient as other
types of capital distributions, such as share buybacks or stock
repurchases. This form of capital redistribution a firm exchanges assets
for outstanding shares, which shrinks the company's assets by the amount
of cash paid out. This action too reduces both its borrowing base and
the shareholders' aggregate equity Baker \& Powell
(\protect\hyperlink{ref-baker2009understanding}{2009: 429}). A clear
benefit to the company is that it is more flexible when compared to the
rigid dividend payout structures. To most higher net worth investors,
tax benefits in the form of lower capital gains taxes also apply in most
jurisdictions. It is thus no surprise that this form of earnings
redistribution has gained traction in some advanced economies such as
the US (with the Financial Times recently reporting that announced stock
buybacks in the US reached an all time high of \$437 billion in 2018).
Surprisingly, their adoption has been relatively slow in some emerging
economies. According to a study by Wesson, Muller \& Ward
(\protect\hyperlink{ref-wesson2014market}{2014}), there were only 195
open market share repurchases announced in South Africa from 1999 to
2009. In comparison, Manconi, Peyer \& Vermaelen
(\protect\hyperlink{ref-manconi2014buybacks}{2014}) estimated that share
repurchases constituted approximately 58\% of total announcements in the
United States, 15\% in Canada, and 11\% in Japan over the same period,
indicative of a significant disparity in the adoption of share buybacks
across the world, despite their popularity in the United States.

The logical question then is why do firms pay dividends? Miller \& Rock
(\protect\hyperlink{ref-miller1985dividend}{1985}) opines that dividends
are irrelevant (MM theory), it argues that shareholders are indifferent
to dividend payments, implying that there is no optimal dividend policy.
They contend that all dividend policies are equally good and thus
payments of dividends could easily be reinvested in shares and make no
difference to share holder wealth. However, the MM theorem fails to
consider real-world market imperfections that may give relevance to
dividend payments. To this end we we unpack the dividend puzzel that
suggest dividends reduce equity value and make investors worse off Black
(\protect\hyperlink{ref-black1996dividend}{1996}). However, dividends
can be seen as a reward to investors who bear the risk associated with
their investments. Also, dividends can be considered a return on
investment rather than relying solely on capital gains Black
(\protect\hyperlink{ref-black1996dividend}{1996}). Various literature
has emerged trying to solve the puzzel, either supporting irrelevance or
relevance in dividend payments.Global assets market face multiple
constraints or imperfections namely flotation costs, transaction costs
(e.g., taxes and flotation costs), information asymmetry, and
principal-agent problems.

Tax preferences play a role in the argument for dividend relevance.
Different investors may be attracted to different stocks based on their
tax treatments, thus investors choose stocks based on their individual
investment needs Van Deventer, Imai \& Mesler
(\protect\hyperlink{ref-van2013advanced}{2013}) Baker \& Powell
(\protect\hyperlink{ref-baker2009understanding}{2009}). However,
supporters of the MM theorem argue that changes in dividend policy
should not significantly impact stock prices due to the substitution
effect. According to this effect, allocation decisions of firms occur
almost simultaneously, resulting in a net zero effect on prices
{[}baker2009understanding{]}. Flotation costs refer to the opportunity
costs incurred by a firm when paying dividends. By distributing
dividends, companies forego opportunities to expand their operations
using retained earnings. In a world without flotation costs, as
suggested by the MM theorem, management would be indifferent between
issuing dividends and borrowing from the market. However, in reality,
external financing comes at a higher cost, leading to trade-offs in
dividend policy decisions.

Information asymmetry between shareholders and managers is another
factor that gives relevance to dividend payments. Investors rely on
dividend announcements to assess a company's stock price. Dividend
signaling conveys information about the company's quality
{[}al2018revisiting \& baker1999corporate{]}. Investors compare dividend
announcements to historical levels while considering company
fundamentals. However, there is a risk of manipulation by management,
making the dividend signal imperfect for determining share prices.
Extending the argument on information asymmetry leads to the
principal-agent problem, where management and shareholders may have
differing goals for the use of retained earnings, leading to conflict
{[}baker2009understanding{]}. The free cash flow hypothesis suggests
that dividend payments force management to raise capital from external
sources, which increases borrowing costs and scrutiny from capital
markets. This, in turn, reduces management's ability to make sub optimal
investments {[}baker2009understanding{]}.

Studies on dividend signaling for returns can be categorized into
academic return signaling studies and practitioner-oriented long-term
return studies. Academic studies, such as Fama \& French
(\protect\hyperlink{ref-fama1988permanent}{1988}), initially found a
positive correlation between increasing predictive power and longer
forecast horizons. However, subsequent studies like Ang \& Bekaert
(\protect\hyperlink{ref-ang2007stock}{2007}) found no evidence of
long-term predictability in stock returns when considering finite sample
influence. This suggests that dividend yield may not be a reliable
predictor of subsequent returns. One possible reason for this declining
predictive power is the increasing use of share buybacks as an
alternative means for capital distribution, which reduces the
contribution of dividend yield to total return
{[}robertson2006dividends{]}.

On the other hand, practitioner-oriented literature focuses on the
long-term returns of systematic dividend portfolios. One popular
strategy is the ``Dogs of the Dow (DOD),'' which involves creating a
portfolio of the top 10 highest-paying dividend stocks on the Dow Jones
Industrial Index at the beginning of the year based on the dividends
paid in the previous 12 months. The portfolio is held for 12 months, and
the process is repeated annually. Various studies have examined the DOD
strategy or similar high-yield dividend strategies in different time
periods and regions, consistently showing superior risk-adjusted returns
compared to the market index. Examples of such studies include Lemmon \&
Nguyen (\protect\hyperlink{ref-lemmon2015dividend}{2015}) in Hong Kong,
Brzeszczyński \& Gajdka
(\protect\hyperlink{ref-brzeszczynski2007dividend}{2007}) in Poland,
Visscher \& Filbeck (\protect\hyperlink{ref-visscher2003dividend}{2003})
in Canada, Filbeck \& Visscher
(\protect\hyperlink{ref-filbeck1997}{1997}) in Britian, and Wang,
Larsen, Ainina, Akhbari \& Gressis
(\protect\hyperlink{ref-wang2011dogs}{2011}) in China. More recently,
Filbeck, Holzhauer \& Zhao
(\protect\hyperlink{ref-filbeck2017dividend}{2017}) investigated the
performance of DOD against a high-yield portfolio of Fortune Most
Desired Companies (MAC) compared to the Dow Jones Industrial Average and
the S\&P 500. The study found significantly higher risk-adjusted returns
for the DOD strategy.

\begin{itemize}
\tightlist
\item
  Look at papers that investigate dividend signalling by sector.
\end{itemize}

\newpage

\hypertarget{references}{%
\section*{References}\label{references}}
\addcontentsline{toc}{section}{References}

\hypertarget{refs}{}
\begin{CSLReferences}{1}{0}
\leavevmode\vadjust pre{\hypertarget{ref-al2018revisiting}{}}%
Al-Najjar, B. \& Kilincarslan, E. 2018. Revisiting firm-specific
determinants of dividend policy: Evidence from turkey. \emph{Economic
issues}. 23(1):3--34.

\leavevmode\vadjust pre{\hypertarget{ref-ang2007stock}{}}%
Ang, A. \& Bekaert, G. 2007. Stock return predictability: Is it there?
\emph{The Review of Financial Studies}. 20(3):651--707.

\leavevmode\vadjust pre{\hypertarget{ref-baker2009understanding}{}}%
Baker, H.K. \& Powell, G. 2009. \emph{Understanding financial
management: A practical guide}. John Wiley \& Sons.

\leavevmode\vadjust pre{\hypertarget{ref-baker1999corporate}{}}%
Baker, H.K. \& Powell, G.E. 1999. How corporate managers view dividend
policy. \emph{Quarterly Journal of Business and Economics}. 17--35.

\leavevmode\vadjust pre{\hypertarget{ref-bhattacharyya2007dividend}{}}%
Bhattacharyya, N. 2007. Dividend policy: A review. \emph{Managerial
Finance}. 33(1):4--13.

\leavevmode\vadjust pre{\hypertarget{ref-black1996dividend}{}}%
Black, F. 1996. The dividend puzzle. \emph{Journal of Portfolio
Management}. 8.

\leavevmode\vadjust pre{\hypertarget{ref-brzeszczynski2007dividend}{}}%
Brzeszczyński, J. \& Gajdka, J. 2007. Dividend-driven trading
strategies: Evidence from the warsaw stock exchange. \emph{International
Advances in Economic Research}. 13:285--300.

\leavevmode\vadjust pre{\hypertarget{ref-cornell2014dividend}{}}%
Cornell, B. 2014. Dividend-price ratios and stock returns: International
evidence. \emph{Journal of Portfolio management}. 40(2):122.

\leavevmode\vadjust pre{\hypertarget{ref-deangelo2006irrelevance}{}}%
DeAngelo, H. \& DeAngelo, L. 2006. The irrelevance of the MM dividend
irrelevance theorem. \emph{Journal of financial economics}.
79(2):293--315.

\leavevmode\vadjust pre{\hypertarget{ref-fama1988permanent}{}}%
Fama, E.F. \& French, K.R. 1988. Permanent and temporary components of
stock prices. \emph{Journal of political Economy}. 96(2):246--273.

\leavevmode\vadjust pre{\hypertarget{ref-filbeck1997}{}}%
Filbeck, G. \& Visscher, S. 1997. Dividend yield strategies in the
british stock market. \emph{The European Journal of Finance}.
3(4):277--289.

\leavevmode\vadjust pre{\hypertarget{ref-filbeck2017dividend}{}}%
Filbeck, G., Holzhauer, H.M. \& Zhao, X. 2017. Dividend-yield
strategies: A new breed of dogs. \emph{The Journal of Investing}.
26(2):26--47.

\leavevmode\vadjust pre{\hypertarget{ref-gordon1963optimal}{}}%
Gordon, M.J. 1963. Optimal investment and financing policy. \emph{The
Journal of finance}. 18(2):264--272.

\leavevmode\vadjust pre{\hypertarget{ref-hussainey2011dividend}{}}%
Hussainey, K., Mgbame, C.O. \& Chijoke-Mgbame, A.M. 2011. Dividend
policy and share price volatility: UK evidence. \emph{The Journal of
risk finance}. 12(1):57--68.

\leavevmode\vadjust pre{\hypertarget{ref-jensen1976theory}{}}%
Jensen, M.C. \& Meckling, W.H. 1976. Theory of the firm: Managerial
behavior, agency costs and ownership structure. \emph{Journal of
financial economics}. 3(4):305--360.

\leavevmode\vadjust pre{\hypertarget{ref-10.2307ux2f3694818}{}}%
Koch, A.S. \& Sun, A.X. 2004. Dividend changes and the persistence of
past earnings changes. \emph{The Journal of Finance}. 59(5):2093--2116.

\leavevmode\vadjust pre{\hypertarget{ref-lemmon2015dividend}{}}%
Lemmon, M.L. \& Nguyen, T. 2015. Dividend yields and stock returns in
hong kong. \emph{Managerial Finance}. 41(2):164--181.

\leavevmode\vadjust pre{\hypertarget{ref-lintner1956distribution}{}}%
Lintner, J. 1956. Distribution of incomes of corporations among
dividends, retained earnings, and taxes. \emph{The American economic
review}. 46(2):97--113.

\leavevmode\vadjust pre{\hypertarget{ref-maio2015dividend}{}}%
Maio, P. \& Santa-Clara, P. 2015. Dividend yields, dividend growth, and
return predictability in the cross section of stocks. \emph{Journal of
Financial and Quantitative Analysis}. 50(1-2):33--60.

\leavevmode\vadjust pre{\hypertarget{ref-manconi2014buybacks}{}}%
Manconi, A., Peyer, U. \& Vermaelen, T. 2014. Buybacks around the world.
\emph{European Corporate Governance Institute (ECGI)-Finance Working
Paper}. 436.

\leavevmode\vadjust pre{\hypertarget{ref-masum2014dividend}{}}%
Masum, A. 2014. Dividend policy and its impact on stock price--a study
on commercial banks listed in dhaka stock exchange. \emph{Global
disclosure of Economics and Business}. 3(1).

\leavevmode\vadjust pre{\hypertarget{ref-miller1985dividend}{}}%
Miller, M.H. \& Rock, K. 1985. Dividend policy under asymmetric
information. \emph{The Journal of finance}. 40(4):1031--1051.

\leavevmode\vadjust pre{\hypertarget{ref-rangvid2014dividend}{}}%
Rangvid, J., Schmeling, M. \& Schrimpf, A. n.d. Dividend predictability
around the world. \emph{Journal of Financial and Quantitative Analysis}.
49(5-6):1255--1277.

\leavevmode\vadjust pre{\hypertarget{ref-robertson2006}{}}%
Robertson, D. \& Wright, S. 2006. Dividends, total cash flow to
shareholders, and predictive return regressions. \emph{Review of
Economics and Statistics}. 88(1):91--99.

\leavevmode\vadjust pre{\hypertarget{ref-suwanna2012impacts}{}}%
Suwanna, T. 2012. Impacts of dividend announcement on stock return.
\emph{Procedia-Social and Behavioral Sciences}. 40:721--725.

\leavevmode\vadjust pre{\hypertarget{ref-van2013advanced}{}}%
Van Deventer, D.R., Imai, K. \& Mesler, M. 2013. \emph{Advanced
financial risk management: Tools and techniques for integrated credit
risk and interest rate risk management}. John Wiley \& Sons.

\leavevmode\vadjust pre{\hypertarget{ref-vijayakumar2010effect}{}}%
Vijayakumar, A. 2010. Effect of financial performance on share prices in
the indian corporate sector: An empirical study. \emph{Management and
Labour Studies}. 35(3):369--381.

\leavevmode\vadjust pre{\hypertarget{ref-visscher2003dividend}{}}%
Visscher, S. \& Filbeck, G. 2003. Dividend-yield strategies in the
canadian stock market. \emph{Financial Analysts Journal}. 59(1):99--106.

\leavevmode\vadjust pre{\hypertarget{ref-wang2011dogs}{}}%
Wang, C., Larsen, J.E., Ainina, M.F., Akhbari, M.L. \& Gressis, N. 2011.
The dogs of the dow in china. \emph{International Journal of Business
and Social Science}. 2(18).

\leavevmode\vadjust pre{\hypertarget{ref-wesson2014market}{}}%
Wesson, N., Muller, C. \& Ward, M. 2014. Market underreaction to open
market share repurchases on the JSE. \emph{South African Journal of
Business Management}. 45(4):59--69.

\end{CSLReferences}

\bibliography{Tex/ref}





\end{document}
