\documentclass[11pt,preprint, authoryear]{elsarticle}

\usepackage{lmodern}
%%%% My spacing
\usepackage{setspace}
\setstretch{1}
\DeclareMathSizes{12}{14}{10}{10}

% Wrap around which gives all figures included the [H] command, or places it "here". This can be tedious to code in Rmarkdown.
\usepackage{float}
\let\origfigure\figure
\let\endorigfigure\endfigure
\renewenvironment{figure}[1][2] {
    \expandafter\origfigure\expandafter[H]
} {
    \endorigfigure
}

\let\origtable\table
\let\endorigtable\endtable
\renewenvironment{table}[1][2] {
    \expandafter\origtable\expandafter[H]
} {
    \endorigtable
}


\usepackage{ifxetex,ifluatex}
\usepackage{fixltx2e} % provides \textsubscript
\ifnum 0\ifxetex 1\fi\ifluatex 1\fi=0 % if pdftex
  \usepackage[T1]{fontenc}
  \usepackage[utf8]{inputenc}
\else % if luatex or xelatex
  \ifxetex
    \usepackage{mathspec}
    \usepackage{xltxtra,xunicode}
  \else
    \usepackage{fontspec}
  \fi
  \defaultfontfeatures{Mapping=tex-text,Scale=MatchLowercase}
  \newcommand{\euro}{€}
\fi

\usepackage{amssymb, amsmath, amsthm, amsfonts}

\def\bibsection{\section*{References}} %%% Make "References" appear before bibliography


\usepackage[round]{natbib}

\usepackage{longtable}
\usepackage[margin=2.3cm,bottom=2cm,top=2.5cm, includefoot]{geometry}
\usepackage{fancyhdr}
\usepackage[bottom, hang, flushmargin]{footmisc}
\usepackage{graphicx}
\numberwithin{equation}{section}
\numberwithin{figure}{section}
\numberwithin{table}{section}
\setlength{\parindent}{0cm}
\setlength{\parskip}{1.3ex plus 0.5ex minus 0.3ex}
\usepackage{textcomp}
\renewcommand{\headrulewidth}{0.2pt}
\renewcommand{\footrulewidth}{0.3pt}

\usepackage{array}
\newcolumntype{x}[1]{>{\centering\arraybackslash\hspace{0pt}}p{#1}}

%%%%  Remove the "preprint submitted to" part. Don't worry about this either, it just looks better without it:
\makeatletter
\def\ps@pprintTitle{%
  \let\@oddhead\@empty
  \let\@evenhead\@empty
  \let\@oddfoot\@empty
  \let\@evenfoot\@oddfoot
}
\makeatother

 \def\tightlist{} % This allows for subbullets!

\usepackage{hyperref}
\hypersetup{breaklinks=true,
            bookmarks=true,
            colorlinks=true,
            citecolor=blue,
            urlcolor=blue,
            linkcolor=blue,
            pdfborder={0 0 0}}


% The following packages allow huxtable to work:
\usepackage{siunitx}
\usepackage{multirow}
\usepackage{hhline}
\usepackage{calc}
\usepackage{tabularx}
\usepackage{booktabs}
\usepackage{caption}


\newenvironment{columns}[1][]{}{}

\newenvironment{column}[1]{\begin{minipage}{#1}\ignorespaces}{%
\end{minipage}
\ifhmode\unskip\fi
\aftergroup\useignorespacesandallpars}

\def\useignorespacesandallpars#1\ignorespaces\fi{%
#1\fi\ignorespacesandallpars}

\makeatletter
\def\ignorespacesandallpars{%
  \@ifnextchar\par
    {\expandafter\ignorespacesandallpars\@gobble}%
    {}%
}
\makeatother

\newenvironment{CSLReferences}[2]{%
}

\urlstyle{same}  % don't use monospace font for urls
\setlength{\parindent}{0pt}
\setlength{\parskip}{6pt plus 2pt minus 1pt}
\setlength{\emergencystretch}{3em}  % prevent overfull lines
\setcounter{secnumdepth}{5}

%%% Use protect on footnotes to avoid problems with footnotes in titles
\let\rmarkdownfootnote\footnote%
\def\footnote{\protect\rmarkdownfootnote}
\IfFileExists{upquote.sty}{\usepackage{upquote}}{}

%%% Include extra packages specified by user
\usepackage{booktabs}
\usepackage{longtable}
\usepackage{array}
\usepackage{multirow}
\usepackage{wrapfig}
\usepackage{float}
\usepackage{colortbl}
\usepackage{pdflscape}
\usepackage{tabu}
\usepackage{threeparttable}
\usepackage{threeparttablex}
\usepackage[normalem]{ulem}
\usepackage{makecell}
\usepackage{xcolor}
\usepackage{caption}

%%% Hard setting column skips for reports - this ensures greater consistency and control over the length settings in the document.
%% page layout
%% paragraphs
\setlength{\baselineskip}{12pt plus 0pt minus 0pt}
\setlength{\parskip}{12pt plus 0pt minus 0pt}
\setlength{\parindent}{0pt plus 0pt minus 0pt}
%% floats
\setlength{\floatsep}{12pt plus 0 pt minus 0pt}
\setlength{\textfloatsep}{20pt plus 0pt minus 0pt}
\setlength{\intextsep}{14pt plus 0pt minus 0pt}
\setlength{\dbltextfloatsep}{20pt plus 0pt minus 0pt}
\setlength{\dblfloatsep}{14pt plus 0pt minus 0pt}
%% maths
\setlength{\abovedisplayskip}{12pt plus 0pt minus 0pt}
\setlength{\belowdisplayskip}{12pt plus 0pt minus 0pt}
%% lists
\setlength{\topsep}{10pt plus 0pt minus 0pt}
\setlength{\partopsep}{3pt plus 0pt minus 0pt}
\setlength{\itemsep}{5pt plus 0pt minus 0pt}
\setlength{\labelsep}{8mm plus 0mm minus 0mm}
\setlength{\parsep}{\the\parskip}
\setlength{\listparindent}{\the\parindent}
%% verbatim
\setlength{\fboxsep}{5pt plus 0pt minus 0pt}



\begin{document}



\begin{frontmatter}  %

\title{Much Ado About Dividends: A Global Perspective}

% Set to FALSE if wanting to remove title (for submission)




\author[Add1]{Gabriel Rambanapasi}
\ead{gabriel.rams44@gmail.com}





\address[Add1]{Stellenbosch University, Cape Town, South Africa}

\cortext[cor]{Corresponding author: Gabriel Rambanapasi}

\begin{abstract}
\small{
Past literature shows great division maongst academics on the relevance
of dividend portfolios. We acknowlege that dividend paying companies do
convey information about management quality and persistence of
cashflows. We investigated the return signalling cue in globally traded
portfolios constructed using high yield or dividend growth signals. We
find that dividend portfolios offer downside protection in periods of
high market volatility. Emerging market portfolio give positive and
higher returns than adavanced economy dividend portfolios.
}
\end{abstract}

\vspace{1cm}





\vspace{0.5cm}

\end{frontmatter}

\setcounter{footnote}{0}



%________________________
% Header and Footers
%%%%%%%%%%%%%%%%%%%%%%%%%%%%%%%%%
\pagestyle{fancy}
\chead{}
\rhead{}
\lfoot{}
\rfoot{\footnotesize Page \thepage}
\lhead{}
%\rfoot{\footnotesize Page \thepage } % "e.g. Page 2"
\cfoot{}

%\setlength\headheight{30pt}
%%%%%%%%%%%%%%%%%%%%%%%%%%%%%%%%%
%________________________

\headsep 35pt % So that header does not go over title




\hypertarget{introduction}{%
\section{Introduction}\label{introduction}}

\hypertarget{literature-review}{%
\section{\texorpdfstring{Literature Review
\label{Literature Review}}{Literature Review }}\label{literature-review}}

\hypertarget{theorectical-underpinning-of-dividends}{%
\subsection{Theorectical underpinning of
dividends}\label{theorectical-underpinning-of-dividends}}

Dividends constitute a form of capital distribution by corporations
towards shareholders. They exist in various forms, such as cash, stock,
liquidating, scrip, or property dividends
(\protect\hyperlink{ref-baker2009understanding}{\textbf{baker2009understanding?}}),
of which cash dividends and share repurchases being the most commonly
used in practice. Within cash dividends, regular dividends are widely
used by corporations and payment frequency across jurisdictions. The
decision to issue dividends is typically made by the board of directors,
and approved by shareholders, however practiced more in Europe and less
so in the United States. The payout policy policy of a corporation,
which are guiding principles for management and board of directors
towards capital distributions considers company investment and is
closely watched by investors and analysts. As such, management strives
to grow or maintain a certain level of dividend payouts as this signals
firm growth and investors share of profitability in the company. Various
liertature has covered the effect of dividend announcements before and
after ex -dividend dates. Figure 1 shows a clear and direct relation
with a decrease in share value to the proportionate to the dividend
announcement.

Given the apparent decrease in shareholder value, the logical question
has encouraged a long running debate on dividend relevance and
irrelevance. In 1961, Miller \& Rock
(\protect\hyperlink{ref-miller1985dividend}{1985}) opined that dividends
are irrelevant (MM theory), he argued that shareholders are indifferent
to dividend payments, thus implying that there is no optimal dividend
policy and that all dividend policies are equally good and payments of
dividends could easily be reinvested in shares and make no difference to
share holder wealth. However, the MM theorem fails to consider
real-world market imperfections that may give relevance to dividend
payments. The bird in the arguments opposes the MM theory, suggesting
that investor would prefer to receive less risky cash flow in the form
of dividends instead of potential capital gains at some point in the
future (\protect\hyperlink{ref-gordon1962}{Gordon, 1962}). This
permeates to the cost of equity, since dividends are less risky,
companies that issue more dividends should have higher share prices.
However, propoents of the MM theory contend this suggesting the risk of
future cash flow is affected by the payment of dividend, leading to
negative effects on share prices after the ex-dividend date. The
dividend puzzle considers real world constraints and gives an
interesting take on its relevance and irrelevance, by suggesting that
dividends reduce equity value and make investors worse off; however, are
a reward to investors who bear the risk associated with their
investments as it provides an additional source of return on investment
from a share Black (\protect\hyperlink{ref-black1996dividend}{1996}).
Various literature has made convincing arguments for corporations to pay
dividends which include Tax considerations, dividend signalling and
agency costs in issuing dividends .

Taxe considerations argue in favor for dividend relevance. Across
jurisdiction dividends have different tax treatments to capital gains
and often tax at a higher income tax rate, thus investors that have
higher tax rates choose stocks with lower dividend payouts and
tranversly pushes up the stock price, this is called the clientele
effect
(\protect\hyperlink{ref-baker2009understanding}{\textbf{baker2009understanding?}}).
Proponents of the MM theory suggest that the client effect causes major
substitution effect, suggesting that if companies change their dividend
policy, investors with preferential tax treatment will simply allocate
more capital to that stock and those out of favor will sell their
shares. Given the large number of investors versus listed companies the
process is instantaneously causing a net zero effect on
prices(\protect\hyperlink{ref-baker2009understanding}{\textbf{baker2009understanding?}}).
Second, flotation costs refer to the opportunity costs incurred by a
firm when paying dividends. Through distributing dividends, companies
forego opportunities to expand their operations using retained earnings.
In a world without flotation costs, as suggested by the MM theorem,
management would be indifferent between issuing dividends and borrowing
from the market thus have no effect on shares prices. However, in
reality, external financing comes at a higher cost, leading to
trade-offs in dividend policy decisions and ultimately share prices.

Information asymmetry between shareholders and managers is another
factor that gives relevance to dividend payments. Managers of businesses
have greater knowledge of operations thus value of a business at any
given point more than shareholders. As such, investors rely on dividend
announcements to assess a company's valuation. Dividend signaling
conveys information about the company's quality Al-Najjar \&
Kilincarslan (\protect\hyperlink{ref-al2018revisiting}{2018}) and Baker
\& Powell (\protect\hyperlink{ref-baker1999corporate}{1999}). Investors
compare dividend announcements to historical levels while considering
company fundamentals. However, there is a risk of manipulation by
management, making the dividend signal imperfect for determining share
prices. Principal agency issues may give another reason for issuance of
dividends. The free cash flow hypothesis suggests that dividend payments
force management to raise capital from external sources, which increases
borrowing costs and scrutiny from capital markets. This, in turn,
reduces management's ability to make sub optimal investments and
aligning management and shareholder objectives
(\protect\hyperlink{ref-baker2009understanding}{\textbf{baker2009understanding?}}).
Supporters of this theory ascertain that dividends payments by the
mechanism encourage good business practices.

\hypertarget{dividend-signals-theoreactial-and-practical-evidence}{%
\subsection{Dividend signals, theoreactial and practical
evidence}\label{dividend-signals-theoreactial-and-practical-evidence}}

The various methods of capital distributions have varying impact on
financial statements which is summarized in Table of the appendix. From
the perceptive of an investor or analyst the dividend yield metric helps
show the additional return dividends paying securities could add to a
portfolio. Consider \ref{eq1} that describes the fundamentals that
influence the dividend yield. Assuming a constant payout ratio, dividend
yield is a function of earnings yield. shows the correlation between DY
and Price overtime for various securities. Various studies have
identified a predictive power of dividend yield thus confirm the
existence of a value signal. Also, another signal for dividends is
dividend growth per share for corporations, and unlike the dividend
yield, it is not affected by price but maintain properties that allow
for inference into management quality. As managemnet is aware of the
signalling effect of dividends, this may induce the value trap, that
forces management to continually increase dividends to maintain a
certain valuation. However such companies are more vulnerable to facing
financial distress.

\(Dividend Yield =Earnings Per Share/Price*Dividend Payout Ratio\)
\label{eq1}

Cash dividends, although widely used, are not as tax-efficient as share
buybacks. In this form of capital redistribution, a firm exchanges
assets for outstanding shares, which shrinks the company's assets by the
amount of cash paid out. This action too reduces both its borrowing base
and the shareholders' aggregate equity
(\protect\hyperlink{ref-baker2009understanding}{\textbf{baker2009understanding?}}).
A clear benefit to the company is that it is more flexible when compared
to the rigid dividend payout structures. To most higher net worth
investors, tax benefits in the form of lower capital gains taxes result
in greater preference for share buybacks. Surprisingly, their adoption
has been relatively slow in some emerging economies. According to a
study by Wesson, Muller \& Ward
(\protect\hyperlink{ref-wesson2014market}{2014}), there were only 195
open market share repurchases announced in South Africa from 1999 to
2009. In comparison, Manconi, Peyer \& Vermaelen
(\protect\hyperlink{ref-manconi2014buybacks}{2014}) estimated that share
repurchases constituted approximately 58\% of total announcements in the
United States, 15\% in Canada, and 11\% in Japan over the same period,
indicative of a significant disparity in the adoption of share buybacks
across the world, despite their popularity in the United States.

Dividend payments and growth in dividends per share provides a return
cue and over the years studies on dividend signaling studies can be
categorized into academic and practitioner-oriented studies. Academic
studies, such as Fama \& French
(\protect\hyperlink{ref-fama1988permanent}{1988}), found a positive
correlation between increasing predictive power and longer forecast
horizons. However, subsequent studies like Ang \& Bekaert
(\protect\hyperlink{ref-ang2007stock}{2007}) found no evidence of
long-term predictability in stock returns when considering finite sample
influence. This suggests that dividend yield may not be a reliable
predictor of subsequent returns. One possible reason for this declining
predictive power is the increasing use of share buybacks as an
alternative means for capital distribution, which reduces the
contribution of dividend yield to total return
(\protect\hyperlink{ref-robertson2006}{Robertson \& Wright, 2006}).

On the other hand, practitioner-oriented literature focuses on the
long-term returns of systematic dividend portfolios. One popular
strategy is the ``Dogs of the Dow (DOD),'' which involves constructing a
portfolio of the top 10 highest-paying dividend stocks on the Dow Jones
Industrial Index at the beginning of the year based on the dividends
paid in the previous 12 months, therefore this entail deploying a high
yield strategy (\protect\hyperlink{ref-mcqueen1997does}{McQueen, Shields
\& Thorley, 1997}). Various studies have examined the DOD strategy or
similar high-yield dividend strategies in different time periods and
regions, consistently showing superior risk-adjusted returns compared to
the market index. Examples of such studies include Lemmon \& Nguyen
(\protect\hyperlink{ref-lemmon2015dividend}{2015}) in Hong Kong
Brzeszczyński \& Gajdka
(\protect\hyperlink{ref-brzeszczynski2007dividend}{2007}) in Poland,
Visscher \& Filbeck (\protect\hyperlink{ref-visscher2003dividend}{2003})
in Canada, Filbeck \& Visscher
(\protect\hyperlink{ref-filbeck1997}{1997}) in Britian, and Wang,
Larsen, Ainina, Akhbari \& Gressis
(\protect\hyperlink{ref-wang2011dogs}{2011}) in China. More recently,
Filbeck, Holzhauer \& Zhao
(\protect\hyperlink{ref-filbeck2017dividend}{2017}) investigated the
performance of DOD against a high-yield portfolio of Fortune Most
Desired Companies (MAC) compared to the Dow Jones Industrial Average and
the S\&P 500. The study found significantly higher risk-adjusted returns
for the DOD strategy.

\newpage

\begin{longtable}{rlll}
\caption*{
{\large Global Dividend Portfolio Signalling Results}
} \\ 
\toprule
Year & Author & Signal & Methodology \\ 
\midrule
1997 & Filbeck & High Dividend Yield & NA \\ 
2001 & Da Silva & High Dividend Yield & NA \\ 
2003 & Visscher and Filbeck & Dividend Growth & NA \\ 
2007 & Brzeszczyński & Dividend Growth & NA \\ 
2007 & Fama and Eugene & High Dividend Yield & NA \\ 
2011 & Wang et al & High Dividend Yield & NA \\ 
2011 & Rennie & High Dividend Yield & NA \\ 
2015 & Lemmon & High Dividend Yield & Grouped high yield stock and tests for yield effect at portfolio level.  Fama-MacBeth methodology and tests for yield effect after controlling for known factors. \\ 
2017 & Filbeck & High Dividend Yield & NA \\ 
2017 & You & High Dividend Yield & NA \\ 
\bottomrule
\end{longtable}

\newpage

\hypertarget{methodology}{%
\section{Methodology}\label{methodology}}

\hypertarget{tax-considerations}{%
\subsection{Tax considerations}\label{tax-considerations}}

Portfolio theory was developed in a perfect world without friction. In
practice, frictions need to be considered and in portfolio construction
this often entails considering the effect of taxes on income and capital
gains as they can erode returns and significantly alter risks and return
characteristics of shares. The contribution of dividends and capital
gains to total return can lead to varying tax efficiencies for shares as
most jurisdictions imposed higher taxes than on capital gains. Therefore
shares with higher contribution of dividends will be less tax efficient
than those with a higher capital gains component and with timing most
jurisdictions tax dividends in the year that they are
receive\footnote{See Deloitte's tax guides and country highlights:
  \url{https://dits.deloitte.com/\#TaxGuides}}.

Jurisdictional laws can also affect the distribution of taxable returns
amongst shares depending on their class namely ordinary shares or
preferred shares. Preferred shares are viewed as a substitute for bonds
and income from preferred shares are often given tax at a lower rate
than those from dividends from ordinary shares.

We will not survey global tax regimes or incorporate all potential tax
complexities into the portfolio construction but assume a high level
commonalities exists amongst all jurisdictions this study uses. This is
a resonable assumption considering the summary of taxes on dividends and
capital gains from major economies. For simplicity, we will assume a
basic tax regime includes the key elements of investment-related taxes
that are representative of what a typical taxable asset owner of a
global portfolio will contend with. The proposed methodology to employ
on the dividend portfolios use the following methodology.

\(r_{a t}=p_d r_{p t}\left(1-t_d\right)+p_a r_{p t}\left(1-t_{c g}\right)\)

where \(r_{a t}\) the after tax return, \(p_d=\) the proportion of
\(r_{p t}\) attributed to dividend income, \(p_a=\) the proportion of
\(r_{p t}\) attributed to price appreciation, \(t_d=\) the dividend tax
rate and \(t_{c g}=\) the capital gains tax rate

\newpage

\hypertarget{portfolio-construction}{%
\subsection{Portfolio Construction}\label{portfolio-construction}}

First, we rebalance at the end of March and September and construct
fully invested, long only portfolios. On each re balancing date, we
first take the top 100 stocks by market capitalization (MC), and then
select the top quintile (20 stocks) based on the relevant signal scores
. We then apply 25bps trading costs to both buying and selling of
stocks, and (where applicable) replace delisted stocks with cash.

We will then use total return values, adjust for stock splits and other
distorting effects on prices to calculate portfolio returns. We also
carefully apply back-dated adjustments to dividends paid to accurately
arrive at on-the-day dividends and actual closing prices when
calculating our Dividend Yield and Dividend Per Share Growth measures.

We also apply at each rebalancing several portfolio optimization
routines (minimum variance, equal risk contribution (ERC), mean-variance
and max diversification measures. The optimization are constrained to
have minimum and maximum weight exposure of 0.5 and 1.5 times the equal
weighted alternative.With our quintile portfolios, this implies weights
ranging between 2.5\% and 7.5\%.

Following this we will construct back-tests on the subset of local data
for the choose dividend signal portfolios. The factor portfolios
considered are as follows

\hypertarget{data}{%
\section{Data}\label{data}}

\hypertarget{econometric-approach}{%
\section{Econometric Approach}\label{econometric-approach}}

\newpage

\hypertarget{references}{%
\section*{References}\label{references}}
\addcontentsline{toc}{section}{References}

\hypertarget{refs}{}
\begin{CSLReferences}{1}{0}
\leavevmode\vadjust pre{\hypertarget{ref-al2018revisiting}{}}%
Al-Najjar, B. \& Kilincarslan, E. 2018. Revisiting firm-specific
determinants of dividend policy: Evidence from turkey. \emph{Economic
issues}. 23(1):3--34.

\leavevmode\vadjust pre{\hypertarget{ref-ang2007stock}{}}%
Ang, A. \& Bekaert, G. 2007. Stock return predictability: Is it there?
\emph{The Review of Financial Studies}. 20(3):651--707.

\leavevmode\vadjust pre{\hypertarget{ref-baker1999corporate}{}}%
Baker, H.K. \& Powell, G.E. 1999. How corporate managers view dividend
policy. \emph{Quarterly Journal of Business and Economics}. 17--35.

\leavevmode\vadjust pre{\hypertarget{ref-bhattacharyya2007dividend}{}}%
Bhattacharyya, N. 2007. Dividend policy: A review. \emph{Managerial
Finance}. 33(1):4--13.

\leavevmode\vadjust pre{\hypertarget{ref-black1996dividend}{}}%
Black, F. 1996. The dividend puzzle. \emph{Journal of Portfolio
Management}. 8.

\leavevmode\vadjust pre{\hypertarget{ref-brzeszczynski2007dividend}{}}%
Brzeszczyński, J. \& Gajdka, J. 2007. Dividend-driven trading
strategies: Evidence from the warsaw stock exchange. \emph{International
Advances in Economic Research}. 13:285--300.

\leavevmode\vadjust pre{\hypertarget{ref-cornell2014dividend}{}}%
Cornell, B. 2014. Dividend-price ratios and stock returns: International
evidence. \emph{Journal of Portfolio management}. 40(2):122.

\leavevmode\vadjust pre{\hypertarget{ref-deangelo2006irrelevance}{}}%
DeAngelo, H. \& DeAngelo, L. 2006. The irrelevance of the MM dividend
irrelevance theorem. \emph{Journal of financial economics}.
79(2):293--315.

\leavevmode\vadjust pre{\hypertarget{ref-fama1988permanent}{}}%
Fama, E.F. \& French, K.R. 1988. Permanent and temporary components of
stock prices. \emph{Journal of political Economy}. 96(2):246--273.

\leavevmode\vadjust pre{\hypertarget{ref-filbeck1997}{}}%
Filbeck, G. \& Visscher, S. 1997. Dividend yield strategies in the
british stock market. \emph{The European Journal of Finance}.
3(4):277--289.

\leavevmode\vadjust pre{\hypertarget{ref-filbeck2017dividend}{}}%
Filbeck, G., Holzhauer, H.M. \& Zhao, X. 2017. Dividend-yield
strategies: A new breed of dogs. \emph{The Journal of Investing}.
26(2):26--47.

\leavevmode\vadjust pre{\hypertarget{ref-gordon1962}{}}%
Gordon, M.J. 1962. The savings investment and valuation of a
corporation. \emph{The Review of Economics and Statistics}. 37--51.

\leavevmode\vadjust pre{\hypertarget{ref-gordon1963optimal}{}}%
Gordon, M.J. 1963. Optimal investment and financing policy. \emph{The
Journal of finance}. 18(2):264--272.

\leavevmode\vadjust pre{\hypertarget{ref-hussainey2011dividend}{}}%
Hussainey, K., Mgbame, C.O. \& Chijoke-Mgbame, A.M. 2011. Dividend
policy and share price volatility: UK evidence. \emph{The Journal of
risk finance}. 12(1):57--68.

\leavevmode\vadjust pre{\hypertarget{ref-jensen1976theory}{}}%
Jensen, M.C. \& Meckling, W.H. 1976. Theory of the firm: Managerial
behavior, agency costs and ownership structure. \emph{Journal of
financial economics}. 3(4):305--360.

\leavevmode\vadjust pre{\hypertarget{ref-10.2307ux2f3694818}{}}%
Koch, A.S. \& Sun, A.X. 2004. Dividend changes and the persistence of
past earnings changes. \emph{The Journal of Finance}. 59(5):2093--2116.

\leavevmode\vadjust pre{\hypertarget{ref-lemmon2015dividend}{}}%
Lemmon, M.L. \& Nguyen, T. 2015. Dividend yields and stock returns in
hong kong. \emph{Managerial Finance}. 41(2):164--181.

\leavevmode\vadjust pre{\hypertarget{ref-lintner1956distribution}{}}%
Lintner, J. 1956. Distribution of incomes of corporations among
dividends, retained earnings, and taxes. \emph{The American economic
review}. 46(2):97--113.

\leavevmode\vadjust pre{\hypertarget{ref-maio2015dividend}{}}%
Maio, P. \& Santa-Clara, P. 2015. Dividend yields, dividend growth, and
return predictability in the cross section of stocks. \emph{Journal of
Financial and Quantitative Analysis}. 50(1-2):33--60.

\leavevmode\vadjust pre{\hypertarget{ref-manconi2014buybacks}{}}%
Manconi, A., Peyer, U. \& Vermaelen, T. 2014. Buybacks around the world.
\emph{European Corporate Governance Institute (ECGI)-Finance Working
Paper}. 436.

\leavevmode\vadjust pre{\hypertarget{ref-masum2014dividend}{}}%
Masum, A. 2014. Dividend policy and its impact on stock price--a study
on commercial banks listed in dhaka stock exchange. \emph{Global
disclosure of Economics and Business}. 3(1).

\leavevmode\vadjust pre{\hypertarget{ref-mcqueen1997does}{}}%
McQueen, G., Shields, K. \& Thorley, S.R. 1997. Does the {``dow-10
investment strategy''} beat the dow statistically and economically?
\emph{Financial Analysts Journal}. 53(4):66--72.

\leavevmode\vadjust pre{\hypertarget{ref-miller1985dividend}{}}%
Miller, M.H. \& Rock, K. 1985. Dividend policy under asymmetric
information. \emph{The Journal of finance}. 40(4):1031--1051.

\leavevmode\vadjust pre{\hypertarget{ref-rangvid2014dividend}{}}%
Rangvid, J., Schmeling, M. \& Schrimpf, A. n.d. Dividend predictability
around the world. \emph{Journal of Financial and Quantitative Analysis}.
49(5-6):1255--1277.

\leavevmode\vadjust pre{\hypertarget{ref-robertson2006}{}}%
Robertson, D. \& Wright, S. 2006. Dividends, total cash flow to
shareholders, and predictive return regressions. \emph{Review of
Economics and Statistics}. 88(1):91--99.

\leavevmode\vadjust pre{\hypertarget{ref-suwanna2012impacts}{}}%
Suwanna, T. 2012. Impacts of dividend announcement on stock return.
\emph{Procedia-Social and Behavioral Sciences}. 40:721--725.

\leavevmode\vadjust pre{\hypertarget{ref-van2013advanced}{}}%
Van Deventer, D.R., Imai, K. \& Mesler, M. 2013. \emph{Advanced
financial risk management: Tools and techniques for integrated credit
risk and interest rate risk management}. John Wiley \& Sons.

\leavevmode\vadjust pre{\hypertarget{ref-vijayakumar2010effect}{}}%
Vijayakumar, A. 2010. Effect of financial performance on share prices in
the indian corporate sector: An empirical study. \emph{Management and
Labour Studies}. 35(3):369--381.

\leavevmode\vadjust pre{\hypertarget{ref-visscher2003dividend}{}}%
Visscher, S. \& Filbeck, G. 2003. Dividend-yield strategies in the
canadian stock market. \emph{Financial Analysts Journal}. 59(1):99--106.

\leavevmode\vadjust pre{\hypertarget{ref-wang2011dogs}{}}%
Wang, C., Larsen, J.E., Ainina, M.F., Akhbari, M.L. \& Gressis, N. 2011.
The dogs of the dow in china. \emph{International Journal of Business
and Social Science}. 2(18).

\leavevmode\vadjust pre{\hypertarget{ref-wesson2014market}{}}%
Wesson, N., Muller, C. \& Ward, M. 2014. Market underreaction to open
market share repurchases on the JSE. \emph{South African Journal of
Business Management}. 45(4):59--69.

\end{CSLReferences}

\bibliography{Tex/ref}





\end{document}
