\documentclass[11pt,preprint, authoryear]{elsarticle}

\usepackage{lmodern}
%%%% My spacing
\usepackage{setspace}
\setstretch{1}
\DeclareMathSizes{12}{14}{10}{10}

% Wrap around which gives all figures included the [H] command, or places it "here". This can be tedious to code in Rmarkdown.
\usepackage{float}
\let\origfigure\figure
\let\endorigfigure\endfigure
\renewenvironment{figure}[1][2] {
    \expandafter\origfigure\expandafter[H]
} {
    \endorigfigure
}

\let\origtable\table
\let\endorigtable\endtable
\renewenvironment{table}[1][2] {
    \expandafter\origtable\expandafter[H]
} {
    \endorigtable
}


\usepackage{ifxetex,ifluatex}
\usepackage{fixltx2e} % provides \textsubscript
\ifnum 0\ifxetex 1\fi\ifluatex 1\fi=0 % if pdftex
  \usepackage[T1]{fontenc}
  \usepackage[utf8]{inputenc}
\else % if luatex or xelatex
  \ifxetex
    \usepackage{mathspec}
    \usepackage{xltxtra,xunicode}
  \else
    \usepackage{fontspec}
  \fi
  \defaultfontfeatures{Mapping=tex-text,Scale=MatchLowercase}
  \newcommand{\euro}{€}
\fi

\usepackage{amssymb, amsmath, amsthm, amsfonts}

\def\bibsection{\section*{References}} %%% Make "References" appear before bibliography


\usepackage[round]{natbib}

\usepackage{longtable}
\usepackage[margin=2.3cm,bottom=2cm,top=2.5cm, includefoot]{geometry}
\usepackage{fancyhdr}
\usepackage[bottom, hang, flushmargin]{footmisc}
\usepackage{graphicx}
\numberwithin{equation}{section}
\numberwithin{figure}{section}
\numberwithin{table}{section}
\setlength{\parindent}{0cm}
\setlength{\parskip}{1.3ex plus 0.5ex minus 0.3ex}
\usepackage{textcomp}
\renewcommand{\headrulewidth}{0.2pt}
\renewcommand{\footrulewidth}{0.3pt}

\usepackage{array}
\newcolumntype{x}[1]{>{\centering\arraybackslash\hspace{0pt}}p{#1}}

%%%%  Remove the "preprint submitted to" part. Don't worry about this either, it just looks better without it:
\makeatletter
\def\ps@pprintTitle{%
  \let\@oddhead\@empty
  \let\@evenhead\@empty
  \let\@oddfoot\@empty
  \let\@evenfoot\@oddfoot
}
\makeatother

 \def\tightlist{} % This allows for subbullets!

\usepackage{hyperref}
\hypersetup{breaklinks=true,
            bookmarks=true,
            colorlinks=true,
            citecolor=blue,
            urlcolor=blue,
            linkcolor=blue,
            pdfborder={0 0 0}}


% The following packages allow huxtable to work:
\usepackage{siunitx}
\usepackage{multirow}
\usepackage{hhline}
\usepackage{calc}
\usepackage{tabularx}
\usepackage{booktabs}
\usepackage{caption}


\newenvironment{columns}[1][]{}{}

\newenvironment{column}[1]{\begin{minipage}{#1}\ignorespaces}{%
\end{minipage}
\ifhmode\unskip\fi
\aftergroup\useignorespacesandallpars}

\def\useignorespacesandallpars#1\ignorespaces\fi{%
#1\fi\ignorespacesandallpars}

\makeatletter
\def\ignorespacesandallpars{%
  \@ifnextchar\par
    {\expandafter\ignorespacesandallpars\@gobble}%
    {}%
}
\makeatother

\newenvironment{CSLReferences}[2]{%
}

\urlstyle{same}  % don't use monospace font for urls
\setlength{\parindent}{0pt}
\setlength{\parskip}{6pt plus 2pt minus 1pt}
\setlength{\emergencystretch}{3em}  % prevent overfull lines
\setcounter{secnumdepth}{5}

%%% Use protect on footnotes to avoid problems with footnotes in titles
\let\rmarkdownfootnote\footnote%
\def\footnote{\protect\rmarkdownfootnote}
\IfFileExists{upquote.sty}{\usepackage{upquote}}{}

%%% Include extra packages specified by user
\usepackage{booktabs}
\usepackage{longtable}
\usepackage{array}
\usepackage{multirow}
\usepackage{wrapfig}
\usepackage{float}
\usepackage{colortbl}
\usepackage{pdflscape}
\usepackage{tabu}
\usepackage{threeparttable}
\usepackage{threeparttablex}
\usepackage[normalem]{ulem}
\usepackage{makecell}
\usepackage{xcolor}

%%% Hard setting column skips for reports - this ensures greater consistency and control over the length settings in the document.
%% page layout
%% paragraphs
\setlength{\baselineskip}{12pt plus 0pt minus 0pt}
\setlength{\parskip}{12pt plus 0pt minus 0pt}
\setlength{\parindent}{0pt plus 0pt minus 0pt}
%% floats
\setlength{\floatsep}{12pt plus 0 pt minus 0pt}
\setlength{\textfloatsep}{20pt plus 0pt minus 0pt}
\setlength{\intextsep}{14pt plus 0pt minus 0pt}
\setlength{\dbltextfloatsep}{20pt plus 0pt minus 0pt}
\setlength{\dblfloatsep}{14pt plus 0pt minus 0pt}
%% maths
\setlength{\abovedisplayskip}{12pt plus 0pt minus 0pt}
\setlength{\belowdisplayskip}{12pt plus 0pt minus 0pt}
%% lists
\setlength{\topsep}{10pt plus 0pt minus 0pt}
\setlength{\partopsep}{3pt plus 0pt minus 0pt}
\setlength{\itemsep}{5pt plus 0pt minus 0pt}
\setlength{\labelsep}{8mm plus 0mm minus 0mm}
\setlength{\parsep}{\the\parskip}
\setlength{\listparindent}{\the\parindent}
%% verbatim
\setlength{\fboxsep}{5pt plus 0pt minus 0pt}



\begin{document}



\begin{frontmatter}  %

\title{Much Ado About Dividends}

% Set to FALSE if wanting to remove title (for submission)




\author[Add1]{Gabriel Rambanapasi}
\ead{gabriel.rams44@gmail.com}

\author[Add2]{Nico Katzke}
\ead{}




\address[Add1]{Stellenbosch University, Cape Town, South Africa}
\address[Add2]{Satrix, Cape Town, South Africa}

\cortext[cor]{Corresponding author: Gabriel Rambanapasi}

\begin{abstract}
\small{
This paper investigates dividends return predictive signals, focusing on
Dividend Yield (DY) and Dividend Growth (DG) signals, within both
domestic and international contexts. Over an extended investment
horizon, dividend portfolios consistently exhibit positive excess
returns, with notable variations across regional indexes. Notably, these
portfolios demonstrate capital protection during periods of heightened
market volatility, particularly the DY strategies. Surprisingly, South
African portfolios perform well during market turbulence, a departure
from the typical flight to safety. In contrast, during interest rate
cycles, all strategies perform optimally in low-interest-rate
environments. Information ratio analysis highlights that dividend
strategies do not uniformly maintain positive ratios, with varying
performance patterns across different regions. When integrated with
drawdown analysis, advanced economies exhibit lower systematic risk,
while South African and Emerging Market drawdowns suggest reduced
systematic risk in these markets. Within the South African context,
dividend portfolios display potential value during high volatility and
rising interest rate cycles, though P/E-based portfolios outperform them
in other scenarios. Consequently, dividend portfolios may not be an
ideal proxy for value. Investors with specific preferences may still
find value in these strategies, but for those constrained by investment
policy statements, DG portfolios offer practical, lower-volatility
alternatives to achieve returns akin to the market index. This study
contributes insights into dividend portfolio dynamics, enabling
investors to make informed choices based on their objectives and
constraints.
}
\end{abstract}

\vspace{1cm}





\vspace{0.5cm}

\end{frontmatter}

\setcounter{footnote}{0}



%________________________
% Header and Footers
%%%%%%%%%%%%%%%%%%%%%%%%%%%%%%%%%
\pagestyle{fancy}
\chead{}
\rhead{}
\lfoot{}
\rfoot{\footnotesize Page \thepage}
\lhead{}
%\rfoot{\footnotesize Page \thepage } % "e.g. Page 2"
\cfoot{}

%\setlength\headheight{30pt}
%%%%%%%%%%%%%%%%%%%%%%%%%%%%%%%%%
%________________________

\headsep 35pt % So that header does not go over title




\hypertarget{introduction}{%
\section*{Introduction}\label{introduction}}
\addcontentsline{toc}{section}{Introduction}

This paper aims to investigate the return predictive signal of
dividend-paying stocks. The value of dividends to shareholders has long
been a subject of debate among academics and practitioners, with
evidence both for their relevance and irrelevance. Miller \& Modigliani
(\protect\hyperlink{ref-miller}{1961}) proposed the dividend irrelevance
theorem, which argues that dividend payments are irrelevant because they
do not affect shareholder wealth, as it is the income a firm generates
that impacts wealth, not the way the firm distributes that income. In
contrast, Gordon (\protect\hyperlink{ref-gordon1962}{1962}) argued that
investors should prefer cash flows because they are certain, as opposed
to the riskier capital gains. According to this theory, the ``Bird in
Hand Theorem'' is used to justify the demand for dividend stocks,
especially for less risk-tolerant investors. Moreover, a closer
examination of the Miller \& Modigliani
(\protect\hyperlink{ref-miller}{1961}) theory reveals unrealistic
assumptions, rendering arguments immaterial once real-world constraints,
such as taxes, transaction costs, and behavioral biases, are considered.
Consequently, theories that consider information asymmetries, tax
considerations, and signaling provide convincing arguments for the
relevance of dividend payments in addressing real-world considerations.
Campbell (2017) posits that the value of a stock is a function of its
future cash flow. If markets are assumed to be efficient, then companies
that do not pay dividends should have a value of zero unless there is an
expectation of future receipts from the investment in said company.
Therefore, we can safely assume a direct relationship between share
price and new information that affects future cash flows. Event studies
on various types of dividend payments address this, and it is often
noted that the payment of dividends leads to a decrease in stock prices
and thus stock valuation. Suwanna (2012) demonstrates that dividend
announcements decrease share prices by the value of the dividend. This
could be because dividends are a capital budgeting decision, and their
payment reduces retained earnings, consequently affecting the share
price.

This study provides an extensive investigation into the return signaling
of dividend portfolios. Firstly, it reviews existing literature on
dividend payments, their rationale, and the theories both for and
against their relevance. Secondly, it outlines our methodology for
constructing dividend portfolios, which includes our utility optimizer
and associated constraints. Thirdly, the study discusses the results by
addressing the first question: When do dividend signals work? To answer
this, we initially examine the cumulative excess returns offered by
global dividend indexes to provide a comprehensive overview of the
performance of geographical dividend portfolios. Our analysis shows that
dividend portfolios, whether high yield (HY) or dividend growth (DG), do
not consistently provide a clear signal for returns. However, when we
stratify our sample according to interest rate cycles (Hiking, Cutting,
and Neutral) and market regimes (High and Low Volatility), unique to
each geographical region, we begin to find that within specific market
regimes, dividend signals exhibit defensive characteristics and offer
higher excess returns during periods of High Volatility. This, however,
does not guarantee positive returns for all indexes, as our proxies for
the United States (US) HY, UK HY, and World indexes returned negative
results over their respective periods.

Regarding performance consistency, when examining the rolling
information ratio, we observe an inherent lack of consistency in
dividend portfolio performance, with most strategies underperforming the
benchmark over the past 15 years. Nevertheless, coupled with drawdowns
experienced in different jurisdictions, we acknowledge that advanced
market dividend portfolios experience the least variation in drawdowns
compared to emerging markets. The results indicate that both growth and
high-yield dividend portfolios tend to underperform relative to their
benchmarks, raising questions about their ability to extract a value
premium.

Lastly, the paper delves into how dividend portfolios work with the goal
of constructing a portfolio that can best harness the existing premium.
We utilize share data from the Johannesburg Stock Exchange (JSE), with
our benchmark being the Capped SWIX Top 50. We construct four portfolios
using 3-month and 9-month measures of HY, 1 and 3-year DG measures,
together with price momentum and dividend coverage ratios. Similar to
the analysis of international dividend portfolios, our portfolios
exhibit defensive properties in high-volatility periods but fail to
consistently capture a premium over investing in a market index.

\hypertarget{literature-review}{%
\section*{Literature Review}\label{literature-review}}
\addcontentsline{toc}{section}{Literature Review}

The observed decline in shareholder value has sparked a long-running
debate on the relevance and irrelevance of dividends. In 1961, Miller
(1985) proposed the Dividend Irrelevance Theory (MM theory), arguing
that dividends are irrelevant to shareholders. According to this theory,
shareholders are indifferent to dividend payments, implying that there
is no optimal dividend policy, and all dividend policies are equally
good. Miller contended that dividend payments could easily be reinvested
in shares and would make no difference to shareholder wealth. However,
the MM theorem fails to consider real-world market imperfections that
may render dividends relevant.

Opposing the MM theory, Gordon (1962) argued that investors prefer
receiving less risky cash flows in the form of dividends rather than
potentially uncertain capital gains in the future. This preference for
dividends has implications for the cost of equity, as companies that
issue more dividends are perceived as less risky, leading to higher
share prices. Supporters of the MM theory, however, argue that the risk
of future cash flows is influenced by dividend payments, which can
negatively affect share prices, especially after the ex-dividend date.
The dividend puzzle takes into account real-world constraints and
suggests that while dividends may reduce equity value, they serve as a
reward to investors who bear the risk associated with their investments,
providing an additional source of return on investment (Black, 1996).

Various literature has presented compelling arguments for corporations
to pay dividends. These include tax considerations, dividend signaling,
and agency costs associated with issuing dividends. Tax considerations
argue in favor of dividend relevance since dividends are often taxed at
higher income tax rates than capital gains in various jurisdictions.
This can lead investors with higher tax rates to favor stocks with lower
dividend payouts, potentially increasing stock prices, known as the
clientele effect (Van, 2013; Baker, 1999). Proponents of the MM theory
counter this argument, suggesting that the clientele effect causes a
substitution effect, where investors adjust their capital allocation
based on tax treatment, leading to a net zero effect on prices.

Another factor that gives relevance to dividend payments is information
asymmetry between shareholders and managers. Managers have greater
knowledge of a business's operations and value at any given point than
shareholders. As a result, investors rely on dividend announcements to
assess a company's valuation. Dividend signaling conveys information
about a company's quality (Al, 2018; Baker, 1999). However, concerns
exist about management potentially ``gaming'' the dividend signal,
making it an imperfect indicator of share prices.

Principal agency issues provide another reason for the issuance of
dividends. The free cash flow hypothesis suggests that dividend payments
force management to raise capital from external sources, increasing
borrowing costs and scrutiny from capital markets. This reduces
management's ability to make suboptimal investments, aligning management
and shareholder objectives (Baker, 1999).

\hypertarget{constructed-dividend-portfolios}{%
\section*{Constructed Dividend
Portfolios}\label{constructed-dividend-portfolios}}
\addcontentsline{toc}{section}{Constructed Dividend Portfolios}

From the dividend relevance theories, it can be concluded that dividend
payments emanate from proxy arguments, rather than being considered an
attractive feature in themselves. High dividend-paying companies can act
as proxies for the quality of management structures over time,
indicating their ability to consistently afford dividend payments. They
can also signal prudent cash-flow management capabilities and often
resemble value stocks (Basu, 1977). This is supported by research by
Basu (1977), who used price-to-earnings ratios to predict stock
performance and found that low price-to-earnings stocks tend to
outperform their higher price-to-earnings counterparts.

When we consider the formula for dividend yield:
\(DY = \frac{EPS}{Price} \times Payout Ratio\), it becomes evident that
dividend yield, when the payout ratio is held constant, is a function of
earnings yield. Therefore, identifying low price-to-earnings stocks
through dividends can serve as a proxy for value. However, it's
important to note that value stocks carry higher levels of risk as they
are more prone to financial distress and uncertainty in future earnings
(Chen, 1998).

To address some of the negative aspects of high dividend yield (HY),
dividend growth (DG) is used as a signal. DG stocks, unlike HY, are not
affected by stock price but maintain properties that allow for
inferences about management quality. As management is aware of the
signaling effect of dividends, this may induce a ``value trap,'' forcing
management to continually increase dividends to maintain a certain
valuation. However, such companies are more vulnerable to financial
distress (Baker, 2009).

The ``Dogs of the Dow'' strategy, as explored by O'Higgins (1991),
involves ranking 30 companies by HY and selecting the 10
highest-yielding shares for a portfolio. This strategy has shown
superior returns compared to the Dow Jones Industrial Average (DJIA) and
lower risk, achieving a higher Sharpe Ratio. Other studies across
jurisdictions have yielded similar conclusions (Lemmon, 2015;
Brzeszczynski, 2007; Visscher, 2003; Filbeck, 1997; Wang, 2011).

In South Africa, Fakir (2013) employed a parametric approach to
investigate dividends as an investment strategy, further contributing to
the literature on the topic.

\hypertarget{methodology}{%
\section*{Methodology}\label{methodology}}
\addcontentsline{toc}{section}{Methodology}

To evaluate the return predictive signal of dividends, we employ an
applied approach that constitutes constructing subset portfolios and
compare in sample performances. Our approach aims to give valuable
insights based on risk and return for systematically constructed
dividend portfolios.

\hypertarget{portfolio-optimzation}{%
\subsection*{Portfolio optimzation}\label{portfolio-optimzation}}
\addcontentsline{toc}{subsection}{Portfolio optimzation}

The Modern Portfolio Theory defines risk of a portfolio of (\(n\))
assets as the variance (\(\sigma^2\)) of its returns (\(r_{t}\)). We add
a refinement to this, and our definition returns is achieved by
decomposing it into common factor (\(Xf\)) and specific return (\(u\))
as (\(r = Xf+ u\)). From these returns we create a factor covariance
matrix, defined as (\(X F X^T+D\)) in which we derive our multiple
factor universe consists of (\(k\)) common factors.

We periodically calculate each asset exposure to the common factors
calculated in the factor covariance matrix. This then assists us in
computing forecasts of the level of each asset specific risk. The short
term risk forecasts will then be used to gauge contribution of each
asset to a portfolio over risk which contributes to the portfolio
construction process. For our optimization, risk takes on two forms
being total risk (only portfolio holdings are considered and benchmark
holdings are irrelevant for the optimization process) and active risk
(difference between portfolio holdings and benchmark holdings are given
consideration in the optimization problem).

where,

\hypertarget{constraints}{%
\subsection*{Constraints}\label{constraints}}
\addcontentsline{toc}{subsection}{Constraints}

To evaluate the return predictive signals of dividends, we employ an
applied approach that involves constructing subset portfolios and
comparing their in-sample performance. Our approach is designed to
provide valuable insights into risk and return for systematically
constructed dividend portfolios.

Portfolio Optimization

In the context of Modern Portfolio Theory, the risk of a portfolio
consisting of \(n\) assets is typically defined as the variance
\((\sigma^2)\) of its returns (rt). We refine this definition by
decomposing returns into common factors (\(Xf\) and specific return
(\(u\)), represented as \(r = Xf + u\). From these returns, we create a
factor covariance matrix, denoted as

\(XFX^T + D\),\(\begin{array} {ll}X & =n \times k \text { matrix of asset exposures to the factors, } \\ F \quad &= k \times k \text { positive semi-definite factor covariance matrix, and } \\ D \quad &=n \times n \text { positive semi-definite covariance matrix representing a } \\ & \text { forecast of asset specific risk. }\end{array}\)
risk.

We periodically calculate each asset's exposure to the common factors,
which aids in computing forecasts of each asset's specific risk level.
These short-term risk forecasts are used to assess the contribution of
each asset to a portfolio in terms of risk, a crucial component of the
portfolio construction process. For optimization purposes, risk is
considered in two forms: total risk, where only portfolio holdings are
considered (benchmark holdings are irrelevant for the optimization
process), and active risk, which takes into account the difference
between portfolio holdings and benchmark holdings.

Defined as: \emph{Total Risk}:
\(\quad h^T\left(\lambda_F X F X^T+\lambda_D D\right) h\)

\emph{Active Risk}:
\(\left(h-h_B\right)^T\left(\lambda_F X F X^T+\lambda_D D\right)\left(h-h_B\right)\)
Where: \[
\begin{aligned}
\lambda_F & =\text { common factor risk aversion parameter, } \\
\lambda_D & =\text { specific risk aversion parameter, } \\
h & =n \times 1 \text { vector of managed portfolio's holdings, and } \\
h_B & =n \times 1 \text { vector of normal (benchmark) portfolio's holdings }
\end{aligned}
\]

The optimization process involves a set of constraints designed factor
in practical considerations faced by portfolio managers. The choice of
constraints varies depending on the risk objectives and goals of
portfolio managers, taking into account investor risk preferences.

We employ the following constraints:

\begin{itemize}
\tightlist
\item
  Common factor and specific risk aversion parameters are set at 0.0075
  and 1, respectively.
\item
  Our investment universe consists of the Top 50 stocks listed on the
  JSE, and the selection criteria depend on market capitalization and
  liquidity.
\item
  The Capped SWIX is used as the benchmark.
\item
  Portfolios are rebalanced quarterly.
\item
  Active risk constraints relative to the benchmark are set at 5\%.
\item
  Sector exposure is constrained within a range of +/-10\%, with no
  property stocks in the portfolio.
\item
  Individual stock exposure is limited to 15\%.
\item
  Quarterly turnover is limited to 10\%.
\end{itemize}

\hypertarget{dividend-signals}{%
\section{Dividend Signals}\label{dividend-signals}}

In constructing our portfolios, we rank the stocks in the Capped SWIX
based on either dividend yield or dividend growth per share. Our
in-house measures for dividend yield (DY 3m fwd and DY 9m fwd) are
constructed using industry analyst estimates of future earnings. These
measures are forward-looking, providing an estimate of potential future
dividend yields. Similarly, our dividend growth measures for 1 and 3
years are constructed using forward-looking estimates. To further refine
our portfolio construction, we employ price momentum and dividend
coverage filters. These filters reward companies with strong price
momentum and sustainable dividend practices, which have been shown to
enhance stock selection for portfolios.

\hypertarget{divi1}{%
\section{DIVI1}\label{divi1}}

Rank score (0 to 100) calculated using a combination of 2/3 DY (3m fwd),
1/3 DY (9m fwd), Dividend Coverage Ratio, and Price Momentum. The signal
uses the following conditions:

If the dividend cover score is in the bottom quintile, it is weighted at
15\% (15\% dividend cover, 66.667\% * 0.85 DY 3m, 33.333\% * 0.85). If
the price momentum score is in the bottom quintile, it is weighted at
35\% (35\% price momentum score = 66.667\% * 0.65 DY 3m, 33.333\% *
0.65). If both dividend cover and momentum are in the bottom quintile, a
combination weight is applied (15\% dividend cover score = 35\% price
momentum score, 66.667\% * 0.5 DY 3-month, 33.333\% * 0.5). This
portfolio utilizes price momentum and dividend cover ratio as filters to
enhance the dividend yield signal. It rewards sustainability in dividend
payments and avoids companies that may not afford to pay dividends,
reducing the risk of capital gain losses.

\hypertarget{divi2}{%
\section{DIVI2}\label{divi2}}

Rank score (0 to 100) is calculated based solely on dividend yield,
which is a blend of forward-looking metrics (2/3 DY 3m fwd and 1/3 DY 9m
fwd).

This portfolio represents a straightforward dividend yield strategy that
ranks stocks based on their dividend payment potential.

\hypertarget{divi3}{%
\section{DIVI3}\label{divi3}}

Rank score (0 to 100) is based on the price-to-earnings (P/E) ratio,
serving as an alternative proxy for value.

\hypertarget{divi4}{%
\section{DIVI4}\label{divi4}}

Rank score (0 to 100) is calculated using a combination of DPS Growth
1-Year (40\%), DPS Growth 3-Year (30\%), Forward 3 (20\%), and Forward 9
(10\%).

This portfolio focuses on dividend growth, utilizing trailing dividend
growth rates and forward-looking measures to make informed stock
selections.

Incorporating these signals in our portfolio construction process
enhances our ability to capture dividend-related return signals and
create diversified portfolios that align with investor objectives and
risk preferences. It's important to note that these strategies are
grounded in empirical research and industry best practices, contributing
to the academic rigor of our approach.

\hypertarget{data}{%
\subsection*{Data}\label{data}}
\addcontentsline{toc}{subsection}{Data}

In this study, we focus on return and risk measures to evaluate the
effectiveness of dividend signals as an investment strategy. Our data
spans from June 30, 2003, to June 30, 2023. The choice of start and end
dates is primarily dictated by data availability on the selected
dividend indices at the time of writing.To conduct our analysis, we
collected historical daily price data for the dividend portfolios and
the constituents of the Capped SWIX Top 50 listed on the Johannesburg
Stock Exchange (JSE) from Bloomberg. Please refer to \ref{reftab} for a
comprehensive guide to the indices used and codenames referenced in the
subsequent results and analysis.

Additionally, we gathered volatility and interest rate proxies for the
various geographical regions under investigation over the same sample
period as our dividend portfolios. Specifically, we used the Chicago
Board Options Exchange (CBOE) VIX Index for the United States and
emerging markets (EM), V2X for Europe, IVUK for the United Kingdom, and
JALSH VR for South Africa as volatility proxies. For interest rate data,
we considered the policy rates set by central banks for the respective
geographies within our study. These included the Federal Fund rate for
the United States and emerging markets, the Minimum Deposit Financing
Rate for the European Union, the Bank of England Bank Rate, and the
South African Reserve Bank Repo rate.

To classify periods of high and low volatility (Hi-vol and Lo-vol), we
devised a rule by computing the top and bottom quantiles in standard
deviation for our respective proxies. Subsequently, we identified the
dates corresponding to these periods and calculated annualized returns
by geometrically chaining the monthly returns.

However, it's important to note that when the VIX, V2X, or JALSH RV
breached the top or bottom quintile for fewer than 50 trading days, we
excluded those periods from our analysis to prevent the annualization of
small sample sizes.

Furthermore, we differentiated between Hiking, Cutting, and Neutral
interest rate cycles. We defined these periods based on changes in
interest rates, specifically as five consecutive quarters of upward
changes for Hiking and downward changes for Cutting. Alternatively, if
central banks maintained interest rates at a constant level, we
categorized the period as Neutral

\hypertarget{results-and-discussion}{%
\section*{Results and Discussion}\label{results-and-discussion}}
\addcontentsline{toc}{section}{Results and Discussion}

\hypertarget{when-do-dividend-strategies-work}{%
\subsection*{When Do Dividend Strategies
Work}\label{when-do-dividend-strategies-work}}
\addcontentsline{toc}{subsection}{When Do Dividend Strategies Work}

We begin our analysis by evaluating several performance metrics of
globally traded dividend portfolios. It is important to note that Table
\ref{tab1} does not facilitate direct comparisons across indexes and
regions due to differences in the inception dates of the respective
instruments. Table \ref{tab1} presents four key performance measures,
encompassing relative return (annualized excess return), total return
(cumulative return), maximum drawdowns, and standard deviation. The
latter two measures provide an initial insight into the risk
characteristics of our selected investment universe, though a more
detailed risk analysis will follow in our subsequent analysis.

An immediate observation from the table is that total returns, as
assessed by cumulative returns, are positive for approximately 42\% of
the sample. Specifically, dividend yield (DY) strategies exhibit the
highest total returns. When examining annualized excess returns, we
notice that the majority of strategies outperform their respective
benchmarks. However, it's worth noting that the relative performance
against their benchmarks is marginal, with relative performance between
- 0.01\% to 0.01\%.

Turning our attention to risk, we find that there is a narrow range in
annualized standard deviation across most strategies. Nevertheless,
drawdowns exhibit variation across strategies and regions. Drawdowns do
not provide a definitive picture of the return characteristics of the
constituents, but in broad terms, DY strategies tend to experience lower
drawdowns compared to dividend growth (DG) strategies.

\begin{table}[H]
\centering
\begin{tabular}{rlrrrr}
  \hline
 & Index & Ann Return & Std dev & Max Drawdowns & Cumulative Return \\ 
  \hline
1 & EM\_HY & -0.00 & 0.04 & 0.13 & 0.58 \\ 
  2 & SA\_HY & 0.01 & 0.04 & 0.17 & 0.22 \\ 
  3 & SA\_DG & 0.01 & 0.04 & 0.19 & 0.21 \\ 
  4 & EU\_DG & 0.01 & 0.03 & 0.14 & 0.11 \\ 
  5 & EU\_HY & -0.00 & 0.04 & 0.24 & 0.03 \\ 
  6 & W\_HY & -0.00 & 0.03 & 0.21 & -0.06 \\ 
  7 & JP\_HY & 0.00 & 0.05 & 0.28 & -0.10 \\ 
  8 & UK\_HY\_B & -0.00 & 0.04 & 0.24 & -0.13 \\ 
  9 & JP\_DG & -0.00 & 0.04 & 0.31 & -0.13 \\ 
  10 & US\_HY & -0.00 & 0.03 & 0.27 & -0.18 \\ 
  11 & US\_DG & -0.00 & 0.03 & 0.29 & -0.20 \\ 
  12 & UK\_HY & 0.01 & 0.03 & 0.29 & -0.24 \\ 
   \hline
\end{tabular}
\caption{Global Index  Portfolio Performance \label{tab1}} 
\end{table}

A more discernible picture emerges as we stratify the performance of
global dividend portfolios according to market volatility and interest
rate cycles. Our categorization of market volatility into ``High Vol''
and ``Low Vol'' cycles is achieved through a two-step process. First, we
calculate the rolling 12-month standard deviations of volatility
proxies. Second, we identify the 5th and 95th percentiles of
observations and extract the corresponding dates. Following this
stratification, we calculate excess returns within these periods,
geometrically chaining them and subsequently annualizing the excess
returns with the appropriate periodicity.

Tables \ref{tab2} and \ref{tab3} present the performance of dividend
portfolios during periods of market distress (Hi Vol) and market calm
(Lo Vol). This stratification allows for comparisons across regions and
strategies, although it's crucial to be aware that the duration of
market volatility can impact strategy performance.

From our stratification, we find that 37.5\% of observations yield
positive excess returns. Among those, HY indexes, which constitute
approximately 67\% of the portfolios with positive returns, tend to
perform better in periods of market distress compared to market calm. In
fact, most indexes in the high volatility periods exhibit returns close
to 0\%. For instance, in Hi Vol conditions, HY consistently outperforms
DG strategies, with SA\_HY (2.74\%) outperforming SA\_DG (-4.07\%),
EU\_HY (0.08\%) outperforming EU\_DG (-4.84\%), and US\_HY (-0.35\%)
outperforming US\_DG (-0.68\%).

In contrast, during Lo Vol periods, we observe greater dispersion in
annualized excess returns, making it less clear whether prolonged
periods of market calm lead to lower returns. The performance is less
consistent in low volatility periods. Overall, our market cycle
stratification reveals that DY and DG strategies tend to offer defensive
characteristics in periods of higher volatility. However, it's important
to note that only a few of our portfolios generate positive excess
returns in these periods. Therefore, idiosyncrasies within the market
play a significant role in determining the performance of these signals.

\begingroup\fontsize{12pt}{13pt}\selectfont
\begin{longtable}{llrr}
  \toprule
Name & Market Period & Months & Annualized Return (\%) \\ 
  \hline 
\endhead 
\hline 
{\footnotesize Continued on next page} 
\endfoot 
\endlastfoot 
 \midrule
UK\_HY\_B & High Vol &  36 & 8.70 \\ 
  EU\_HY & High Vol &  36 & 5.40 \\ 
  EU\_DG & Low Vol Period &  55 & 3.53 \\ 
  EM\_HY & Low Vol Period &  69 & 3.33 \\ 
  SA\_DG & Low Vol Period &  44 & 3.24 \\ 
  SA\_HY & High Vol &  39 & 2.74 \\ 
  JP\_DG & High Vol &  58 & 2.52 \\ 
  JP\_HY & High Vol &  58 & 0.37 \\ 
  EM\_HY & High Vol &  58 & 0.08 \\ 
   \bottomrule
\caption{Over Performance Volatility Stratification\label{tab2}} 
\end{longtable}
\endgroup
\begingroup\fontsize{12pt}{13pt}\selectfont
\begin{longtable}{llrr}
  \toprule
Name & Market Period & Months & Annualized Return (\%) \\ 
  \hline 
\endhead 
\hline 
{\footnotesize Continued on next page} 
\endfoot 
\endlastfoot 
 \midrule
EU\_HY & Low Vol Period &  55 & -0.11 \\ 
  JP\_HY & Low Vol Period &  69 & -0.28 \\ 
  US\_HY & High Vol &  58 & -0.35 \\ 
  US\_DG & High Vol &  58 & -0.68 \\ 
  W\_HY & High Vol &  58 & -0.69 \\ 
  US\_DG & Low Vol Period &  69 & -0.76 \\ 
  W\_HY & Low Vol Period &  69 & -1.22 \\ 
  SA\_HY & Low Vol Period &  44 & -1.99 \\ 
  US\_HY & Low Vol Period &  69 & -2.42 \\ 
  UK\_HY\_B & Low Vol Period &  55 & -3.63 \\ 
  SA\_DG & High Vol &  39 & -4.07 \\ 
  EU\_DG & High Vol &  36 & -4.84 \\ 
  JP\_DG & Low Vol Period &  69 & -6.46 \\ 
  UK\_HY & Low Vol Period &  55 & -7.20 \\ 
  UK\_HY & High Vol &  36 & -24.01 \\ 
   \bottomrule
\caption{Under Performance Volatility Stratification\label{tab3}} 
\end{longtable}
\endgroup

When we stratify our analysis according to interest rate cycles in
Tables \ref{tab4} and \ref{tab5,} focusing on Hiking, Cutting, and
Neutral cycles, we encounter an interesting anomaly in Japan. Unlike
other economies, Japan does not have distinct hiking or cutting cycles,
as its central bank has largely maintained constant interest rates.
Therefore, we exclusively assess Japan's performance within the confines
of a neutral interest rate cycle. Upon stratification, we geometrically
chain quarterly excess returns and then annualize them to enable
comparisons across indices. Our analysis reveals several insights. The
average and median annualized cumulative excess returns of the US and EU
dividend indexes during cutting cycles suggest that DG strategies tend
to outperform, particularly when interest rates are declining.
Underperformance during hiking cycles is less pronounced in the US,
especially when considering annualized excess returns between Hiking and
Cutting cycles. Finally, EU, both DY and DG strategies exhibit
outperformance during hiking cycles, with dividend growth strategies
outperforming to a greater extent. To provide further context for this
analysis, we include results from a principal component analysis on
returns of each dividend portfolio and its benchmark in Appendix 2.
After identifying the principal components, we regress the first three
principal components against returns to formalize our assessment of the
return drivers for our portfolios. Notably, dividend portfolios, whether
high yield (HY) or dividend growth (DG), exhibit similar loadings
relative to their benchmarks, although the loadings are slightly larger.
Given practical considerations in constructing indexes and the results
from our models, we posit that dividend portfolios may effectively proxy
the market index, providing investors with exposure similar to the
hypothetical index.

\begingroup\fontsize{12pt}{13pt}\selectfont
\begin{longtable}{llrr}
  \toprule
Name & Market Period & Quarters & Annualized Return (\%) \\ 
  \hline 
\endhead 
\hline 
{\footnotesize Continued on next page} 
\endfoot 
\endlastfoot 
 \midrule
US\_DG & Cut &  15 & 13.40 \\ 
  EU\_DG & Cut &  14 & 6.15 \\ 
  EU\_DG & Neutral &  29 & 3.37 \\ 
  US\_DG & Hiking &  36 & 3.19 \\ 
  EM\_HY & Hiking &  36 & 2.81 \\ 
  JP\_HY & Neutral &  49 & 1.88 \\ 
  SA\_HY & Hiking &  39 & 1.74 \\ 
  JP\_DG & Neutral &  49 & 1.29 \\ 
  EU\_DG & Hiking &  27 & 1.12 \\ 
  SA\_DG & Hiking &  39 & 1.10 \\ 
  US\_HY & Cut &  15 & 0.05 \\ 
   \bottomrule
\caption{Over Performance in Interest Rate Regimes\label{tab4}} 
\end{longtable}
\endgroup
\begingroup\fontsize{12pt}{13pt}\selectfont
\begin{longtable}{llrr}
  \toprule
Name & Market Period & Quarters & Annualized Return (\%) \\ 
  \hline 
\endhead 
\hline 
{\footnotesize Continued on next page} 
\endfoot 
\endlastfoot 
 \midrule
EU\_HY & Cut &  14 & -0.87 \\ 
  EU\_HY & Neutral &  29 & -1.07 \\ 
  EU\_HY & Hiking &  27 & -2.01 \\ 
  US\_HY & Hiking &  36 & -2.51 \\ 
  EM\_HY & Cut &  15 & -2.73 \\ 
  UK\_HY\_B & Neutral &  22 & -2.91 \\ 
  US\_DG & Neutral &  20 & -3.32 \\ 
  SA\_DG & Cut &  27 & -6.77 \\ 
  SA\_HY & Cut &  27 & -7.48 \\ 
  EM\_HY & Neutral &  20 & -7.88 \\ 
  US\_HY & Neutral &  20 & -8.83 \\ 
  UK\_HY\_B & Cut &  19 & -13.11 \\ 
  UK\_HY & Neutral &  22 & -14.72 \\ 
  UK\_HY & Cut &  19 & -25.36 \\ 
  UK\_HY\_B & Hiking &  30 & -27.06 \\ 
  UK\_HY & Hiking &  30 & -34.58 \\ 
   \bottomrule
\caption{Under Performance in Interest Rate Regimes\label{tab5}} 
\end{longtable}
\endgroup

In our final analysis, we consider a dynamic measure to evaluate the
risk-adjusted performance of dividend portfolios over time. Figure
\ref{fig1} provides a visual representation of the consistency in the
performance of dividend portfolios. We achieve this by employing a
rolling 60-month information ratio, which helps avoid the influence of
short-term events that may skew performance results. This ratio is
computed by determining the rolling excess return of the index relative
to its benchmark and then dividing this by the volatility of those
excess returns. The red line in the figure represents out performance
relative to the benchmark while considering risk, serving as a yardstick
to assess satisfactory performance. From Figure \ref{fig1}, UK\_HY
exhibits undesirable consistency in returns over the sample period. The
EM and Japan dividend portfolios exhibit polarizing performances
throughout the sample period. From 2005 to 2015, returns for these
portfolios were consistently positive, but over the last eight years,
information ratios have been negative.

In contrast, South African portfolios, particularly dividend growth
portfolios, have shown positive information ratios since 2010 to 2020.
SA\_HY only turned positive since 2017. The US and EU indexes exhibit
similar intra-region performance, with information ratios that remain
close to 0 and show minimal deviation over time. Overall, our analysis
based on information ratios yields similar results to our whole sample
analysis on total return in Table \ref{tab1}. It suggests that dividend
strategies, with the exception of SA\_HY (which has seen a steady
decline in its information ratio), generally fail to outperform their
benchmark. This result contradicts our findings from stratification,
where we observed outperformance of dividend strategies in low-interest
rate environments or periods of high volatility. The varying performance
of these strategies highlights their sensitivity to the economic and
market conditions in which they operate.

\begin{figure}[H]

\includegraphics{Much_Ado_About_Dividends_files/figure-latex/unnamed-chunk-1-1} \hfill{}

\caption{Rolling 3 Year Returns \label{fig1}}\label{fig:unnamed-chunk-1}
\end{figure}

\hypertarget{application-to-south-africa}{%
\section*{Application to South
Africa}\label{application-to-south-africa}}
\addcontentsline{toc}{section}{Application to South Africa}

\hypertarget{backtest-results-from-dividend-portfolio-signals}{%
\subsection*{Backtest Results from Dividend Portfolio
Signals}\label{backtest-results-from-dividend-portfolio-signals}}
\addcontentsline{toc}{subsection}{Backtest Results from Dividend
Portfolio Signals}

Figure \ref{fig3} provides a visual representation of the cumulative
returns of our dividend portfolios, along with a display of the total
capital invested during the sample period. These portfolios are
categorized into Dividend Yield (DY), Dividend Growth (DG), Price
Momentum, and Sustainability, and they are compared to the performance
benchmark represented by the SWIX Top 40 index. In line with our earlier
analysis of the SA\_HY and SA\_DG portfolios, a noticeable pattern
emerges, indicating that the returns over the sample period fall short
of the benchmark set by the market index. Furthermore, our vanilla
portfolio, Dividend High Yield (HY), displays the lowest cumulative
returns. Similarly, the Price Momentum and Sustainability portfolios
exhibit diminished performance when compared to both the Value and DG
portfolios. This observation highlights a consistent trend of
underperformance in our portfolios when evaluated against the broader
market index.

This outcome prompts further examination and investigation to uncover
the underlying factors and potential implications within the context of
dividend-oriented investment strategies. It is essential to understand
the reasons behind this consistent underperformance and consider
adjustments or enhancements to the strategies to potentially improve
their effectiveness.

\begin{figure}[H]

\includegraphics{Much_Ado_About_Dividends_files/figure-latex/Figure3-1} \hfill{}

\caption{Rolling 3 Year Returns \label{fig3}}\label{fig:Figure3}
\end{figure}

Tables \ref{tab6} and \ref{tab7} offer a detailed breakdown of total
investments during periods characterized by market cycles and interest
rate regimes, specifically high or low volatility, and hiking or cutting
cycles. These categorizations align with the realized market volatility
and interest rate regime stratification dates we used in our analysis of
Tables \ref{tab2} and \ref{tab3} for the SA\_HY and SA\_DG indexes.

Consistent with our findings from internationally traded portfolios, we
once again observe the favorable characteristics of dividend portfolios
when stratified according to interest rate regimes and volatility
levels. Notably, these dividend portfolios exhibit significant defensive
attributes, with the price momentum-adjusted and sustainability
portfolios generating the highest returns during hiking periods and
times of high volatility. Furthermore, it's essential to emphasize that
our portfolios consistently outperform the market index during these
periods.

These results highlight the resilience and attractiveness of
dividend-oriented investment strategies in challenging market
conditions, underscoring their potential as a valuable component of an
investor's portfolio. Further analysis and research could shed more
light on the underlying factors contributing to this outperformance and
help investors make informed decisions in constructing their portfolios.

\begin{table}[H]
\centering
\begin{tabular}{rlrrrl}
  \hline
 & Portfolio & ROI & Ann ROI \% & SD \% & MarketCycle \\ 
  \hline
1 & BM & 1.06 & -0.87 & 0.02 & High Volatility  \\ 
  2 & Divi1 & 1.02 & -0.91 & 0.03 & High Volatility  \\ 
  3 & Divi2 & 1.04 & -0.89 & 0.03 & High Volatility  \\ 
  4 & Divi3 & 1.00 & -0.92 & 0.01 & High Volatility  \\ 
  5 & Divi4 & 1.06 & -0.88 & 0.04 & High Volatility  \\ 
  6 & BM & 1.83 & 2.09 & 0.08 & Low Volatility  \\ 
  7 & Divi1 & 1.13 & -0.82 & 0.06 & Low Volatility  \\ 
  8 & Divi2 & 1.26 & -0.66 & 0.05 & Low Volatility  \\ 
  9 & Divi3 & 1.45 & -0.23 & 0.08 & Low Volatility  \\ 
  10 & Divi4 & 1.66 & 0.73 & 0.09 & Low Volatility  \\ 
   \hline
\end{tabular}
\caption{Market Cycle Perforomance \label{tab6} } 
\end{table}

During cutting cycles or during periods characterized by low market
volatility, an examination of the correlation between market cycles and
the performance of South African portfolios reveals a consistent and
noteworthy relationship. Specifically, in such scenarios, akin to the
broader spectrum of return on investment, it becomes evident that the
market index tends to yield relatively superior returns. It is worth
noting that among the various portfolios under scrutiny, our DIVI2
portfolio has exhibited a relative underperformance when compared to our
other dividend portfolios. Consequently, this analysis leads to the
inference that dividend-oriented portfolios can serve as an effective
instrument for investors seeking to augment their returns during phases
of heightened market volatility.

\begin{table}[H]
\centering
\begin{tabular}{rlrrrl}
  \hline
 & Portfolio & ROI & Ann ROI \% & SD \% & MarketCycle \\ 
  \hline
1 & BM & 0.16 & -1.00 & 0.01 & Hiking \\ 
  2 & Divi1 & 0.67 & -0.99 & 0.07 & Hiking \\ 
  3 & Divi2 & 0.41 & -1.00 & 0.06 & Hiking \\ 
  4 & Divi3 & 0.32 & -1.00 & 0.02 & Hiking \\ 
  5 & Divi4 & 0.20 & -1.00 & 0.03 & Hiking \\ 
  6 & BM & 3.41 & 6.15 & 0.26 & Cutting \\ 
  7 & Divi1 & 1.22 & -0.88 & 0.12 & Cutting \\ 
  8 & Divi2 & 1.48 & -0.74 & 0.15 & Cutting \\ 
  9 & Divi3 & 1.95 & -0.24 & 0.14 & Cutting \\ 
  10 & Divi4 & 2.51 & 1.07 & 0.23 & Cutting \\ 
   \hline
\end{tabular}
\caption{Interest Rate Regime Performance \label{tab7}} 
\end{table}

To ascertain the periods during which our dividend portfolios offer
value, we implement a relative performance evaluation method by
computing the product of excess returns and excess weights, yielding a
relative performance metric. This involves calculating the disparity in
monthly excess returns from the most recent rebalancing date within our
back-testing, conducted annually in October, covering the period from
October 30, 2007, to October 30, 2022\footnote{It is crucial to note
  that within our analysis of back-test results, there were instances,
  specifically on October 30, 2008, where no weight data was available
  for any of the portfolios. Consequently, these specific data points
  have been excluded from our analysis, as their isolated occurrence did
  not exert a significant influence on the overarching long-term trends,
  thus upholding the overall integrity of our analysis. Also, it is
  important to emphasize that our analysis does not consider
  sector-specific returns and weights for the portfolios and benchmarks.
  Nevertheless, we enhance our measurement by amalgamating the excess
  returns and excess weights to derive a return attribution analysis for
  the comprehensive strategy. This refined approach enables us to
  discern subtle performance nuances attributed to strategic decisions
  concerning the overweighting or underweighting of assets during
  opportune moments. For an in-depth exploration of the rationale
  underlying our use of this return attribution methodology and its
  merits, interested readers are encouraged to refer to the work of
  Brinson \& Fachler
  (\protect\hyperlink{ref-brinson1985measuring}{1985})}

Figure \ref{fig4}, visually represents our metric for assessing relative
return performance, encompassing the influence of excess weights across
our entire investment horizon.

\begin{figure}[H]

\includegraphics{Much_Ado_About_Dividends_files/figure-latex/unnamed-chunk-5-1} \hfill{}

\caption{Rolling 3 Year Returns \label{fig4}}\label{fig:unnamed-chunk-5}
\end{figure}

In contrast to the findings presented in \ref{fig3}, where all dividend
signals yielded diminished Return on Investment (ROI) over the course of
our comprehensive back-test period, our dividend portfolio displayed
superior performance compared to the Capped SWIX Top 50 index, as
measured by Hit Rates (HR). Specifically, Divi1 exhibited a HR of 50\%,
Divi2 achieved a HR of 42.9\%, Divi3 attained a remarkable HR of 68.8\%,
and Divi4 registered a HR of 31.2\% during the observed period.

Significantly, our value-oriented portfolio (Divi 3) exhibited the
highest HR among the dividend portfolios, followed by Divi 1, Divi 2,
and Divi 4, in descending order. While these results align with our ROI
analysis for the first portfolio, it is important to note that the
sequence of performance differs for Divi 4, Divi 1, and Divi 2.

Based on this comprehensive analysis, we draw the conclusion that the
utilization of a proxy for value, as opposed to a strict dividend
signal, contributes more significantly to the overall performance of

\newpage

\hypertarget{conclusion}{%
\section*{Conclusion}\label{conclusion}}
\addcontentsline{toc}{section}{Conclusion}

Over the course of time, dividend portfolios, encompassing both DY and
DG strategies, have consistently displayed positive excess returns,
evident in the cumulative excess return figures. Although the UK\_HY
index notably exhibits the highest cumulative return, this trend is not
uniformly observed across various regional indexes. However, when we
segment these portfolios based on different periods of market
volatility, a distinct pattern emerges. During phases of heightened
market volatility, dividend strategies prove to be effective in
providing capital protection when integrated into a diversified asset
portfolio. Furthermore, during such high volatility periods, DY
strategies outperform the DG strategies. An intriguing observation is
that South African portfolios tend to perform well during these high
volatility periods, which is somewhat unconventional, given that such
times are typically associated with a flight to safety. Emerging Markets
(EM) and, by extension, South Africa, are generally perceived as riskier
investments.

Expanding our analysis to incorporate interest rate cycles reveals a
contrasting effect compared to the volatility-based stratification. It
becomes evident that all strategies tend to yield the highest returns
during low interest rate cycles. Information ratio analysis indicates
that dividend strategies generally do not exhibit consistent return
performance. At a broader level, these portfolios do not consistently
maintain a positive information ratio over an extended investment
horizon. However, disparities in performance emerge, with South African
dividend indexes consistently delivering positive ratios over the past
decade. Conversely, Emerging Markets and Japanese indexes have
experienced substantial declines in their information ratios, despite
seemingly consistent performance prior to 2015. Meanwhile, the United
States, European Union, and United Kingdom indexes have exhibited
unpredictable performance over the sampled period.

When we combine our information ratio findings with drawdown analysis,
we observe that advanced economies have experienced fewer drawdowns over
the sample period, with the exception of the UK. This could suggest a
relatively lower level of systematic risk in these economies.
Conversely, South African and Emerging Market drawdowns have been less
severe, possibly indicating reduced systematic risk in emerging markets.

In the context of South Africa, focusing on the top 50 companies by
market capitalization, we observe a performance pattern similar to our
international analysis. Firstly, dividend portfolios do not convincingly
deliver superior total returns, as calculated by cumulative returns.
However, their value becomes apparent during periods of high volatility
and rising interest rate cycles. Moreover, when we consider more
traditional proxies for value, such as the Price to Earnings (P/E)
ratio, the efficacy of dividend signals diminishes. In other words,
portfolios constructed based on P/E ratios perform exceptionally well in
our back-test and performance criteria, outperforming the market index
68.8\% of the time.

In conclusion, dividend portfolios may not serve as an ideal proxy for
value, and investors may benefit more from exploring investment products
that provide an income component to their total returns. However, it's
worth noting that investor preferences can vary, potentially driving
demand for specific dividend portfolio strategies. Based on the
evidence, investors constrained by investment policy statements may find
value in equity portfolios constructed using Dividend Growth (DG)
strategies as these use signals that do a better job of capturing
company cash flow management and managerial qualities. Despite their
lower hit rate, DG portfolios exhibit lower volatility in achieving
returns, making them a practical and profitable means of attaining
returns that closely align with the market index.

\newpage

\hypertarget{appendix}{%
\section*{Appendix}\label{appendix}}
\addcontentsline{toc}{section}{Appendix}

\begingroup\fontsize{8pt}{9pt}\selectfont
\begin{longtable}{llll}
  \toprule
TICKER & NAME & Codename & Inception Dates \\ 
  \hline 
\endhead 
\hline 
{\footnotesize Continued on next page} 
\endfoot 
\endlastfoot 
 \midrule
FUDP & FTSE UK Dividend+ Index & UK\_HY &  \\ 
  M2EFDY & MSCI EM HY Gross Total Return USD Index & EM\_HY &  \\ 
  M2GBDY & MSCI UK HY Gross Total Return USD Index & UK\_HY &  \\ 
  M2JPDY & MSCI Japan HY Gross Total Return USD & JP\_HY &  \\ 
  M2USADVD & MSCI USA HY Gross Total Return USD Index & US\_HY &  \\ 
  M2WDHDVD & MSCI World HY Gross Total Return Total Return USD Index & W\_HY &  \\ 
  SPDAEET & S\&P EU 350 Dividends Aristocrats Total Return Index & EU\_DG &  \\ 
  SPJXDAJT & S\&P/JPX Dividend Aristocrats Total Return Index & JP\_DG &  \\ 
  SPDAUDT & S\&P 500 Dividend Aristocrats Total Return Index & US\_DG &  \\ 
  SPSADAZT & S\&P South Africa Dividend Aristocrats Index ZAR Gross TR & SA\_DG &  \\ 
  TJDIVD & FTSE/JSE Dividend+ Index Total Return Index & SA\_HY &  \\ 
  M2EUGDY & MSCI Europe Ex UK HYGross Total Return USD Index & EU\_HY &  \\ 
  TUKXG & FTSE 100 Total Return Index GBP & UK &  \\ 
  GDUEEGF & MSCI Daily TR Gross EM USD & EM &  \\ 
  GDDUUK & MSCI UK Gross Total Return USD Index & UK\_B &  \\ 
  TPXDDVD & Topix Total Return Index JPY & JP &  \\ 
  GDDUUS & MSCI Daily TR Gross USA USD & US &  \\ 
  GDDUWI & MSCI Daily TR Gross World USD & W &  \\ 
  SPTR350E & S\&P Europe 350 Gross Total Return Index & EU\_2 &  \\ 
  SPXT & S\&P 500 Total Return Index & JP &  \\ 
  SPXT & S\&P 500 Total Return Index & US\_2 &  \\ 
  JALSH & FTSE/JSE Africa All Share Index & SA &  \\ 
  JALSH & FTSE/JSE Africa All Share Index & SA &  \\ 
  GDDUE15X & MSCI Daily TR Gross Europe Ex UK USD & EU &  \\ 
   \bottomrule
\caption{Index Description \label{tabdes}} 
\end{longtable}
\endgroup

\hypertarget{dividend-defintions}{%
\section*{Dividend Defintions}\label{dividend-defintions}}
\addcontentsline{toc}{section}{Dividend Defintions}

Bloomberg has two main categories for distributions: Cash Dividends and
Stock Dividends. Various kinds of distributions appear under these
definitions that do not necessarily only apply to ordinary issued shares
(the only security type that we consider in our study). In the next two
subsections we define the types of distributions that fall under these
categories and in some cases provide additional information. Our sample
only comprises of final, interim and regular cash dividends. These
dividends are categorized by Bloomberg as Normal Cash.

\hypertarget{cash-dividends}{%
\subsubsection*{Cash Dividends}\label{cash-dividends}}
\addcontentsline{toc}{subsubsection}{Cash Dividends}

\begin{itemize}
\tightlist
\item
  Final: dividend declared for the financial year-end
\item
  Interim (includes 2nd interim, 3rd interim and 4th interim): dividend
  paid after a reporting period (eg. quarterly or semi-annually) Special
  Cash: dividend declared for the financial year-end or interim period
  over and above the normal dividend
\item
  Regular Cash: a dividend distribution made in cash
\item
  Omitted: A company has elected to skip a scheduled payment
\item
  Discontinued: The discontinuance of dividend payments on an ongoing
  basis
\item
  Interest on Capital: interest paid on fixed income instruments
\item
  Income: mutual fund dividends, in most cases
\item
  Liquidation: a distribution of a companies assets to shareholders
  during (interim) or after delisting (final)
\item
  Return of Capital: a non-taxable cash payment to investors from the
  company that represents a return on invested capital as opposed to a
  dividend
\item
  Memorial: a special dividend. For example a company celebrating an
  anniversary might pay a memorial dividend
\item
  Proceeds from sale of shares: a distribution of cash to shareholders
  after selling shares. For example this may occur when the company
  sells the shares of a shareholder who was not eligible to receive
  shares in an offering and then distributes the proceeds to
  shareholders
\item
  Cancelled: the cancellation of a previously declared dividend
\item
  Return Premium: special cash dividend paid from a special reserve
\item
  Preferred Rights Redemption: a company pays a dividend in exchange for
  previously issued preferred rights
\end{itemize}

\hypertarget{stock-dividends}{%
\section*{Stock Dividends}\label{stock-dividends}}
\addcontentsline{toc}{section}{Stock Dividends}

Bonus: also known as a scrip or capitalization issue. Shareholders are
given additional stock in proportion to their holdings - Scrip: a free
issue or bonus of shares - Stock Dividend: portion of a company's
retained earnings that are distributed to shareholders in stock. The JSE
treats stock dividends as a capitalization issue \newpage

\hypertarget{appendix-2}{%
\section*{Appendix 2}\label{appendix-2}}
\addcontentsline{toc}{section}{Appendix 2}

\hypertarget{principal-component-analysis-results}{%
\section*{Principal Component Analysis
Results}\label{principal-component-analysis-results}}
\addcontentsline{toc}{section}{Principal Component Analysis Results}

\begin{table}[H]
\centering
\begin{tabular}{rlll}
  \hline
 & (Intercept) & EM & EM\_HY \\ 
  \hline
1 &  & -1.60278736871812e-05 & 5.72316592929999e-05 \\ 
  2 & lag(ret) & (8.81517172979451e-05) & (8.4009832916283e-05) \\ 
  3 &  & 0.0626773846003959 *** & 0.068477837097736 *** \\ 
  4 & PC1 & (0.00768871355176577) & (0.00773785727574978) \\ 
  5 &  & -0.211323715240927 *** & -0.197839406763045 *** \\ 
  6 & PC2 & (0.0021054107455268) & (0.00200626418696844) \\ 
  7 &  & -0.181093632569239 *** & -0.174627648441878 *** \\ 
  8 & PC3 & (0.00455691863989387) & (0.0043357228611195) \\ 
  9 &  & 0.0553668947582391 *** & 0.0249326112521369 *** \\ 
  10 & N & (0.00540584158153239) & (0.00515349303427833) \\ 
  11 & R2 & 5217 & 5217 \\ 
  12 & *** p $<$ 0.001;  ** p $<$ 0.01;  * p $<$ 0.05. & 0.71193328402896 & 0.707540316745546 \\ 
  13 &  & *** p $<$ 0.001;  ** p $<$ 0.01;  * p $<$ 0.05. & *** p $<$ 0.001;  ** p $<$ 0.01;  * p $<$ 0.05. \\ 
   \hline
\end{tabular}
\caption{PCA Results} 
\end{table}
\begin{table}[H]
\centering
\begin{tabular}{rlllll}
  \hline
 & (Intercept) & EU & EU\_2 & EU\_DG & EU\_HY \\ 
  \hline
1 &  & -7.59999898186529e-06 & -1.10071320078484e-05 & 9.00587178365088e-05 & -2.67809526620728e-05 \\ 
  2 & lag(ret) & (7.10642224371418e-05) &  & (7.86321555212887e-05) & (8.04092632534596e-05) \\ 
  3 &  & -0.0226465097709638 *** & -0.0128979223892329 * & 0.00792276011442966 & -0.00153111734124876 \\ 
  4 & PC1 & (0.00576197722007948) & (0.00575507231249308) & (0.00795541237359177) & (0.00611274416639685) \\ 
  5 &  & -0.280883723797031 *** & -0.237445919816283 *** & -0.203588370755585 *** & -0.29397548972484 *** \\ 
  6 & PC2 & (0.00169729922318746) & (0.00139733652974771) & (0.00187659773140612) & (0.00191952935477372) \\ 
  7 &  & -0.031933386457366 *** & -0.0030388217366709 & -0.00621726079098141 & -0.0319940536542409 *** \\ 
  8 & PC3 & (0.00381344544966598) & (0.00319068365639232) & (0.00420207609930851) & (0.0043152899262049) \\ 
  9 &  & -0.155678383289745 *** & -0.135775502214704 *** & -0.127585411182103 *** & -0.17922756824018 *** \\ 
  10 & N & (0.00445119616658202) & (0.00368269427686028) & (0.00490212472352653) & (0.00503160857397482) \\ 
  11 & R2 & 5217 & 5217 & 5217 & 5217 \\ 
  12 & *** p $<$ 0.001;  ** p $<$ 0.01;  * p $<$ 0.05. & 0.855570865468774 & 0.861482054224444 & 0.719860264447883 & 0.836998608479176 \\ 
   \hline
\end{tabular}
\caption{PCA Results} 
\end{table}
\begin{table}[H]
\centering
\begin{tabular}{rllll}
  \hline
 & (Intercept) & JP & JP\_DG & JP\_HY \\ 
  \hline
1 &  & -9.99214323094622e-05 & -3.00703367334358e-05 & 3.4782188148871e-05 \\ 
  2 & lag(ret) & (7.46579373997006e-05) & (8.66511282175394e-05) & (8.20393212732995e-05) \\ 
  3 &  & 0.00566154299912426 & -0.00937744419485014 & -0.019606973971878 ** \\ 
  4 & PC1 & (0.00584090791543381) & (0.00725886667366239) & (0.00673256816192562) \\ 
  5 &  & -0.121814953493272 *** & -0.100873310563148 *** & -0.0845457504304278 *** \\ 
  6 & PC2 & (0.00178023967587767) & (0.00206606448195023) & (0.00195600102306667) \\ 
  7 &  & -0.402988732366009 *** & -0.348154775899471 *** & -0.397228501728333 *** \\ 
  8 & PC3 & (0.00376080427125469) & (0.00436402895533763) & (0.00413848635768162) \\ 
  9 &  & 0.389512643216594 *** & 0.362500842309265 *** & 0.36799907150433 *** \\ 
  10 & N & (0.00455716388404002) & (0.00528940026452518) & (0.0050263894091161) \\ 
  11 & R2 & 5217 & 5217 & 5217 \\ 
  12 & *** p $<$ 0.001;  ** p $<$ 0.01;  * p $<$ 0.05. & 0.822256241789988 & 0.725506416469889 & 0.766563221216072 \\ 
  13 &  & *** p $<$ 0.001;  ** p $<$ 0.01;  * p $<$ 0.05. & *** p $<$ 0.001;  ** p $<$ 0.01;  * p $<$ 0.05. & *** p $<$ 0.001;  ** p $<$ 0.01;  * p $<$ 0.05. \\ 
   \hline
\end{tabular}
\caption{PCA Results} 
\end{table}
\begin{table}[H]
\centering
\begin{tabular}{rllll}
  \hline
 & ...1 & SA & SA\_DG & SA\_HY \\ 
  \hline
1 & (Intercept) & 0.000174424066652349 & 0.000154599642594047 & 0.000331610039039925 \\ 
  2 &  & (0.000104831488132788) & (0.000135743539551013) & (0.000191341400418785) \\ 
  3 & lag(ret) & -0.0260201275509458 ** & 0.0162962308506602 & -0.214644994699929 *** \\ 
  4 &  & (0.00910355525354117) & (0.0116751097704669) & (0.0115045665456028) \\ 
  5 & PC1 & -0.192079580415969 *** & -0.125588385439684 *** & -0.155483613609965 *** \\ 
  6 &  & (0.00249964852578935) & (0.00323707057649618) & (0.00456228099800359) \\ 
  7 & PC2 & -0.134145033571322 *** & -0.113928595170158 *** & -0.22468771098418 *** \\ 
  8 &  & (0.00540516980619619) & (0.00688136494957077) & (0.00966127340855607) \\ 
  9 & PC3 & -0.127959733778058 *** & -0.12232707950307 *** & -0.154078714389083 *** \\ 
  10 &  & (0.00645043177866384) & (0.00830039210400402) & (0.0117282146734433) \\ 
  11 & N & 5217 & 5217 & 5217 \\ 
  12 & R2 & 0.595503774888565 & 0.302127435161338 & 0.31984720998824 \\ 
  13 & *** p $<$ 0.001;  ** p $<$ 0.01;  * p $<$ 0.05. & *** p $<$ 0.001;  ** p $<$ 0.01;  * p $<$ 0.05. & *** p $<$ 0.001;  ** p $<$ 0.01;  * p $<$ 0.05. & *** p $<$ 0.001;  ** p $<$ 0.01;  * p $<$ 0.05. \\ 
   \hline
\end{tabular}
\caption{PCA Results} 
\end{table}
\begin{table}[H]
\centering
\begin{tabular}{rlllll}
  \hline
 & ...1 & UK & UK\_B & UK\_HY & UK\_HY\_B \\ 
  \hline
1 & (Intercept) & -2.892336667118e-06 & -7.57345221694089e-05 & -0.000239298424245262 ** & -0.000146004759178592 \\ 
  2 &  & (6.50211737869564e-05) & (5.67565475934448e-05) & (8.96977996357748e-05) & (8.97879506987205e-05) \\ 
  3 & lag(ret) & -0.0387511686606318 *** & -0.0376835890925414 *** & 0.0286863872874779 *** & 0.00564174751399983 \\ 
  4 &  & (0.00650337136955424) & (0.0046125612803052) & (0.00812413977212375) & (0.00663098904930069) \\ 
  5 & PC1 & -0.221863534152518 *** & -0.280858203670326 *** & -0.218803222819842 *** & -0.283326297276248 *** \\ 
  6 &  & (0.00155225612418535) & (0.00135498861352277) & (0.00214205751004051) & (0.00214280428175452) \\ 
  7 & PC2 & -0.0178361496500972 *** & -0.0603959906146276 *** & -0.0227752473153885 *** & -0.0570357629364837 *** \\ 
  8 &  & (0.0035135719566689) & (0.00304188777775765) & (0.0047794082980455) & (0.00477795458383635) \\ 
  9 & PC3 & -0.141868473212785 *** & -0.191398139763492 *** & -0.166685531629944 *** & -0.232784790812376 *** \\ 
  10 &  & (0.00407108588530721) & (0.00354224579865907) & (0.00557254776270046) & (0.00559733103421363) \\ 
  11 & N & 5217 & 5217 & 5217 & 5217 \\ 
  12 & R2 & 0.818124055502306 & 0.906484923580086 & 0.704214574847334 & 0.803364808928872 \\ 
  13 & *** p $<$ 0.001;  ** p $<$ 0.01;  * p $<$ 0.05. & *** p $<$ 0.001;  ** p $<$ 0.01;  * p $<$ 0.05. & *** p $<$ 0.001;  ** p $<$ 0.01;  * p $<$ 0.05. & *** p $<$ 0.001;  ** p $<$ 0.01;  * p $<$ 0.05. & *** p $<$ 0.001;  ** p $<$ 0.01;  * p $<$ 0.05. \\ 
   \hline
\end{tabular}
\caption{PCA Results} 
\end{table}
\begin{table}[H]
\centering
\begin{tabular}{rlllll}
  \hline
 & ...1 & US & US\_2 & US\_DG & US\_HY \\ 
  \hline
1 & (Intercept) & -1.15532052845367e-05 & 1.63629844176586e-05 & 0.00014024400880847 ** & -2.7561557884578e-06 \\ 
  2 &  & (3.35413352219375e-05) & (4.60750384698967e-05) & (5.40237942081449e-05) & (3.83581975256876e-05) \\ 
  3 & lag(ret) & -0.0294796579219486 *** & -0.020354188614454 *** & 0.00320093608628991 & -0.0183892435379719 *** \\ 
  4 &  & (0.00341321696819235) & (0.00452661143914563) & (0.00552324285891578) & (0.00416906056214098) \\ 
  5 & PC1 & -0.233864956202208 *** & -0.230159358796096 *** & -0.208766048087452 *** & -0.208761657418179 *** \\ 
  6 &  & (0.000819555292176499) & (0.0011230023500934) & (0.00131460718338295) & (0.00093403325645807) \\ 
  7 & PC2 & 0.308957501335324 *** & 0.31675590488915 *** & 0.309811010196237 *** & 0.295054406083461 *** \\ 
  8 &  & (0.00199246158915227) & (0.00268173299058554) & (0.00310184861640173) & (0.0022461655392403) \\ 
  9 & PC3 & 0.267585573231036 *** & 0.267259539144946 *** & 0.249094505719275 *** & 0.24580475325279 *** \\ 
  10 &  & (0.00204680429842257) & (0.00281157396528542) & (0.00329509317963199) & (0.00234084919103889) \\ 
  11 & N & 5217 & 5217 & 5217 & 5217 \\ 
  12 & R2 & 0.958636359037242 & 0.923969196587223 & 0.883575207473768 & 0.936197300328372 \\ 
  13 & *** p $<$ 0.001;  ** p $<$ 0.01;  * p $<$ 0.05. & *** p $<$ 0.001;  ** p $<$ 0.01;  * p $<$ 0.05. & *** p $<$ 0.001;  ** p $<$ 0.01;  * p $<$ 0.05. & *** p $<$ 0.001;  ** p $<$ 0.01;  * p $<$ 0.05. & *** p $<$ 0.001;  ** p $<$ 0.01;  * p $<$ 0.05. \\ 
   \hline
\end{tabular}
\caption{PCA Results} 
\end{table}
\begin{table}[H]
\centering
\begin{tabular}{rlll}
  \hline
 & (Intercept) & W & W\_HY \\ 
  \hline
1 &  & -2.68877305546502e-05 & -1.60412046201327e-05 \\ 
  2 & lag(ret) & (2.66934004733464e-05) & (3.62741169219996e-05) \\ 
  3 &  & 0.00173290808467087 & 0.012868806430677 ** \\ 
  4 & PC1 & (0.00305791906183213) & (0.00413386998598336) \\ 
  5 &  & -0.232673492817017 *** & -0.22906874305533 *** \\ 
  6 & PC2 & (0.000644347752549236) & (0.00087055856947328) \\ 
  7 &  & 0.120226620730606 *** & 0.0858355323895155 *** \\ 
  8 & PC3 & (0.0015371543580596) & (0.0020483834598137) \\ 
  9 &  & 0.149969169492583 *** & 0.0454159637409487 *** \\ 
  10 & N & (0.00163434044095793) & (0.00223590470954708) \\ 
  11 & R2 & 5217 & 5217 \\ 
  12 & *** p $<$ 0.001;  ** p $<$ 0.01;  * p $<$ 0.05. & 0.964070088675355 & 0.931627271013059 \\ 
   \hline
\end{tabular}
\caption{PCA Results} 
\end{table}
\newpage

\hypertarget{references}{%
\section*{References}\label{references}}
\addcontentsline{toc}{section}{References}

\hypertarget{refs}{}
\begin{CSLReferences}{1}{0}
\leavevmode\vadjust pre{\hypertarget{ref-brinson1985measuring}{}}%
Brinson, G.P. \& Fachler, N. 1985. Measuring non-US. Equity portfolio
performance. \emph{The Journal of Portfolio Management}. 11(3):73--76.

\leavevmode\vadjust pre{\hypertarget{ref-gordon1962}{}}%
Gordon, M.J. 1962. The savings investment and valuation of a
corporation. \emph{The Review of Economics and Statistics}. 37--51.

\leavevmode\vadjust pre{\hypertarget{ref-miller}{}}%
Miller, M.H. \& Modigliani, F. 1961. Dividend policy, growth, and the
valuation of shares. \emph{The Journal of Business}. 34(4):411--433.

\end{CSLReferences}

\bibliography{Tex/ref}





\end{document}
