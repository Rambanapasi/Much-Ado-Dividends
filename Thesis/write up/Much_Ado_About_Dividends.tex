\documentclass[11pt,preprint, authoryear]{elsarticle}

\usepackage{lmodern}
%%%% My spacing
\usepackage{setspace}
\setstretch{1}
\DeclareMathSizes{12}{14}{10}{10}

% Wrap around which gives all figures included the [H] command, or places it "here". This can be tedious to code in Rmarkdown.
\usepackage{float}
\let\origfigure\figure
\let\endorigfigure\endfigure
\renewenvironment{figure}[1][2] {
    \expandafter\origfigure\expandafter[H]
} {
    \endorigfigure
}

\let\origtable\table
\let\endorigtable\endtable
\renewenvironment{table}[1][2] {
    \expandafter\origtable\expandafter[H]
} {
    \endorigtable
}


\usepackage{ifxetex,ifluatex}
\usepackage{fixltx2e} % provides \textsubscript
\ifnum 0\ifxetex 1\fi\ifluatex 1\fi=0 % if pdftex
  \usepackage[T1]{fontenc}
  \usepackage[utf8]{inputenc}
\else % if luatex or xelatex
  \ifxetex
    \usepackage{mathspec}
    \usepackage{xltxtra,xunicode}
  \else
    \usepackage{fontspec}
  \fi
  \defaultfontfeatures{Mapping=tex-text,Scale=MatchLowercase}
  \newcommand{\euro}{€}
\fi

\usepackage{amssymb, amsmath, amsthm, amsfonts}

\def\bibsection{\section*{References}} %%% Make "References" appear before bibliography


\usepackage[round]{natbib}

\usepackage{longtable}
\usepackage[margin=2.3cm,bottom=2cm,top=2.5cm, includefoot]{geometry}
\usepackage{fancyhdr}
\usepackage[bottom, hang, flushmargin]{footmisc}
\usepackage{graphicx}
\numberwithin{equation}{section}
\numberwithin{figure}{section}
\numberwithin{table}{section}
\setlength{\parindent}{0cm}
\setlength{\parskip}{1.3ex plus 0.5ex minus 0.3ex}
\usepackage{textcomp}
\renewcommand{\headrulewidth}{0.2pt}
\renewcommand{\footrulewidth}{0.3pt}

\usepackage{array}
\newcolumntype{x}[1]{>{\centering\arraybackslash\hspace{0pt}}p{#1}}

%%%%  Remove the "preprint submitted to" part. Don't worry about this either, it just looks better without it:
\makeatletter
\def\ps@pprintTitle{%
  \let\@oddhead\@empty
  \let\@evenhead\@empty
  \let\@oddfoot\@empty
  \let\@evenfoot\@oddfoot
}
\makeatother

 \def\tightlist{} % This allows for subbullets!

\usepackage{hyperref}
\hypersetup{breaklinks=true,
            bookmarks=true,
            colorlinks=true,
            citecolor=blue,
            urlcolor=blue,
            linkcolor=blue,
            pdfborder={0 0 0}}


% The following packages allow huxtable to work:
\usepackage{siunitx}
\usepackage{multirow}
\usepackage{hhline}
\usepackage{calc}
\usepackage{tabularx}
\usepackage{booktabs}
\usepackage{caption}


\newenvironment{columns}[1][]{}{}

\newenvironment{column}[1]{\begin{minipage}{#1}\ignorespaces}{%
\end{minipage}
\ifhmode\unskip\fi
\aftergroup\useignorespacesandallpars}

\def\useignorespacesandallpars#1\ignorespaces\fi{%
#1\fi\ignorespacesandallpars}

\makeatletter
\def\ignorespacesandallpars{%
  \@ifnextchar\par
    {\expandafter\ignorespacesandallpars\@gobble}%
    {}%
}
\makeatother

\newenvironment{CSLReferences}[2]{%
}

\urlstyle{same}  % don't use monospace font for urls
\setlength{\parindent}{0pt}
\setlength{\parskip}{6pt plus 2pt minus 1pt}
\setlength{\emergencystretch}{3em}  % prevent overfull lines
\setcounter{secnumdepth}{5}

%%% Use protect on footnotes to avoid problems with footnotes in titles
\let\rmarkdownfootnote\footnote%
\def\footnote{\protect\rmarkdownfootnote}
\IfFileExists{upquote.sty}{\usepackage{upquote}}{}

%%% Include extra packages specified by user
\usepackage{booktabs}
\usepackage{longtable}
\usepackage{array}
\usepackage{multirow}
\usepackage{wrapfig}
\usepackage{float}
\usepackage{colortbl}
\usepackage{pdflscape}
\usepackage{tabu}
\usepackage{threeparttable}
\usepackage{threeparttablex}
\usepackage[normalem]{ulem}
\usepackage{makecell}
\usepackage{xcolor}

%%% Hard setting column skips for reports - this ensures greater consistency and control over the length settings in the document.
%% page layout
%% paragraphs
\setlength{\baselineskip}{12pt plus 0pt minus 0pt}
\setlength{\parskip}{12pt plus 0pt minus 0pt}
\setlength{\parindent}{0pt plus 0pt minus 0pt}
%% floats
\setlength{\floatsep}{12pt plus 0 pt minus 0pt}
\setlength{\textfloatsep}{20pt plus 0pt minus 0pt}
\setlength{\intextsep}{14pt plus 0pt minus 0pt}
\setlength{\dbltextfloatsep}{20pt plus 0pt minus 0pt}
\setlength{\dblfloatsep}{14pt plus 0pt minus 0pt}
%% maths
\setlength{\abovedisplayskip}{12pt plus 0pt minus 0pt}
\setlength{\belowdisplayskip}{12pt plus 0pt minus 0pt}
%% lists
\setlength{\topsep}{10pt plus 0pt minus 0pt}
\setlength{\partopsep}{3pt plus 0pt minus 0pt}
\setlength{\itemsep}{5pt plus 0pt minus 0pt}
\setlength{\labelsep}{8mm plus 0mm minus 0mm}
\setlength{\parsep}{\the\parskip}
\setlength{\listparindent}{\the\parindent}
%% verbatim
\setlength{\fboxsep}{5pt plus 0pt minus 0pt}



\begin{document}



\begin{frontmatter}  %

\title{Much Ado About Dividends}

% Set to FALSE if wanting to remove title (for submission)




\author[Add1]{Gabriel Rambanapasi}
\ead{gabriel.rams44@gmail.com}

\author[Add2]{Nico Katzke}
\ead{}




\address[Add1]{Stellenbosch University, Cape Town, South Africa}
\address[Add2]{Satrix, Cape Town, South Africa}

\cortext[cor]{Corresponding author: Gabriel Rambanapasi}

\begin{abstract}
\small{
This paper tests wheter dividends signals provide informative return
predictors. We consider total return global indexes constructed using
high yield (HY) or dividend growth per share signals (DG) over a sample
of 20 years. We find that global indexes fail to meaningfully outperform
their market index as measured by cumulative excess return and mean
return. Upon stratification, we observe that all stratgeies offer
defensive charcateristics in periods of high volatility and HY
strategies give their highest return in periods of low interest rates.
Underscoring our total return analysis we conclude that all strategies
have underperfomed on a rolling 60 month basis in most geographies,
hence they are a poor proxy for capturing the value premium. For the
local, we construct our own dividend portfolios, using a Maximum Utility
Optimizer, applying a ranking criteria similar to Damodaran
(\protect\hyperlink{ref-damodaran2004investment}{2004}) and find that
both DG and HY underperform the Capped SWIX Top 50. However, our value
proxy (price to earnings) outperforms our dividend signals. We opine
that if investors are inclined to have dividend exposure in their,
portfolio we recommnded investors use DG considering that being a delta
to HY offers a purer measure of value.
}
\end{abstract}

\vspace{1cm}





\vspace{0.5cm}

\end{frontmatter}

\setcounter{footnote}{0}



%________________________
% Header and Footers
%%%%%%%%%%%%%%%%%%%%%%%%%%%%%%%%%
\pagestyle{fancy}
\chead{}
\rhead{}
\lfoot{}
\rfoot{\footnotesize Page \thepage}
\lhead{}
%\rfoot{\footnotesize Page \thepage } % "e.g. Page 2"
\cfoot{}

%\setlength\headheight{30pt}
%%%%%%%%%%%%%%%%%%%%%%%%%%%%%%%%%
%________________________

\headsep 35pt % So that header does not go over title




\hypertarget{introduction}{%
\section*{Introduction}\label{introduction}}
\addcontentsline{toc}{section}{Introduction}

This paper seeks to investigate the return signalling cue of dividend
paying stock. The value of dividends towards shareholders has long been
debated with academics and practitioners providing evidence for their
irrelevance and relevance. Miller \& Modigliani
(\protect\hyperlink{ref-miller}{1961}) proposed the dividend irrelevance
theorem that essentially argues that dividend payments were irrelevant
as this detracted from the shareholder value as wealth is affected by
the income a firm generates, not the way the firm distributes that
income. Against this, Gordon (\protect\hyperlink{ref-gordon1962}{1962})
argued that cash flows should be preferred by investors , as they are
certain, as opposed to the riskier capital gains according this theory,
Bird in hand Theorem is used to justify demand for dividend stock,
especially for the less riskier investor. Moreover, a deeper assessment
of the MM theory reveals unrealistic assumptions, rendering arguments
immaterial once real world constraints such as taxes and transactions
cost, behavioural biases are considered. Consequently theories that
consider information asymmetries, tax considerations, signalling all
give convincing arguments of dividend payment address some real world
considerations. Campbell
(\protect\hyperlink{ref-campbell2017financial}{2017}) opines that the
value of a stock is a function of its future cashflow, if markets are
assumed to be efficient then companies that do not pay dividends should
have a value of zero unless there is some expectation of the future
reciepts of investment is said company. Therefore, we can safely assume
a direct relationship between share price and new information that
affects future cash flows. Event studies on types dividend payments
address this and often times it is noted that the payment of dividends
leads to a decrease in prices thus stock valuation, Suwanna
(\protect\hyperlink{ref-suwanna2012impacts}{2012}) shows that dividend
announcement decrease share prices to the value of the dividend. This
could be that dividend are a capital budgeting decision and their
payment reduces retained earnings and consequently affects that the
share price.

This study provides an extensive investigation into the return signaling
of dividend portfolios. Firstly, looking at past literature on dividend
payments, their rationale, theories for and against their relevance in
the literature review. Secondly, outline our methodology in constructing
our dividend portfolios, which includes our utility optimizer along with
constraints. Thirdly, start to discuss the results of our study by
answering the first question of when do dividend signals work? To
achieve this we first look at the cumulative excess returns offered by
globally dividend indexes, to give a holistic picture of geographical
dividend portfolio performance. Our analysis shows us that dividend
portfolios, wheter high yield (HY) or dividend growth (DG) do not give
an clear signal to return, however once we stratify our sample according
to interest rate cycles (Hiking, Cutting and Neutral) and market regimes
(High and Low Volatility), unique to geography and we start to find that
within market regimes, dividend signals offer defensive characteristics,
thus giving higher excess returns in periods of High Volatility. This
does not infer positive return for all indexes, our proxies for United
States (US) HY , UK HY and World indexes returned negative over their
respective periods. On performance consistency, rolling information
ratio, we note an inherent lack of consistency in dividend portfolios
performance, with most staretgies underperforming the benchmark over the
past 15 years. However, coupled with draw downs experienced in different
jurisdictions, we acknowledge that advanced market dividend portfolios
experience the least variation in draw downs compared to emerging
markets. The results indicate that both growth and high-yield dividend
portfolios tend to under perform relative to their benchmarks, raising
questions about their ability to extract a value premium. Lastly, move
on to how dividend portfolios work, with the goal of constructing a
portfolio that can best harness the existing premium. We use share data
from the Johannesburg Stock Exchange (JSE), with our benchmark being the
Capped SWIX Top 50. We construct four portfolios using 3-month and 9
month measures of HY, 1 and 3 year DG measures together with price
momentum and dividend coverage ratios. Similar to the analysis on
international dividend portfolios, our portfolios exhibit defensive
properties in High volatility periods but fail to capture a premium
consistently over investing in a market index.

\hypertarget{literature-review}{%
\section*{Literature Review}\label{literature-review}}
\addcontentsline{toc}{section}{Literature Review}

\hypertarget{what-are-dividends}{%
\subsection*{What are dividends}\label{what-are-dividends}}
\addcontentsline{toc}{subsection}{What are dividends}

Dividends constitute a form of capital distribution by corporations
towards shareholders. They exist in various forms, such as cash, stock,
liquidating, scrip, or property dividends Baker \& Powell
(\protect\hyperlink{ref-baker1999corporate}{1999}), of which cash
dividends and share repurchases being the most commonly used in
practice. Within cash dividends, regular dividends are widely used by
corporations and payment frequency across jurisdictions. The decision to
issue dividends is typically made by the board of directors, and
approved by shareholders, however practiced more in Europe and less so
in the United States. The payout policy policy of a corporation, which
are guiding principles for management and board of directors towards
capital distributions considers company investment and is closely
watched by investors and analysts. As such, management strives to grow
or maintain a certain level of dividend payouts as this signals firm
growth and investors share of profitability in the company.

\hypertarget{theorectical-arguments-on-dividend-payments}{%
\subsection*{Theorectical Arguments on Dividend
Payments}\label{theorectical-arguments-on-dividend-payments}}
\addcontentsline{toc}{subsection}{Theorectical Arguments on Dividend
Payments}

Given the apparent decrease in shareholder value, the logical question
has encouraged a long running debate on dividend relevance and
irrelevance. In 1961, Miller \& Rock
(\protect\hyperlink{ref-miller1985dividend}{1985}) opined that dividends
are irrelevant (MM theory), he argued that shareholders are indifferent
to dividend payments, thus implying that there is no optimal dividend
policy and that all dividend policies are equally good and payments of
dividends could easily be reinvested in shares and make no difference to
share holder wealth. However, the MM theorem fails to consider
real-world market imperfections that may give relevance to dividend
payments. The bird in the arguments opposes the MM theory, suggesting
that investor would prefer to receive less risky cash flow in the form
of dividends instead of potential capital gains at some point in the
future (\protect\hyperlink{ref-gordon1962}{Gordon, 1962}). This
permeates to the cost of equity, since dividends are less risky,
companies that issue more dividends should have higher share prices.
However, propoents of the MM theory contend this suggesting the risk of
future cash flow is affected by the payment of dividend, leading to
negative effects on share prices after the ex-dividend date. The
dividend puzzle considers real world constraints and gives an
interesting take on its relevance and irrelevance, by suggesting that
dividends reduce equity value and make investors worse off; however, are
a reward to investors who bear the risk associated with their
investments as it provides an additional source of return on investment
from a share Black (\protect\hyperlink{ref-black1996dividend}{1996}).
Various literature has made convincing arguments for corporations to pay
dividends which include tax considerations, dividend signalling and
agency costs in issuing dividends.Tax considerations argue in favor for
dividend relevance. Across jurisdiction dividends have different tax
treatments to capital gains and often tax at a higher income tax rate,
thus investors that have higher tax rates choose stocks with lower
dividend payouts and transversly pushes up the stock price, this is
called the clientele effect Baker \& Powell
(\protect\hyperlink{ref-baker1999corporate}{1999}). However a major
pushback emanates from proponents of the MM theory, that suggest the
client effect causes major substitution effect, meaning that if
companies change their dividend policy, investors with preferential tax
treatment will simply allocate more capital to that stock and those out
of favor will sell their shares. Given the large number of investors
versus listed companies the process is instantaneously causing a net
zero effect on prices(\protect\hyperlink{ref-baker1999corporate}{Baker
\& Powell, 1999}). Second, flotation costs refer to the opportunity
costs incurred by a firm when paying dividends. Through distributing
dividends, companies forego opportunities to expand their operations
using retained earnings. In a world without flotation costs, as
suggested by the MM theorem, management would be indifferent between
issuing dividends and borrowing from the market thus have no effect on
shares prices. However, in reality, external financing comes at a higher
cost, leading to trade-offs in dividend policy decisions and ultimately
share prices.

Information asymmetry between shareholders and managers is another
factor that gives relevance to dividend payments. Managers of businesses
have greater knowledge of operations thus value of a business at any
given point more than shareholders. As such, investors rely on dividend
announcements to assess a company's valuation. Dividend signaling
conveys information about the company's quality Baker \& Powell
(\protect\hyperlink{ref-baker1999corporate}{1999}). Investors compare
dividend announcements to historical levels while considering company
fundamentals. However, a major concern towards its ability to be
``gamed' by management, making the dividend signal imperfect for
determining share prices. Principal agency issues may give another
reason for issuance of dividends. The free cash flow hypothesis suggests
that dividend payments force management to raise capital from external
sources, which increases borrowing costs and scrutiny from capital
markets. This, in turn, reduces management's ability to make sub optimal
investments and aligning management and shareholder objectives
(\protect\hyperlink{ref-baker1999corporate}{Baker \& Powell, 1999}).
Supporters of this theory ascertain that dividends payments by the
mechanism encourage good business practices.

\hypertarget{constructed-dividend-portfolios}{%
\subsection*{Constructed Dividend
Portfolios}\label{constructed-dividend-portfolios}}
\addcontentsline{toc}{subsection}{Constructed Dividend Portfolios}

From the dividend relevance theories we can conclude that the payments
of dividends emanates from proxy arguments (as opposed to it being
considered an attractive feature in itself). High dividend-paying
companies could proxy for the quality of management structures over time
(through their ability to consistently afford dividend payments) or
similarly point to prudent cash-flow management capabilities and stock
closely resemble value stock. Basu
(\protect\hyperlink{ref-basu1977investment}{1977}) used price to
earnings ratios of companies to predict stock performance of companies,
and found that low price to earnings stock overtime out perform their
more expensive counterparts that have high PE ratios. So when we
consider the following,
\(D Y=\frac{E P S}{\text { Price }} \times Payout Ratio\), dividend
yield, once payout ratio is held constant, is a function of earnings
yield, therefore identifying low price to earnings through dividends can
lead us to use it as a proxy for value. This measure is not full proof,
value stock carry high levels of risk as there more prone to be under
financial distress and face uncertainty in future earnings
(\protect\hyperlink{ref-chen1998risk}{Chen \& Zhang, 1998}). For this
reason, the use of dividend growth (DG) as a signal provides attributes
that aim to curtail negative aspects of dividend yield (HY). That is, DG
stock, unlike HY, it is not affected by price but maintains properties
that allow for inference into management quality. As management is aware
of the signaling effect of dividends, this may induce the value trap,
forcing management to continually increase dividends to maintain a
certain valuation. However, such companies are more vulnerable to facing
financial distress
(\protect\hyperlink{ref-baker2009understanding}{\textbf{baker2009understanding?}}).
O'higgins \& Downes (\protect\hyperlink{ref-o1991beating}{1991})
constructed a portfolio of companies that used constituents of the Dow
Jones Industrial Average (DJIA) called the ``Dogs of the Dow'' (DOD). By
ranking 30 companies by HY and including only the 10 highest-yielding
shares in a portfolio, this achieved a return higher than the DJIA
(16.6\% per annum versus the DJIA's 10.4\%). This had lower risk than
the DJIA, thus achieving a higher Sharpe Ratio. Testing O'higgins \&
Downes (\protect\hyperlink{ref-o1991beating}{1991})'s strategy resulted
in ``Beat the Dow 5,'' which involved annually investing in only the
five lowest-priced of the HY10 shares each year, in other words, high
dividend yield. This strategy gave superior returns of 19.4\% versus the
DJIA. As opined by Gardner, Gardner \& Maranjian
(\protect\hyperlink{ref-gardner2002motley}{2002}), this strategy
leverages the fact that low-priced stocks experience the most
volatility, by courting future volatility in the 10 stocks that have
some potential upside, expecting their stock prices to rise in return.
Other studies across jurisdiction came to similar conclusions, Lemmon \&
Nguyen (\protect\hyperlink{ref-lemmon2015dividend}{2015}) in Hong Kong,
Brzeszczyński \& Gajdka
(\protect\hyperlink{ref-brzeszczynski2007dividend}{2007}) in Poland,
Visscher \& Filbeck (\protect\hyperlink{ref-visscher2003dividend}{2003})
in Canada, Filbeck \& Visscher
(\protect\hyperlink{ref-filbeck1997}{1997}) in Britain, and Wang,
Larsen, Ainina, Akhbari \& Gressis
(\protect\hyperlink{ref-wang2011dogs}{2011}) in China. More recently,
Filbeck, Holzhauer \& Zhao
(\protect\hyperlink{ref-filbeck2017dividend}{2017}) investigated the
performance of DOD against a high-yield portfolio of Fortune Most
Desired Companies (MAC) compared to the Dow Jones Industrial Average and
the S\&P 500. The study found significantly higher risk-adjusted returns
for the DOD strategy. In South Africa, Fakir \& others
(\protect\hyperlink{ref-fakir2013dividend}{2013}) employs a parametric
approach to investigate dividends as an investment strategy.

\hypertarget{methodology}{%
\section*{Methodology}\label{methodology}}
\addcontentsline{toc}{section}{Methodology}

To evaluate the return predictive signal of dividends, we employ an
applied approach that constitutes constructing subset portfolios and
compare in sample performances. Our approach aims to give valuable
insights based on risk and return for systematically constructed
dividend portfolios.

\hypertarget{portfolio-optimzation}{%
\subsection*{Portfolio optimzation}\label{portfolio-optimzation}}
\addcontentsline{toc}{subsection}{Portfolio optimzation}

The Modern Portfolio Theory defines risk of a portfolio of (\(n\))
assets as the variance (\(\sigma^2\)) of its returns (\(r_{t}\)). We add
a refinement to this, and our definition returns is achieved by
decomposing it into common factor (\(Xf\)) and specific return (\(u\))
as (\(r = Xf+ u\)). From these returns we create a factor covariance
matrix, defined as (\(X F X^T+D\)) in which we derive our multiple
factor universe consists of (\(k\)) common factors.

\(\begin{array} {ll}X & =n \times k \text { matrix of asset exposures to the factors, } \\ F \quad &= k \times k \text { positive semi-definite factor covariance matrix, and } \\ D \quad &=n \times n \text { positive semi-definite covariance matrix representing a } \\ & \text { forecast of asset specific risk. }\end{array}\)

We periodically calculate each asset exposure to the common factors
calculated in the factor covariance matrix. This then assists us in
computing forecasts of the level of each asset specific risk. The short
term risk forecasts will then be used to gauge contribution of each
asset to a portfolio over risk which contributes to the portfolio
construction process. For our optimization, risk takes on two forms
being total risk (only portfolio holdings are considered and benchmark
holdings are irrelevant for the optimization process) and active risk
(difference between portfolio holdings and benchmark holdings are given
consideration in the optimization problem).

Defined as: \emph{Total Risk}:
\(\quad h^T\left(\lambda_F X F X^T+\lambda_D D\right) h\)

\emph{Active Risk}:
\(\left(h-h_B\right)^T\left(\lambda_F X F X^T+\lambda_D D\right)\left(h-h_B\right)\)

where, \[
\begin{aligned}
\lambda_F & =\text { common factor risk aversion parameter, } \\
\lambda_D & =\text { specific risk aversion parameter, } \\
h & =n \times 1 \text { vector of managed portfolio's holdings, and } \\
h_B & =n \times 1 \text { vector of normal (benchmark) portfolio's holdings }
\end{aligned}
\] \#\# Constraints\{-\}

Optimization involves using set of constraints that helps in attaining
convergence, i.e giving a set of weights that determine our optimum
portfolio. In practice, this is unique to portfolio managers given their
risk objectives and goals\footnote{see
  \url{https://www.sciencedirect.com/science/article/pii/S1057521921002556}
  for a detailed explanation on advantages of using maximum utility
  operators to efficiently factor investor risk preferences}.

We use:

\begin{itemize}
\tightlist
\item
  Common factor and specific risk aversion parameters of 0.0075 and 1,
  respectively.
\item
  Our investment universe is the Top 50 stock listed on the
  JSE,therefore our selection criteria depends on market capitalization
  and liquidity. We use the Capped SWIX as the benchmark.
\item
  Portfolios are re-balanced quarterly.
\item
  Active risk constraints to parent benchmark 5\%
\item
  Our sector exposure has a +/-10\% limit; holds no property stocks in
  the portfolio
\item
  Individual stock have a 15\% max exposure limit
\item
  Quarterly turnover is limited to 10\%.\\
\end{itemize}

\hypertarget{tax-considerations}{%
\subsubsection*{Tax considerations}\label{tax-considerations}}
\addcontentsline{toc}{subsubsection}{Tax considerations}

Portfolio theory was developed in a perfect world without friction. In
practice, frictions need to be considered and in portfolio construction
this often entails considering the effect of taxes on income and capital
gains as they can erode returns and significantly alter risks and return
characteristics of shares. The contribution of dividends and capital
gains to total return can lead to varying tax inefficiencies for shares
as most jurisdictions imposed higher taxes than on capital gains.
Therefore shares with higher contribution of dividends will be less tax
efficient than those with a higher capital gains component and with
timing most jurisdictions tax dividends in the year that they are
receive\footnote{See Deloitte's tax guides and country highlights:
  \url{https://dits.deloitte.com/\#TaxGuides}}.

Jurisdictional laws can also affect the distribution of taxable returns
amongst shares depending on their class namely ordinary shares or
preferred shares. Preferred shares are viewed as a substitute for bonds
and income from preferred shares are often given tax at a lower rate
than those from dividends from ordinary shares.

We will not survey global tax regimes or incorporate all potential tax
complexities into the portfolio construction but assume a high level
commonalities exists amongst all jurisdictions this study uses. This is
a reasonable assumption considering the summary of taxes on dividends
and capital gains from major economies. For simplicity, we will assume a
basic tax regime includes the key elements of investment-related taxes
that are representative of what a typical taxable asset owner of a
global portfolio will contend with. The proposed methodology to employ
on the dividend portfolios use the following methodology.

\begin{align}
r_{a t}=p_d r_{p t}\left(1-t_d\right)+p_a r_{p t}\left(1-t_{c g}\right) \notag
\end{align}

where: \[
\begin{aligned}
r_{at} & =\text { the after tax return} \\
p_{d} & =\text {total return from to dividend income } \\
p_{a} &= \text { total return from capital appreciation } \\
t_{d}&= \text { tax on dividends}\\
t_{a}& = \text { tax on capital gains }
\end{aligned}
\]

\hypertarget{dividend-signals}{%
\subsection*{Dividend Signals}\label{dividend-signals}}
\addcontentsline{toc}{subsection}{Dividend Signals}

In constructing our portfolios, we rank the stocks on the capped SWIX
according to the either a dividend yield and dividend growth per share.
Our in house measures for dividend yield (DY 3m fwd and DY 9m fwd) are
constructed based on industry analyst estimates on future earnings,
hence the appeal is that the measure is a forward looking
estimate\footnote{ this ensures that we mitigate against the \emph{value
  trap}.} Similarly, dividend growth measure of either 1 and 3 years are
constructed on forward looking estimates for the same reasons. We employ
a price momentum and dividend coverage filters to construct portfolios
that reward companies which have had strong price momentum and
sustainable dividend practices{[}mention studies that have shown that
employing price momentum filters helps with picking the right stock for
a portfolio{]}.

\hypertarget{divi1}{%
\subsubsection*{DIVI1}\label{divi1}}
\addcontentsline{toc}{subsubsection}{DIVI1}

Rank score (i.e.~between 0 and 100) calculated using:

We use a combination of 2/3 DY (3m fwd) and 1/3 DY (9m fwd), Dividend
Coverage Ratio and Price Momentum. The signal uses conditions: - if
dvd\_cover score is in bottom quintile, then add it at 15\% (15\% dvd
cover, 66.667\% * 0.85 DY3m, 33.333\% * 0.85) if Price momentum score is
in bottom quintile, then add it at 35\% (35\% PX momentum score =
66.667\% * 0.65 DY3m, 33.333\% * 0.65) - if both dvdcover and momentum
in bottom quintile, then: ( 15\% dvd cover score= 35\% PX momentum
score, 66.667\% * 0.5 DY3-month, 33.333\% * 0.5)

This portfolio uses price momentum and dividend cover ratio as filters
to the dividend yield. This rewards sustainability in dividend paymensts
and avoids comapnies that companies that can not afford to pay
shareholders dividends thus avoid reactionary capital gain losses.

\hypertarget{divi2}{%
\subsubsection*{DIVI2}\label{divi2}}
\addcontentsline{toc}{subsubsection}{DIVI2}

Rank score (i.e.~between 0 and 100):

We use the dividend as the only signal. Similar to Divi1 its a blend of
forwarding looking metrics. That is, 2/3 DY (3m fwd) and 1/3 DY (9m
fwd).

This is our vanilla dividend yield portfolio i.e.~just ranks according
to the highest dividend payers.

\hypertarget{divi3}{%
\subsubsection*{DIVI3}\label{divi3}}
\addcontentsline{toc}{subsubsection}{DIVI3}

Rank score (i.e.~between 0 and 100):

We use the P/E ratio as the alternative proxy to value.

\hypertarget{divi4}{%
\subsubsection*{DIVI4}\label{divi4}}
\addcontentsline{toc}{subsubsection}{DIVI4}

Rank score (i.e.~between 0 and 100) calculated using:

Using: DPS\_Growth\_1Y = 40\%, DPS\_Growth\_3Y = 30\%, Fwd\_3 = 20\%,
Fwd\_9 = 10\%

Our dividend growth portfolio using trailing dividend growth rates
coupled with 3 and 9 month forward measures.

\hypertarget{data}{%
\subsection*{Data}\label{data}}
\addcontentsline{toc}{subsection}{Data}

Our metric of interests in this study are return and risk measures to
evaluate dividend signals as an ionvestment strategy. We use data from
dividend portfolios over the period 30/06/03--30/06/23, with the start
date and the end date purely driven by data availability on the selected
dividend indices at the time of writing. We obtained historical daily
price data of the dividend portfolios and constituents of the Capped
SWIX Top 50 listed in the Johannesburg Stock Exchange (JSE) from
Bloomberg\footnote{see \ref{reftab} for a detailed guide to indices used
  and codenames used later in the results and analysis}. Also we
retrieved volatility and interest rate proxies for geograhies under
investigation for the same sample period as our dividend portfolios.
That is, Chicago Board of Options Exchange (CBOE) VIX Index for the US
and EM, V2X for Europe, IVUK for UK and JALSH VR for SA volatility
proxies. For interest rate data we considered policy rates for central
banks for instruments geography within our study, these are the Federal
Fund rate for the US and EM, Minimum Deposit Financing Rate for the EU,
Bank of England Bank Rate and the South African Reserve Bank Repo rate.
To this end, we calculate our excess returns, we geometrically chain the
excess returns for the different periods before annualizing. This
produces comparable cumulative annualized excess return (CAER) results
in \ref{tab1}, defined as: \begin{align}
C A E R=\left[\prod_{t=1}^n\left(1+E R_t\right)\right]^{\frac{222}{n}}-1 \notag
\end{align}

Our rule to identifying volatility periods either high volatility
(Hi-vol) or low volatility (Lo-vol) is achieved by computing the top and
bottom quantile in standard deviation for our respective proxies. We
then pull the dates corresponding to the periods, and compute annualized
returns after geometrically chaining the monthly returns. The amount of
daily data for the respective interest rate cycles is large enough to
annualized, however, when the VIX, V2X or JALSH RV breach the top or
bottom quintile for less than 50 trading days, the period is excluded in
order to avoid annualizing small samples. To stratifying between Hiking,
Cutting and Neutral interest rate cycles we define these periods as
either 5 quarters of changes (upwards for Hiking and downwards for
Cutting) or otherwise if central bank held interest rates constant.

\hypertarget{results-and-discussion}{%
\section*{Results and Discussion}\label{results-and-discussion}}
\addcontentsline{toc}{section}{Results and Discussion}

\hypertarget{when-do-dividend-strategies-work}{%
\subsection*{When Do Dividend Strategies
Work}\label{when-do-dividend-strategies-work}}
\addcontentsline{toc}{subsection}{When Do Dividend Strategies Work}

We first evaluate several performance metrics of globally traded
dividend portfolios\footnote{ Table \ref{tab1} does not compare across
  indexes and regions as instruments have different inception dates}.
Table \ref{tab1} shows four performance measures in relative return
(annualized excess return) and total return(cumulative return), maximum
draw downs and standard deviation. The last two measures give an idea of
risk characteristics of our selected investment universe, however an in
depth analysis will follow in our analysis. Immediately we notice that
total returns as measured by cumulative returns for the dividend
portfolios that we investigate are positive for approximately 42\% of
the sample. Specifically, HY strategies provide the highest total
returns, and coupled with annualized excess returns assessement, we
notice that most strategies beat their benchmark, however, their
relative performance against their respective benchmarks is marginal
with relative performance being +/- 0.01\%. Looking at risk from a very
high level, there exists a tight range in annualized standard deviation
with most strategies, however draw downs vary across strategies and
regions and dont give a clear pciture of return characteristics of the
constituents, but broadly HY strategies have lower draw downs than DG
strageties. Therefore, from table \ref{tab1} there exists no clear
evidence that dividend signals are able to capture some premium over the
market index regardless of region.

\begin{table}[H]
\centering
\begin{tabular}{rlrrrr}
  \hline
 & Index & Ann Return & Std dev & Max Drawdowns & Cumulative Return \\ 
  \hline
1 & EM\_HY & -0.00 & 0.04 & 0.13 & 0.58 \\ 
  2 & SA\_HY & 0.01 & 0.04 & 0.17 & 0.22 \\ 
  3 & SA\_DG & 0.01 & 0.04 & 0.19 & 0.21 \\ 
  4 & EU\_DG & 0.01 & 0.03 & 0.14 & 0.11 \\ 
  5 & EU\_HY & -0.00 & 0.04 & 0.24 & 0.03 \\ 
  6 & W\_HY & -0.00 & 0.03 & 0.21 & -0.06 \\ 
  7 & JP\_HY & 0.00 & 0.05 & 0.28 & -0.10 \\ 
  8 & UK\_HY\_B & -0.00 & 0.04 & 0.24 & -0.13 \\ 
  9 & JP\_DG & -0.00 & 0.04 & 0.31 & -0.13 \\ 
  10 & US\_HY & -0.00 & 0.03 & 0.27 & -0.18 \\ 
  11 & US\_DG & -0.00 & 0.03 & 0.29 & -0.20 \\ 
  12 & UK\_HY & 0.01 & 0.03 & 0.29 & -0.24 \\ 
   \hline
\end{tabular}
\caption{Global Index  Portfolio Performance \label{tab1}} 
\end{table}

We see a clearer picture emerge once we stratify dividend global
portfolio performance according to market volatility and interest rate
cycles. Our market volatility is categorized into two distinct cycles:
the ``High Vol'' and ``Low Vol'', this is conducted by first,
calculating the rolling 12 month standard deviations of volatility
proxies, second taking the 5 and 95 percentile of observations and
pulling corresponding dates. Subsequent to this stratification, we
geometrically chain excess returns in these periods and annualize excess
returns with the appropriate periodicy. Table \ref{tab2} and \ref{tab3}
shows performance in periods of market distress (Hi Vol) or market calm
(Lo vol). Unlike our results from \ref{tab1}, stratification enables
comparison across regions and strategies, however, being cognizant of
duration in market volatility and may affect strategy performance. From
our stratification, 37.5\% of observations yield positive excesss
returns. Of those, HY indexes (which make up a proportion of 67\% of
those with positive returns), we notice that most strategies give the
highest return in periods of market distress versus market calm and most
index in this market distress give returns close to 0\%. Considering Hi
Vol, HY always outperform DG strategies, i.e.~SA\_HY (2.74\%)
outperforms SA\_DG (-4.07\%) , EU\_HY (0.08\%) outperforms EU\_DG
(-4.84\%) and US\_HY (-0.35\%) outperforms US\_DG (-0.68\%). In Lo Vol
periods, we notice greater dispersion in annualized excess returns and
thus no real indication of wheter longer duration of calmness lead to
lower returns, the performance is less concrete in low volatility
periods. From our market cycle stratification HY and DG strategies offer
defensive characteristics in periods of higher volatility however, only
a few of our portfolios give positive excess return in those periods, as
such idiosyncrasies within market determine signal performance.

\begingroup\fontsize{12pt}{13pt}\selectfont
\begin{longtable}{llrr}
  \toprule
Name & Market Period & Months & Annualized Return (\%) \\ 
  \hline 
\endhead 
\hline 
{\footnotesize Continued on next page} 
\endfoot 
\endlastfoot 
 \midrule
UK\_HY\_B & High Vol &  36 & 8.70 \\ 
  EU\_HY & High Vol &  36 & 5.40 \\ 
  EU\_DG & Low Vol Period &  55 & 3.53 \\ 
  EM\_HY & Low Vol Period &  69 & 3.33 \\ 
  SA\_DG & Low Vol Period &  44 & 3.24 \\ 
  SA\_HY & High Vol &  39 & 2.74 \\ 
  JP\_DG & High Vol &  58 & 2.52 \\ 
  JP\_HY & High Vol &  58 & 0.37 \\ 
  EM\_HY & High Vol &  58 & 0.08 \\ 
   \bottomrule
\caption{Over Performance Volatility Stratification\label{tab2}} 
\end{longtable}
\endgroup
\begingroup\fontsize{12pt}{13pt}\selectfont
\begin{longtable}{llrr}
  \toprule
Name & Market Period & Months & Annualized Return (\%) \\ 
  \hline 
\endhead 
\hline 
{\footnotesize Continued on next page} 
\endfoot 
\endlastfoot 
 \midrule
EU\_HY & Low Vol Period &  55 & -0.11 \\ 
  JP\_HY & Low Vol Period &  69 & -0.28 \\ 
  US\_HY & High Vol &  58 & -0.35 \\ 
  US\_DG & High Vol &  58 & -0.68 \\ 
  W\_HY & High Vol &  58 & -0.69 \\ 
  US\_DG & Low Vol Period &  69 & -0.76 \\ 
  W\_HY & Low Vol Period &  69 & -1.22 \\ 
  SA\_HY & Low Vol Period &  44 & -1.99 \\ 
  US\_HY & Low Vol Period &  69 & -2.42 \\ 
  UK\_HY\_B & Low Vol Period &  55 & -3.63 \\ 
  SA\_DG & High Vol &  39 & -4.07 \\ 
  EU\_DG & High Vol &  36 & -4.84 \\ 
  JP\_DG & Low Vol Period &  69 & -6.46 \\ 
  UK\_HY & Low Vol Period &  55 & -7.20 \\ 
  UK\_HY & High Vol &  36 & -24.01 \\ 
   \bottomrule
\caption{Under Performance Volatility Stratification\label{tab3}} 
\end{longtable}
\endgroup

Once we consider stratifying according to interest rate cyles in Table
\ref{tab4} and \ref{tab5}, specifically assessing excess returns in
Hiking, Cutting, and Neutral cycles. Japan stands as an anomaly among
these economies i.e.~it does not have hiking or cutting cycles, its
central bank largely maintained constant rates. Consequently, we assess
its performance exclusively within the confides of a neutral interest
rate cycle. Once stratified, we geometrically chain, quarterly excess
returns before we annualized to compare between indices. We find that
the average and median annualized cumulative excess returns of the US
and EU dividend indexes during cutting cycles indicate that DG
strategies outperform, especially when interest rates decline. Under
performance during hiking cycles is less pronounced in the US when
taking into account annualized excess ruturn between Hiking and Cutting
cyclees. In the EU, HY and growth both outperform during hiking cycles
(dividend growth outperforms to a larger extent). To aid this analysis,
in Appendix 2 we show results of a principal component analysis on
returns of each dividend portfolio and benchmark. After finding the
principal components, we regress the first 3 principal components to
returns to formalize our assessment on the return drivers for our
portfolios. Dividend portfolios either HY or DG have similar loading
their relative benchmarks, albeit slightly larger. As there are
practical considerations in constructing indexes and coupled with
results from our models, we opine that dividend portfolio may proxy the
market index and thus give investors exposure similar to the
hypothetical index. \begingroup\fontsize{12pt}{13pt}\selectfont

\begin{longtable}{llrr}
  \toprule
Name & Market Period & Quarters & Annualized Return (\%) \\ 
  \hline 
\endhead 
\hline 
{\footnotesize Continued on next page} 
\endfoot 
\endlastfoot 
 \midrule
US\_DG & Cut &  15 & 13.40 \\ 
  EU\_DG & Cut &  14 & 6.15 \\ 
  EU\_DG & Neutral &  29 & 3.37 \\ 
  US\_DG & Hiking &  36 & 3.19 \\ 
  EM\_HY & Hiking &  36 & 2.81 \\ 
  JP\_HY & Neutral &  49 & 1.88 \\ 
  SA\_HY & Hiking &  39 & 1.74 \\ 
  JP\_DG & Neutral &  49 & 1.29 \\ 
  EU\_DG & Hiking &  27 & 1.12 \\ 
  SA\_DG & Hiking &  39 & 1.10 \\ 
  US\_HY & Cut &  15 & 0.05 \\ 
   \bottomrule
\caption{Over Performance in Interest Rate Regimes\label{tab4}} 
\end{longtable}
\endgroup
\begingroup\fontsize{12pt}{13pt}\selectfont
\begin{longtable}{llrr}
  \toprule
Name & Market Period & Quarters & Annualized Return (\%) \\ 
  \hline 
\endhead 
\hline 
{\footnotesize Continued on next page} 
\endfoot 
\endlastfoot 
 \midrule
EU\_HY & Cut &  14 & -0.87 \\ 
  EU\_HY & Neutral &  29 & -1.07 \\ 
  EU\_HY & Hiking &  27 & -2.01 \\ 
  US\_HY & Hiking &  36 & -2.51 \\ 
  EM\_HY & Cut &  15 & -2.73 \\ 
  UK\_HY\_B & Neutral &  22 & -2.91 \\ 
  US\_DG & Neutral &  20 & -3.32 \\ 
  SA\_DG & Cut &  27 & -6.77 \\ 
  SA\_HY & Cut &  27 & -7.48 \\ 
  EM\_HY & Neutral &  20 & -7.88 \\ 
  US\_HY & Neutral &  20 & -8.83 \\ 
  UK\_HY\_B & Cut &  19 & -13.11 \\ 
  UK\_HY & Neutral &  22 & -14.72 \\ 
  UK\_HY & Cut &  19 & -25.36 \\ 
  UK\_HY\_B & Hiking &  30 & -27.06 \\ 
  UK\_HY & Hiking &  30 & -34.58 \\ 
   \bottomrule
\caption{Under Performance in Interest Rate Regimes\label{tab5}} 
\end{longtable}
\endgroup

Finally we consider a dynamic measure we evaluate risk adjusted
performance dividend portfolios achieving their returns overtime. Figure
\ref{fig1} illustrates the consistency in the performance of dividend
portfolios by employing a rolling 60 month information ratio to avoid
looking at short term events that may skew performance results. This
ratio is computed by determining the rolling excess return of the index
relative to its benchmark and then dividing this by the volatility of
those excess returns. The red line represents out performance to the
benchmark, whilst considering risk, thus a yardstick to consider
satisfactory performance. UK\_HY has delivered undesirable consistency
in returns over the sample period, whilst The EM and Japan dividend
portfolios had a polarizing performances throughout the sample period.
For one, from 2005 to 2015, returns for the portfolio were consistently
positive. Since then, over the last 8 years information ratios for these
portfolios have been negative. This contrasts South African portfolios,
of which dividend growth portfolios have since 2010 to 2020 have shown
positive information ratios. The SA\_HY only turned positive since 2017.
US and EU indexes have mirroring in intra region performance
i.e.~deliver ratios that close to 0 and dont deviate much from that mark
over time. Therefore our analysis based on information ratios shows the
same as our whole sample analysis on total return in \ref{tab1},
dividend strategies fail to out perform their benchmark, barring SA\_HY
(of which its information ratio has been in a steady decline). This is
opposite to our results from stratification, where we see out
performance of dividend strategies in low interest rate environment or
periods of high volatility.

\begin{figure}[H]

\includegraphics{Much_Ado_About_Dividends_files/figure-latex/unnamed-chunk-1-1} \hfill{}

\caption{Rolling 3 Year Returns \label{fig1}}\label{fig:unnamed-chunk-1}
\end{figure}

\hypertarget{application-to-south-africa}{%
\section*{Application to South
Africa}\label{application-to-south-africa}}
\addcontentsline{toc}{section}{Application to South Africa}

\hypertarget{backtest-results-from-dividend-portfolio-signals}{%
\subsection*{Backtest Results from Dividend Portfolio
Signals}\label{backtest-results-from-dividend-portfolio-signals}}
\addcontentsline{toc}{subsection}{Backtest Results from Dividend
Portfolio Signals}

Figure \ref{fig3} illustrates the cumulative returns of our dividend
portfolios, accompanied by a display of the total capital invested
during the sample period. The portfolio categories include those
structured around Dividend Yield (DY), Dividend Growth (DG), Price
Momentum, and Sustainability. These portfolios are compared to the
performance benchmark represented by the SWIX Top 40 index.Consistent
with our previous analysis of the SA\_HY and SA\_DG portfolios, a
discernible pattern emerges, showing that the returns over the sample
period fall short of the benchmark set by the market index.
Additionally, our vanilla portfolio, the Dividend High Yield (HY),
exhibits the lowest cumulative returns. Similarly, the Price Momentum
and Sustainability portfolios demonstrate diminished performance when
compared to both the Value and DG portfolios. This observation
underscores a trend of under performance in our portfolios when
evaluated against the broader market index.

This result calls for further examination and investigation to discern
the underlying factors and potential implications within the context of
dividend-oriented investment strategies.

\begin{figure}[H]

\includegraphics{Much_Ado_About_Dividends_files/figure-latex/Figure3-1} \hfill{}

\caption{Rolling 3 Year Returns \label{fig3}}\label{fig:Figure3}
\end{figure}

Tables \ref{tab6} and \ref{tab7} provide a breakdown of total
investments during periods characterized by market cycles and interest
rate regimes. These periods are categorized as either high or low
volatility and pertain to hiking or cutting cycles in response to
realized market volatility and the interest rate regime, our
stratifcation dates are etracted from those used in our analysis of
\ref{tab2} and \ref{tab3} for the SA\_HY and SA\_DG indexes. Similar to
that analysis, in our findings from internationally traded portfolios,
we once again observe the advantageous qualities of dividend portfolios
when stratified according to interest rate regimes and volatility
levels. Notably, the dividend portfolios exhibit notable defensive
attributes, with the price momentum-adjusted and sustainability
portfolios yielding the highest returns during hiking periods and times
of high volatility. Moreover, it is worth highlighting that our
portfolios consistently outperform the market index during these
periods.

\begin{table}[H]
\centering
\begin{tabular}{rlrrrl}
  \hline
 & Portfolio & ROI & Ann ROI \% & SD \% & MarketCycle \\ 
  \hline
1 & BM & 1.06 & -0.87 & 0.02 & High Volatility  \\ 
  2 & Divi1 & 1.02 & -0.91 & 0.03 & High Volatility  \\ 
  3 & Divi2 & 1.04 & -0.89 & 0.03 & High Volatility  \\ 
  4 & Divi3 & 1.00 & -0.92 & 0.01 & High Volatility  \\ 
  5 & Divi4 & 1.06 & -0.88 & 0.04 & High Volatility  \\ 
  6 & BM & 1.83 & 2.09 & 0.08 & Low Volatility  \\ 
  7 & Divi1 & 1.13 & -0.82 & 0.06 & Low Volatility  \\ 
  8 & Divi2 & 1.26 & -0.66 & 0.05 & Low Volatility  \\ 
  9 & Divi3 & 1.45 & -0.23 & 0.08 & Low Volatility  \\ 
  10 & Divi4 & 1.66 & 0.73 & 0.09 & Low Volatility  \\ 
   \hline
\end{tabular}
\caption{Market Cycle Perforomance \label{tab6} } 
\end{table}

Moerver, during cutting cycles or in times of low volatility, we observe
the relationship is maintained as our market cycle analsysis on South
African portfolios. In these scenarios, much like the overall return on
investment, the market index appears to offer higher returns. Notably,
among our portfolios, our HY portfolio that fails to deliver substantial
returns compared to our other dividend-oriented strategies. Given this,
we conclude that that dividend portfolios can be used as a tool for
investors to add return in period of heightened market volatility.

\begin{table}[H]
\centering
\begin{tabular}{rlrrrl}
  \hline
 & Portfolio & ROI & Ann ROI \% & SD \% & MarketCycle \\ 
  \hline
1 & BM & 0.16 & -1.00 & 0.01 & Hiking \\ 
  2 & Divi1 & 0.67 & -0.99 & 0.07 & Hiking \\ 
  3 & Divi2 & 0.41 & -1.00 & 0.06 & Hiking \\ 
  4 & Divi3 & 0.32 & -1.00 & 0.02 & Hiking \\ 
  5 & Divi4 & 0.20 & -1.00 & 0.03 & Hiking \\ 
  6 & BM & 3.41 & 6.15 & 0.26 & Cutting \\ 
  7 & Divi1 & 1.22 & -0.88 & 0.12 & Cutting \\ 
  8 & Divi2 & 1.48 & -0.74 & 0.15 & Cutting \\ 
  9 & Divi3 & 1.95 & -0.24 & 0.14 & Cutting \\ 
  10 & Divi4 & 2.51 & 1.07 & 0.23 & Cutting \\ 
   \hline
\end{tabular}
\caption{Interest Rate Regime Performance \label{tab7}} 
\end{table}

Finally, to assess when our dividend portfolios provide value, we
evaluate their relative performance by taking the product excess returns
and excess weights to give a relative performance measure. To do this,
we simply calculated the difference in monthly excess returns from the
last re balancing date in our back-test, which was October of each year,
spanning from 2007-10-30 to 2022-10-30\footnote{In our back-test
  results, on the date 2008-10-30, there were no weights available for
  any of the portfolios, so we excluded these dates from our analysis.
  As these were point measures and had no significant impact on
  long-term trends, our overall analysis remains unaffected.} While we
do not consider sector returns and weights for portfolios and
benchmarks, we modify the measure by collapsing the excess returns and
excess weights to obtain return attribution for the overall strategy.
Consequently, we can now assess nuances in performance that reward
overweight/underweight decisions when it was opportune\footnote{For a
  more in-depth background and the merits of our using this return
  attribution, see Brinson \& Fachler
  (\protect\hyperlink{ref-brinson1985measuring}{1985}).}

Figure \ref{fig4} illustrates our measure of relative return
performance, considering the excess weights throughout our investment
horizon.

\begin{figure}[H]

\includegraphics{Much_Ado_About_Dividends_files/figure-latex/unnamed-chunk-5-1} \hfill{}

\caption{Rolling 3 Year Returns \label{fig4}}\label{fig:unnamed-chunk-5}
\end{figure}

In contrast to the results in Figure \ref{fig3}, where all dividend
signals resulted in lower ROI over our back-test period, here our
dividend portfolio outperformed the Capped SWIX Top 50 at different
rates, measured by Hit Rates (HR). Specifically, Divi1 had a HR of 50\%,
Divi2 at 42.9\%, Divi3 at 68.8\%, and Divi4 at 31.2\% of the time.
Notably, our value portfolio (Divi 3) shows the highest HR, followed by
Divi 1, Divi 2, and Divi 4, in that order. While the result from our ROI
analysis is consistent with the first portfolio, the order of
performance is different for portfolios Divi 4, Divi 1, and Divi 2. From
this analysis, we conclude that using a proxy for value rather than a
dividend signal adds more value to portfolio performance.

\newpage

\hypertarget{conclusion}{%
\section*{Conclusion}\label{conclusion}}
\addcontentsline{toc}{section}{Conclusion}

Over time, dividend portfolios, whether HY or DG, have exhibited
positive excess returns as indicated by excess cumulative returns. While
the UK\_HY index has shown the highest cumulative return, this trend is
not consistently observed across other regional indexes. However, upon
stratifying these portfolios according to different periods of market
volatility, it becomes evident that during high volatility periods,
dividend strategies offer capital protection when addeed to a portfolio
of assets.Moreover, in this case, HY strategies outperform DG
strategies. Surprisingly, portfolios based in South Africa (SA) tend to
perform well during these high volatility periods, which is somewhat
unconventional as such times are typically associated with a flight to
safety, and Emerging Markets (EM) and, by extension, South Africa, are
considered riskier. When extending our analysis to encompass interest
rate cycles, we observe a contrasting effect compared to the
volatility-based stratification. We find that all strategies appear to
give the highest return in low interest rate cycles.

Dividend strategies offer poor deliver poor return consistency, as
measured by our information ratio. Initially, we discern that, at a
broad level, dividend portfolios do not consistently maintain a positive
ratio over an extended investment horizon. However, disparities in
performance emerge. Notably, South African (SA) and dividend indexes
have consistently delivered positive ratios over the past decade. In
contrast, Emerging Markets (EM) and Japanese (JP) indexes have
experienced substantial declines in their information ratios, despite
seemingly consistent performance prior to 2015. Meanwhile, the United
States (US), European Union (EU), and United Kingdom (UK) indexes have
exhibited unpredictable performance over the sampled period.When we
integrate our information ratio findings with drawdown analysis, we
observe that advanced economies have experienced the fewest draw downs
over the sample period, with the exception of the UK. This could suggest
a relatively lower level of systematic risk in these economies.
Conversely, South African (SA) and Emerging Market (EM) drawdowns have
lower draawdowns, possibly indicating a reduced systematic risk in
emerging markets.

In the context of South Africa and considering the top 50 companies by
market capitalization, we note similar performance to our international
analysis. Firstly, dividend portfolios don't offer convincing total
returns, as measured by our cumulative returns calculations. However,
their value emerges from high volatility and hiking cycles. Moreover, we
observe that once we consider more traditional proxies for value, such
as the Price to Earnings (P/E) ratio, the value of dividend signals
diminishes. In other words, the P/E portfolios perform highly according
to our backtest and performance criteria, notably outperforming the
market index 68.8\% of the time. To conclude, dividend portfolios are a
poor proxy for value and investors are better placed to look into
investment products to give an income component to portfolio total
returns. However, we note that investor preferences vary thus may propel
demand for specific strategies for dividend portfolios. Given our
evidence, investors under constraints investment policy statements
should opt for equity portfolios constructed DG strategies, despite
their lower hit rate, they possess lower volatility in acheiving
returns, this will be the most profitable practical way of attaining
returns close to the market index which is uninvestable.

\newpage

\hypertarget{appendix}{%
\section*{Appendix}\label{appendix}}
\addcontentsline{toc}{section}{Appendix}

\begingroup\fontsize{8pt}{9pt}\selectfont
\begin{longtable}{llll}
  \toprule
TICKER & NAME & Codename & Inception Dates \\ 
  \hline 
\endhead 
\hline 
{\footnotesize Continued on next page} 
\endfoot 
\endlastfoot 
 \midrule
FUDP & FTSE UK Dividend+ Index & UK\_HY &  \\ 
  M2EFDY & MSCI EM HY Gross Total Return USD Index & EM\_HY &  \\ 
  M2GBDY & MSCI UK HY Gross Total Return USD Index & UK\_HY &  \\ 
  M2JPDY & MSCI Japan HY Gross Total Return USD & JP\_HY &  \\ 
  M2USADVD & MSCI USA HY Gross Total Return USD Index & US\_HY &  \\ 
  M2WDHDVD & MSCI World HY Gross Total Return Total Return USD Index & W\_HY &  \\ 
  SPDAEET & S\&P EU 350 Dividends Aristocrats Total Return Index & EU\_DG &  \\ 
  SPJXDAJT & S\&P/JPX Dividend Aristocrats Total Return Index & JP\_DG &  \\ 
  SPDAUDT & S\&P 500 Dividend Aristocrats Total Return Index & US\_DG &  \\ 
  SPSADAZT & S\&P South Africa Dividend Aristocrats Index ZAR Gross TR & SA\_DG &  \\ 
  TJDIVD & FTSE/JSE Dividend+ Index Total Return Index & SA\_HY &  \\ 
  M2EUGDY & MSCI Europe Ex UK HYGross Total Return USD Index & EU\_HY &  \\ 
  TUKXG & FTSE 100 Total Return Index GBP & UK &  \\ 
  GDUEEGF & MSCI Daily TR Gross EM USD & EM &  \\ 
  GDDUUK & MSCI UK Gross Total Return USD Index & UK\_B &  \\ 
  TPXDDVD & Topix Total Return Index JPY & JP &  \\ 
  GDDUUS & MSCI Daily TR Gross USA USD & US &  \\ 
  GDDUWI & MSCI Daily TR Gross World USD & W &  \\ 
  SPTR350E & S\&P Europe 350 Gross Total Return Index & EU\_2 &  \\ 
  SPXT & S\&P 500 Total Return Index & JP &  \\ 
  SPXT & S\&P 500 Total Return Index & US\_2 &  \\ 
  JALSH & FTSE/JSE Africa All Share Index & SA &  \\ 
  JALSH & FTSE/JSE Africa All Share Index & SA &  \\ 
  GDDUE15X & MSCI Daily TR Gross Europe Ex UK USD & EU &  \\ 
   \bottomrule
\caption{Index Description \label{tabdes}} 
\end{longtable}
\endgroup

\hypertarget{dividend-defintions}{%
\section*{Dividend Defintions}\label{dividend-defintions}}
\addcontentsline{toc}{section}{Dividend Defintions}

Bloomberg has two main categories for distributions: Cash Dividends and
Stock Dividends. Various kinds of distributions appear under these
definitions that do not necessarily only apply to ordinary issued shares
(the only security type that we consider in our study). In the next two
subsections we define the types of distributions that fall under these
categories and in some cases provide additional information. Our sample
only comprises of final, interim and regular cash dividends. These
dividends are categorized by Bloomberg as Normal Cash.

\hypertarget{cash-dividends}{%
\subsubsection*{Cash Dividends}\label{cash-dividends}}
\addcontentsline{toc}{subsubsection}{Cash Dividends}

\begin{itemize}
\tightlist
\item
  Final: dividend declared for the financial year-end
\item
  Interim (includes 2nd interim, 3rd interim and 4th interim): dividend
  paid after a reporting period (eg. quarterly or semi-annually) Special
  Cash: dividend declared for the financial year-end or interim period
  over and above the normal dividend
\item
  Regular Cash: a dividend distribution made in cash
\item
  Omitted: A company has elected to skip a scheduled payment
\item
  Discontinued: The discontinuance of dividend payments on an ongoing
  basis
\item
  Interest on Capital: interest paid on fixed income instruments
\item
  Income: mutual fund dividends, in most cases
\item
  Liquidation: a distribution of a companies assets to shareholders
  during (interim) or after delisting (final)
\item
  Return of Capital: a non-taxable cash payment to investors from the
  company that represents a return on invested capital as opposed to a
  dividend
\item
  Memorial: a special dividend. For example a company celebrating an
  anniversary might pay a memorial dividend
\item
  Proceeds from sale of shares: a distribution of cash to shareholders
  after selling shares. For example this may occur when the company
  sells the shares of a shareholder who was not eligible to receive
  shares in an offering and then distributes the proceeds to
  shareholders
\item
  Cancelled: the cancellation of a previously declared dividend
\item
  Return Premium: special cash dividend paid from a special reserve
\item
  Preferred Rights Redemption: a company pays a dividend in exchange for
  previously issued preferred rights
\end{itemize}

\hypertarget{stock-dividends}{%
\section*{Stock Dividends}\label{stock-dividends}}
\addcontentsline{toc}{section}{Stock Dividends}

Bonus: also known as a scrip or capitalization issue. Shareholders are
given additional stock in proportion to their holdings - Scrip: a free
issue or bonus of shares - Stock Dividend: portion of a company's
retained earnings that are distributed to shareholders in stock. The JSE
treats stock dividends as a capitalization issue \newpage

\hypertarget{appendix-2}{%
\section*{Appendix 2}\label{appendix-2}}
\addcontentsline{toc}{section}{Appendix 2}

\hypertarget{principal-component-analysis-results}{%
\section*{Principal Component Analysis
Results}\label{principal-component-analysis-results}}
\addcontentsline{toc}{section}{Principal Component Analysis Results}

\begin{table}[H]
\centering
\begin{tabular}{rlll}
  \hline
 & (Intercept) & EM & EM\_HY \\ 
  \hline
1 &  & -1.60278736871812e-05 & 5.72316592929999e-05 \\ 
  2 & lag(ret) & (8.81517172979451e-05) & (8.4009832916283e-05) \\ 
  3 &  & 0.0626773846003959 *** & 0.068477837097736 *** \\ 
  4 & PC1 & (0.00768871355176577) & (0.00773785727574978) \\ 
  5 &  & -0.211323715240927 *** & -0.197839406763045 *** \\ 
  6 & PC2 & (0.0021054107455268) & (0.00200626418696844) \\ 
  7 &  & -0.181093632569239 *** & -0.174627648441878 *** \\ 
  8 & PC3 & (0.00455691863989387) & (0.0043357228611195) \\ 
  9 &  & 0.0553668947582391 *** & 0.0249326112521369 *** \\ 
  10 & N & (0.00540584158153239) & (0.00515349303427833) \\ 
  11 & R2 & 5217 & 5217 \\ 
  12 & *** p $<$ 0.001;  ** p $<$ 0.01;  * p $<$ 0.05. & 0.71193328402896 & 0.707540316745546 \\ 
  13 &  & *** p $<$ 0.001;  ** p $<$ 0.01;  * p $<$ 0.05. & *** p $<$ 0.001;  ** p $<$ 0.01;  * p $<$ 0.05. \\ 
   \hline
\end{tabular}
\caption{PCA Results} 
\end{table}
\begin{table}[H]
\centering
\begin{tabular}{rlllll}
  \hline
 & (Intercept) & EU & EU\_2 & EU\_DG & EU\_HY \\ 
  \hline
1 &  & -7.59999898186529e-06 & -1.10071320078484e-05 & 9.00587178365088e-05 & -2.67809526620728e-05 \\ 
  2 & lag(ret) & (7.10642224371418e-05) &  & (7.86321555212887e-05) & (8.04092632534596e-05) \\ 
  3 &  & -0.0226465097709638 *** & -0.0128979223892329 * & 0.00792276011442966 & -0.00153111734124876 \\ 
  4 & PC1 & (0.00576197722007948) & (0.00575507231249308) & (0.00795541237359177) & (0.00611274416639685) \\ 
  5 &  & -0.280883723797031 *** & -0.237445919816283 *** & -0.203588370755585 *** & -0.29397548972484 *** \\ 
  6 & PC2 & (0.00169729922318746) & (0.00139733652974771) & (0.00187659773140612) & (0.00191952935477372) \\ 
  7 &  & -0.031933386457366 *** & -0.0030388217366709 & -0.00621726079098141 & -0.0319940536542409 *** \\ 
  8 & PC3 & (0.00381344544966598) & (0.00319068365639232) & (0.00420207609930851) & (0.0043152899262049) \\ 
  9 &  & -0.155678383289745 *** & -0.135775502214704 *** & -0.127585411182103 *** & -0.17922756824018 *** \\ 
  10 & N & (0.00445119616658202) & (0.00368269427686028) & (0.00490212472352653) & (0.00503160857397482) \\ 
  11 & R2 & 5217 & 5217 & 5217 & 5217 \\ 
  12 & *** p $<$ 0.001;  ** p $<$ 0.01;  * p $<$ 0.05. & 0.855570865468774 & 0.861482054224444 & 0.719860264447883 & 0.836998608479176 \\ 
   \hline
\end{tabular}
\caption{PCA Results} 
\end{table}
\begin{table}[H]
\centering
\begin{tabular}{rllll}
  \hline
 & (Intercept) & JP & JP\_DG & JP\_HY \\ 
  \hline
1 &  & -9.99214323094622e-05 & -3.00703367334358e-05 & 3.4782188148871e-05 \\ 
  2 & lag(ret) & (7.46579373997006e-05) & (8.66511282175394e-05) & (8.20393212732995e-05) \\ 
  3 &  & 0.00566154299912426 & -0.00937744419485014 & -0.019606973971878 ** \\ 
  4 & PC1 & (0.00584090791543381) & (0.00725886667366239) & (0.00673256816192562) \\ 
  5 &  & -0.121814953493272 *** & -0.100873310563148 *** & -0.0845457504304278 *** \\ 
  6 & PC2 & (0.00178023967587767) & (0.00206606448195023) & (0.00195600102306667) \\ 
  7 &  & -0.402988732366009 *** & -0.348154775899471 *** & -0.397228501728333 *** \\ 
  8 & PC3 & (0.00376080427125469) & (0.00436402895533763) & (0.00413848635768162) \\ 
  9 &  & 0.389512643216594 *** & 0.362500842309265 *** & 0.36799907150433 *** \\ 
  10 & N & (0.00455716388404002) & (0.00528940026452518) & (0.0050263894091161) \\ 
  11 & R2 & 5217 & 5217 & 5217 \\ 
  12 & *** p $<$ 0.001;  ** p $<$ 0.01;  * p $<$ 0.05. & 0.822256241789988 & 0.725506416469889 & 0.766563221216072 \\ 
  13 &  & *** p $<$ 0.001;  ** p $<$ 0.01;  * p $<$ 0.05. & *** p $<$ 0.001;  ** p $<$ 0.01;  * p $<$ 0.05. & *** p $<$ 0.001;  ** p $<$ 0.01;  * p $<$ 0.05. \\ 
   \hline
\end{tabular}
\caption{PCA Results} 
\end{table}
\begin{table}[H]
\centering
\begin{tabular}{rllll}
  \hline
 & ...1 & SA & SA\_DG & SA\_HY \\ 
  \hline
1 & (Intercept) & 0.000174424066652349 & 0.000154599642594047 & 0.000331610039039925 \\ 
  2 &  & (0.000104831488132788) & (0.000135743539551013) & (0.000191341400418785) \\ 
  3 & lag(ret) & -0.0260201275509458 ** & 0.0162962308506602 & -0.214644994699929 *** \\ 
  4 &  & (0.00910355525354117) & (0.0116751097704669) & (0.0115045665456028) \\ 
  5 & PC1 & -0.192079580415969 *** & -0.125588385439684 *** & -0.155483613609965 *** \\ 
  6 &  & (0.00249964852578935) & (0.00323707057649618) & (0.00456228099800359) \\ 
  7 & PC2 & -0.134145033571322 *** & -0.113928595170158 *** & -0.22468771098418 *** \\ 
  8 &  & (0.00540516980619619) & (0.00688136494957077) & (0.00966127340855607) \\ 
  9 & PC3 & -0.127959733778058 *** & -0.12232707950307 *** & -0.154078714389083 *** \\ 
  10 &  & (0.00645043177866384) & (0.00830039210400402) & (0.0117282146734433) \\ 
  11 & N & 5217 & 5217 & 5217 \\ 
  12 & R2 & 0.595503774888565 & 0.302127435161338 & 0.31984720998824 \\ 
  13 & *** p $<$ 0.001;  ** p $<$ 0.01;  * p $<$ 0.05. & *** p $<$ 0.001;  ** p $<$ 0.01;  * p $<$ 0.05. & *** p $<$ 0.001;  ** p $<$ 0.01;  * p $<$ 0.05. & *** p $<$ 0.001;  ** p $<$ 0.01;  * p $<$ 0.05. \\ 
   \hline
\end{tabular}
\caption{PCA Results} 
\end{table}
\begin{table}[H]
\centering
\begin{tabular}{rlllll}
  \hline
 & ...1 & UK & UK\_B & UK\_HY & UK\_HY\_B \\ 
  \hline
1 & (Intercept) & -2.892336667118e-06 & -7.57345221694089e-05 & -0.000239298424245262 ** & -0.000146004759178592 \\ 
  2 &  & (6.50211737869564e-05) & (5.67565475934448e-05) & (8.96977996357748e-05) & (8.97879506987205e-05) \\ 
  3 & lag(ret) & -0.0387511686606318 *** & -0.0376835890925414 *** & 0.0286863872874779 *** & 0.00564174751399983 \\ 
  4 &  & (0.00650337136955424) & (0.0046125612803052) & (0.00812413977212375) & (0.00663098904930069) \\ 
  5 & PC1 & -0.221863534152518 *** & -0.280858203670326 *** & -0.218803222819842 *** & -0.283326297276248 *** \\ 
  6 &  & (0.00155225612418535) & (0.00135498861352277) & (0.00214205751004051) & (0.00214280428175452) \\ 
  7 & PC2 & -0.0178361496500972 *** & -0.0603959906146276 *** & -0.0227752473153885 *** & -0.0570357629364837 *** \\ 
  8 &  & (0.0035135719566689) & (0.00304188777775765) & (0.0047794082980455) & (0.00477795458383635) \\ 
  9 & PC3 & -0.141868473212785 *** & -0.191398139763492 *** & -0.166685531629944 *** & -0.232784790812376 *** \\ 
  10 &  & (0.00407108588530721) & (0.00354224579865907) & (0.00557254776270046) & (0.00559733103421363) \\ 
  11 & N & 5217 & 5217 & 5217 & 5217 \\ 
  12 & R2 & 0.818124055502306 & 0.906484923580086 & 0.704214574847334 & 0.803364808928872 \\ 
  13 & *** p $<$ 0.001;  ** p $<$ 0.01;  * p $<$ 0.05. & *** p $<$ 0.001;  ** p $<$ 0.01;  * p $<$ 0.05. & *** p $<$ 0.001;  ** p $<$ 0.01;  * p $<$ 0.05. & *** p $<$ 0.001;  ** p $<$ 0.01;  * p $<$ 0.05. & *** p $<$ 0.001;  ** p $<$ 0.01;  * p $<$ 0.05. \\ 
   \hline
\end{tabular}
\caption{PCA Results} 
\end{table}
\begin{table}[H]
\centering
\begin{tabular}{rlllll}
  \hline
 & ...1 & US & US\_2 & US\_DG & US\_HY \\ 
  \hline
1 & (Intercept) & -1.15532052845367e-05 & 1.63629844176586e-05 & 0.00014024400880847 ** & -2.7561557884578e-06 \\ 
  2 &  & (3.35413352219375e-05) & (4.60750384698967e-05) & (5.40237942081449e-05) & (3.83581975256876e-05) \\ 
  3 & lag(ret) & -0.0294796579219486 *** & -0.020354188614454 *** & 0.00320093608628991 & -0.0183892435379719 *** \\ 
  4 &  & (0.00341321696819235) & (0.00452661143914563) & (0.00552324285891578) & (0.00416906056214098) \\ 
  5 & PC1 & -0.233864956202208 *** & -0.230159358796096 *** & -0.208766048087452 *** & -0.208761657418179 *** \\ 
  6 &  & (0.000819555292176499) & (0.0011230023500934) & (0.00131460718338295) & (0.00093403325645807) \\ 
  7 & PC2 & 0.308957501335324 *** & 0.31675590488915 *** & 0.309811010196237 *** & 0.295054406083461 *** \\ 
  8 &  & (0.00199246158915227) & (0.00268173299058554) & (0.00310184861640173) & (0.0022461655392403) \\ 
  9 & PC3 & 0.267585573231036 *** & 0.267259539144946 *** & 0.249094505719275 *** & 0.24580475325279 *** \\ 
  10 &  & (0.00204680429842257) & (0.00281157396528542) & (0.00329509317963199) & (0.00234084919103889) \\ 
  11 & N & 5217 & 5217 & 5217 & 5217 \\ 
  12 & R2 & 0.958636359037242 & 0.923969196587223 & 0.883575207473768 & 0.936197300328372 \\ 
  13 & *** p $<$ 0.001;  ** p $<$ 0.01;  * p $<$ 0.05. & *** p $<$ 0.001;  ** p $<$ 0.01;  * p $<$ 0.05. & *** p $<$ 0.001;  ** p $<$ 0.01;  * p $<$ 0.05. & *** p $<$ 0.001;  ** p $<$ 0.01;  * p $<$ 0.05. & *** p $<$ 0.001;  ** p $<$ 0.01;  * p $<$ 0.05. \\ 
   \hline
\end{tabular}
\caption{PCA Results} 
\end{table}
\begin{table}[H]
\centering
\begin{tabular}{rlll}
  \hline
 & (Intercept) & W & W\_HY \\ 
  \hline
1 &  & -2.68877305546502e-05 & -1.60412046201327e-05 \\ 
  2 & lag(ret) & (2.66934004733464e-05) & (3.62741169219996e-05) \\ 
  3 &  & 0.00173290808467087 & 0.012868806430677 ** \\ 
  4 & PC1 & (0.00305791906183213) & (0.00413386998598336) \\ 
  5 &  & -0.232673492817017 *** & -0.22906874305533 *** \\ 
  6 & PC2 & (0.000644347752549236) & (0.00087055856947328) \\ 
  7 &  & 0.120226620730606 *** & 0.0858355323895155 *** \\ 
  8 & PC3 & (0.0015371543580596) & (0.0020483834598137) \\ 
  9 &  & 0.149969169492583 *** & 0.0454159637409487 *** \\ 
  10 & N & (0.00163434044095793) & (0.00223590470954708) \\ 
  11 & R2 & 5217 & 5217 \\ 
  12 & *** p $<$ 0.001;  ** p $<$ 0.01;  * p $<$ 0.05. & 0.964070088675355 & 0.931627271013059 \\ 
   \hline
\end{tabular}
\caption{PCA Results} 
\end{table}
\newpage

\hypertarget{references}{%
\section*{References}\label{references}}
\addcontentsline{toc}{section}{References}

\hypertarget{refs}{}
\begin{CSLReferences}{1}{0}
\leavevmode\vadjust pre{\hypertarget{ref-al2018revisiting}{}}%
Al-Najjar, B. \& Kilincarslan, E. 2018. Revisiting firm-specific
determinants of dividend policy: Evidence from turkey. \emph{Economic
issues}. 23(1):3--34.

\leavevmode\vadjust pre{\hypertarget{ref-baker1999corporate}{}}%
Baker, H.K. \& Powell, G.E. 1999. How corporate managers view dividend
policy. \emph{Quarterly Journal of Business and Economics}. 17--35.

\leavevmode\vadjust pre{\hypertarget{ref-basu1977investment}{}}%
Basu, S. 1977. Investment performance of common stocks in relation to
their price-earnings ratios: A test of the efficient market hypothesis.
\emph{The journal of Finance}. 32(3):663--682.

\leavevmode\vadjust pre{\hypertarget{ref-black1996dividend}{}}%
Black, F. 1996. The dividend puzzle. \emph{Journal of Portfolio
Management}. 8.

\leavevmode\vadjust pre{\hypertarget{ref-brinson1985measuring}{}}%
Brinson, G.P. \& Fachler, N. 1985. Measuring non-US. Equity portfolio
performance. \emph{The Journal of Portfolio Management}. 11(3):73--76.

\leavevmode\vadjust pre{\hypertarget{ref-brzeszczynski2007dividend}{}}%
Brzeszczyński, J. \& Gajdka, J. 2007. Dividend-driven trading
strategies: Evidence from the warsaw stock exchange. \emph{International
Advances in Economic Research}. 13:285--300.

\leavevmode\vadjust pre{\hypertarget{ref-campbell2017financial}{}}%
Campbell, J.Y. 2017. \emph{Financial decisions and markets: A course in
asset pricing}. Princeton University Press.

\leavevmode\vadjust pre{\hypertarget{ref-chen1998risk}{}}%
Chen, N. \& Zhang, F. 1998. Risk and return of value stocks. \emph{The
Journal of Business}. 71(4):501--535.

\leavevmode\vadjust pre{\hypertarget{ref-damodaran2004investment}{}}%
Damodaran, A. 2004. \emph{Investment fables: Exposing the myths of"
can't miss" investment strategies}. FT Press.

\leavevmode\vadjust pre{\hypertarget{ref-fakir2013dividend}{}}%
Fakir, R. et al. 2013. Dividend yield as a superior investment strategy.
PhD thesis. University of Pretoria.

\leavevmode\vadjust pre{\hypertarget{ref-filbeck1997}{}}%
Filbeck, G. \& Visscher, S. 1997. Dividend yield strategies in the
british stock market. \emph{The European Journal of Finance}.
3(4):277--289.

\leavevmode\vadjust pre{\hypertarget{ref-filbeck2017dividend}{}}%
Filbeck, G., Holzhauer, H.M. \& Zhao, X. 2017. Dividend-yield
strategies: A new breed of dogs. \emph{The Journal of Investing}.
26(2):26--47.

\leavevmode\vadjust pre{\hypertarget{ref-gardner2002motley}{}}%
Gardner, D., Gardner, T. \& Maranjian, S. 2002. \emph{The motley fool
investment guide for teens: 8 steps to having more money than your
parents ever dreamed of}. Vol. 10. Simon; Schuster.

\leavevmode\vadjust pre{\hypertarget{ref-gordon1962}{}}%
Gordon, M.J. 1962. The savings investment and valuation of a
corporation. \emph{The Review of Economics and Statistics}. 37--51.

\leavevmode\vadjust pre{\hypertarget{ref-lemmon2015dividend}{}}%
Lemmon, M.L. \& Nguyen, T. 2015. Dividend yields and stock returns in
hong kong. \emph{Managerial Finance}. 41(2):164--181.

\leavevmode\vadjust pre{\hypertarget{ref-miller}{}}%
Miller, M.H. \& Modigliani, F. 1961. Dividend policy, growth, and the
valuation of shares. \emph{The Journal of Business}. 34(4):411--433.

\leavevmode\vadjust pre{\hypertarget{ref-miller1985dividend}{}}%
Miller, M.H. \& Rock, K. 1985. Dividend policy under asymmetric
information. \emph{The Journal of finance}. 40(4):1031--1051.

\leavevmode\vadjust pre{\hypertarget{ref-o1991beating}{}}%
O'higgins, M. \& Downes, J. 1991. \emph{(No Title)}.

\leavevmode\vadjust pre{\hypertarget{ref-suwanna2012impacts}{}}%
Suwanna, T. 2012. Impacts of dividend announcement on stock return.
\emph{Procedia-Social and Behavioral Sciences}. 40:721--725.

\leavevmode\vadjust pre{\hypertarget{ref-van2013advanced}{}}%
Van Deventer, D.R., Imai, K. \& Mesler, M. 2013. \emph{Advanced
financial risk management: Tools and techniques for integrated credit
risk and interest rate risk management}. John Wiley \& Sons.

\leavevmode\vadjust pre{\hypertarget{ref-visscher2003dividend}{}}%
Visscher, S. \& Filbeck, G. 2003. Dividend-yield strategies in the
canadian stock market. \emph{Financial Analysts Journal}. 59(1):99--106.

\leavevmode\vadjust pre{\hypertarget{ref-wang2011dogs}{}}%
Wang, C., Larsen, J.E., Ainina, M.F., Akhbari, M.L. \& Gressis, N. 2011.
The dogs of the dow in china. \emph{International Journal of Business
and Social Science}. 2(18).

\end{CSLReferences}

\bibliography{Tex/ref}





\end{document}
