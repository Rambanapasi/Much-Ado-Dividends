\documentclass[11pt,preprint, authoryear]{elsarticle}

\usepackage{lmodern}
%%%% My spacing
\usepackage{setspace}
\setstretch{1}
\DeclareMathSizes{12}{14}{10}{10}

% Wrap around which gives all figures included the [H] command, or places it "here". This can be tedious to code in Rmarkdown.
\usepackage{float}
\let\origfigure\figure
\let\endorigfigure\endfigure
\renewenvironment{figure}[1][2] {
    \expandafter\origfigure\expandafter[H]
} {
    \endorigfigure
}

\let\origtable\table
\let\endorigtable\endtable
\renewenvironment{table}[1][2] {
    \expandafter\origtable\expandafter[H]
} {
    \endorigtable
}


\usepackage{ifxetex,ifluatex}
\usepackage{fixltx2e} % provides \textsubscript
\ifnum 0\ifxetex 1\fi\ifluatex 1\fi=0 % if pdftex
  \usepackage[T1]{fontenc}
  \usepackage[utf8]{inputenc}
\else % if luatex or xelatex
  \ifxetex
    \usepackage{mathspec}
    \usepackage{xltxtra,xunicode}
  \else
    \usepackage{fontspec}
  \fi
  \defaultfontfeatures{Mapping=tex-text,Scale=MatchLowercase}
  \newcommand{\euro}{€}
\fi

\usepackage{amssymb, amsmath, amsthm, amsfonts}

\def\bibsection{\section*{References}} %%% Make "References" appear before bibliography


\usepackage[round]{natbib}

\usepackage{longtable}
\usepackage[margin=2.3cm,bottom=2cm,top=2.5cm, includefoot]{geometry}
\usepackage{fancyhdr}
\usepackage[bottom, hang, flushmargin]{footmisc}
\usepackage{graphicx}
\numberwithin{equation}{section}
\numberwithin{figure}{section}
\numberwithin{table}{section}
\setlength{\parindent}{0cm}
\setlength{\parskip}{1.3ex plus 0.5ex minus 0.3ex}
\usepackage{textcomp}
\renewcommand{\headrulewidth}{0.2pt}
\renewcommand{\footrulewidth}{0.3pt}

\usepackage{array}
\newcolumntype{x}[1]{>{\centering\arraybackslash\hspace{0pt}}p{#1}}

%%%%  Remove the "preprint submitted to" part. Don't worry about this either, it just looks better without it:
\makeatletter
\def\ps@pprintTitle{%
  \let\@oddhead\@empty
  \let\@evenhead\@empty
  \let\@oddfoot\@empty
  \let\@evenfoot\@oddfoot
}
\makeatother

 \def\tightlist{} % This allows for subbullets!

\usepackage{hyperref}
\hypersetup{breaklinks=true,
            bookmarks=true,
            colorlinks=true,
            citecolor=blue,
            urlcolor=blue,
            linkcolor=blue,
            pdfborder={0 0 0}}


% The following packages allow huxtable to work:
\usepackage{siunitx}
\usepackage{multirow}
\usepackage{hhline}
\usepackage{calc}
\usepackage{tabularx}
\usepackage{booktabs}
\usepackage{caption}


\newenvironment{columns}[1][]{}{}

\newenvironment{column}[1]{\begin{minipage}{#1}\ignorespaces}{%
\end{minipage}
\ifhmode\unskip\fi
\aftergroup\useignorespacesandallpars}

\def\useignorespacesandallpars#1\ignorespaces\fi{%
#1\fi\ignorespacesandallpars}

\makeatletter
\def\ignorespacesandallpars{%
  \@ifnextchar\par
    {\expandafter\ignorespacesandallpars\@gobble}%
    {}%
}
\makeatother

\newenvironment{CSLReferences}[2]{%
}

\urlstyle{same}  % don't use monospace font for urls
\setlength{\parindent}{0pt}
\setlength{\parskip}{6pt plus 2pt minus 1pt}
\setlength{\emergencystretch}{3em}  % prevent overfull lines
\setcounter{secnumdepth}{5}

%%% Use protect on footnotes to avoid problems with footnotes in titles
\let\rmarkdownfootnote\footnote%
\def\footnote{\protect\rmarkdownfootnote}
\IfFileExists{upquote.sty}{\usepackage{upquote}}{}

%%% Include extra packages specified by user
\usepackage{booktabs}
\usepackage{longtable}
\usepackage{array}
\usepackage{multirow}
\usepackage{wrapfig}
\usepackage{float}
\usepackage{colortbl}
\usepackage{pdflscape}
\usepackage{tabu}
\usepackage{threeparttable}
\usepackage{threeparttablex}
\usepackage[normalem]{ulem}
\usepackage{makecell}
\usepackage{xcolor}

%%% Hard setting column skips for reports - this ensures greater consistency and control over the length settings in the document.
%% page layout
%% paragraphs
\setlength{\baselineskip}{12pt plus 0pt minus 0pt}
\setlength{\parskip}{12pt plus 0pt minus 0pt}
\setlength{\parindent}{0pt plus 0pt minus 0pt}
%% floats
\setlength{\floatsep}{12pt plus 0 pt minus 0pt}
\setlength{\textfloatsep}{20pt plus 0pt minus 0pt}
\setlength{\intextsep}{14pt plus 0pt minus 0pt}
\setlength{\dbltextfloatsep}{20pt plus 0pt minus 0pt}
\setlength{\dblfloatsep}{14pt plus 0pt minus 0pt}
%% maths
\setlength{\abovedisplayskip}{12pt plus 0pt minus 0pt}
\setlength{\belowdisplayskip}{12pt plus 0pt minus 0pt}
%% lists
\setlength{\topsep}{10pt plus 0pt minus 0pt}
\setlength{\partopsep}{3pt plus 0pt minus 0pt}
\setlength{\itemsep}{5pt plus 0pt minus 0pt}
\setlength{\labelsep}{8mm plus 0mm minus 0mm}
\setlength{\parsep}{\the\parskip}
\setlength{\listparindent}{\the\parindent}
%% verbatim
\setlength{\fboxsep}{5pt plus 0pt minus 0pt}



\begin{document}



\begin{frontmatter}  %

\title{A Dive into Dividend Portfolios, When and How to They Work}

% Set to FALSE if wanting to remove title (for submission)




\author[Add1]{Gabriel Rambanapasi}
\ead{gabriel.rams44@gmail.com}





\address[Add1]{Stellenbosch University, Cape Town, South Africa}

\cortext[cor]{Corresponding author: Gabriel Rambanapasi}

\begin{abstract}
\small{
Dividend paying stock offer an additional componenet to otherwise non
dividend paying stock. This paper studies the return signalling cue from
dividend portfolio. We find that dividend portfolios around the around
offer downside protection. However emerging market portfolios have
positive return during market turmoil which is considerably above
returns from advanced economy portfolios.
}
\end{abstract}

\vspace{1cm}





\vspace{0.5cm}

\end{frontmatter}

\setcounter{footnote}{0}



%________________________
% Header and Footers
%%%%%%%%%%%%%%%%%%%%%%%%%%%%%%%%%
\pagestyle{fancy}
\chead{}
\rhead{}
\lfoot{}
\rfoot{\footnotesize Page \thepage}
\lhead{}
%\rfoot{\footnotesize Page \thepage } % "e.g. Page 2"
\cfoot{}

%\setlength\headheight{30pt}
%%%%%%%%%%%%%%%%%%%%%%%%%%%%%%%%%
%________________________

\headsep 35pt % So that header does not go over title




\hypertarget{introduction}{%
\section{Introduction}\label{introduction}}

\hypertarget{problem-statement}{%
\section{Problem Statement}\label{problem-statement}}

Dividend yield is a poor proxy for stock returns, as it doesn't
distinguish itself from price effects of a stock. Unfortunately,
practice in industry when constructing dividend portfolios fails to
recognize this flaw thus leading to sub optimal allocation within
portfolio.

\hypertarget{research-aim}{%
\section{Research Aim}\label{research-aim}}

We propose to enhance the dividend signalling by considering price
changes and dividend payment sustainability offered by stock. Our
solution corrects for this oversight by using price momentum filters and
adjusting for unsustainable payout ratios in portfolio construction to
offer superior risk adjusted return overtime. We will consider multiple
back testing samples and highlight periods in which strategies offer
diversification benefits to portfolio construction.

\hypertarget{literature-review}{%
\section{Literature Review}\label{literature-review}}

\hypertarget{what-are-dividends}{%
\subsection{What are dividends}\label{what-are-dividends}}

Dividends constitute a form of capital distribution by corporations
towards shareholders. They exist in various forms, such as cash, stock,
liquidating, scrip, or property dividends
(\protect\hyperlink{ref-baker2009understanding}{\textbf{baker2009understanding?}}),
of which cash dividends and share repurchases being the most commonly
used in practice. Within cash dividends, regular dividends are widely
used by corporations and payment frequency across jurisdictions. The
decision to issue dividends is typically made by the board of directors,
and approved by shareholders, however practiced more in Europe and less
so in the United States. The payout policy policy of a corporation,
which are guiding principles for management and board of directors
towards capital distributions considers company investment and is
closely watched by investors and analysts. As such, management strives
to grow or maintain a certain level of dividend payouts as this signals
firm growth and investors share of profitability in the company. Various
liertature has covered the effect of dividend announcements before and
after ex -dividend dates. Figure 1 shows a clear and direct relation
with a decrease in share value to the proportionate to the dividend
announcement.

Given the apparent decrease in shareholder value, the logical question
has encouraged a long running debate on dividend relevance and
irrelevance. In 1961, Miller \& Rock
(\protect\hyperlink{ref-miller1985dividend}{1985}) opined that dividends
are irrelevant (MM theory), he argued that shareholders are indifferent
to dividend payments, thus implying that there is no optimal dividend
policy and that all dividend policies are equally good and payments of
dividends could easily be reinvested in shares and make no difference to
share holder wealth. However, the MM theorem fails to consider
real-world market imperfections that may give relevance to dividend
payments. The bird in the arguments opposes the MM theory, suggesting
that investor would prefer to receive less risky cash flow in the form
of dividends instead of potential capital gains at some point in the
future (\protect\hyperlink{ref-gordon1962}{Gordon, 1962}). This
permeates to the cost of equity, since dividends are less risky,
companies that issue more dividends should have higher share prices.
However, propoents of the MM theory contend this suggesting the risk of
future cash flow is affected by the payment of dividend, leading to
negative effects on share prices after the ex-dividend date. The
dividend puzzle considers real world constraints and gives an
interesting take on its relevance and irrelevance, by suggesting that
dividends reduce equity value and make investors worse off; however, are
a reward to investors who bear the risk associated with their
investments as it provides an additional source of return on investment
from a share Black (\protect\hyperlink{ref-black1996dividend}{1996}).
Various literature has made convincing arguments for corporations to pay
dividends which include Tax considerations, dividend signalling and
agency costs in issuing dividends .

Taxe considerations argue in favor for dividend relevance. Across
jurisdiction dividends have different tax treatments to capital gains
and often tax at a higher income tax rate, thus investors that have
higher tax rates choose stocks with lower dividend payouts and
tranversly pushes up the stock price, this is called the clientele
effect
(\protect\hyperlink{ref-baker2009understanding}{\textbf{baker2009understanding?}}).
Proponents of the MM theory suggest that the client effect causes major
substitution effect, suggesting that if companies change their dividend
policy, investors with preferential tax treatment will simply allocate
more capital to that stock and those out of favor will sell their
shares. Given the large number of investors versus listed companies the
process is instantaneously causing a net zero effect on
prices(\protect\hyperlink{ref-baker2009understanding}{\textbf{baker2009understanding?}}).
Second, flotation costs refer to the opportunity costs incurred by a
firm when paying dividends. Through distributing dividends, companies
forego opportunities to expand their operations using retained earnings.
In a world without flotation costs, as suggested by the MM theorem,
management would be indifferent between issuing dividends and borrowing
from the market thus have no effect on shares prices. However, in
reality, external financing comes at a higher cost, leading to
trade-offs in dividend policy decisions and ultimately share prices.

Information asymmetry between shareholders and managers is another
factor that gives relevance to dividend payments. Managers of businesses
have greater knowledge of operations thus value of a business at any
given point more than shareholders. As such, investors rely on dividend
announcements to assess a company's valuation. Dividend signaling
conveys information about the company's quality Al-Najjar \&
Kilincarslan (\protect\hyperlink{ref-al2018revisiting}{2018}) and Baker
\& Powell (\protect\hyperlink{ref-baker1999corporate}{1999}). Investors
compare dividend announcements to historical levels while considering
company fundamentals. However, there is a risk of manipulation by
management, making the dividend signal imperfect for determining share
prices. Principal agency issues may give another reason for issuance of
dividends. The free cash flow hypothesis suggests that dividend payments
force management to raise capital from external sources, which increases
borrowing costs and scrutiny from capital markets. This, in turn,
reduces management's ability to make sub optimal investments and
aligning management and shareholder objectives
(\protect\hyperlink{ref-baker2009understanding}{\textbf{baker2009understanding?}}).
Supporters of this theory ascertain that dividends payments by the
mechanism encourage good business practices.

\hypertarget{empirical-review}{%
\subsection{Empirical review}\label{empirical-review}}

The various methods of capital distributions have varying impact on
financial statements which is summarized in Table of the appendix. From
the perceptive of an investor or analyst the dividend yield metric helps
show the additional return dividends paying securities could add to a
portfolio. Consider that describes the fundamentals that influence the
dividend yield. Assuming a constant payout ratio, dividend yield is a
function of earnings yield. shows the correlation between DY and Price
overtime for various securities. Various studies have identified a
predictive power of dividend yield thus confirm the existence of a value
signal. Also, another signal for dividends is dividend growth per share
for corporations, and unlike the dividend yield, it is not affected by
price but maintain properties that allow for inference into management
quality. As managemnet is aware of the signalling effect of dividends,
this may induce the value trap, that forces management to continually
increase dividends to maintain a certain valuation. However such
companies are more vulnerable to facing financial distress.

Cash dividends, although widely used, are not as tax-efficient as share
buybacks. In this form of capital redistribution, a firm exchanges
assets for outstanding shares, which shrinks the company's assets by the
amount of cash paid out. This action too reduces both its borrowing base
and the shareholders' aggregate equity
(\protect\hyperlink{ref-baker2009understanding}{\textbf{baker2009understanding?}}).
A clear benefit to the company is that it is more flexible when compared
to the rigid dividend payout structures. To most higher net worth
investors, tax benefits in the form of lower capital gains taxes result
in greater preference for share buybacks. Surprisingly, their adoption
has been relatively slow in some emerging economies. According to a
study by Wesson, Muller \& Ward
(\protect\hyperlink{ref-wesson2014market}{2014}), there were only 195
open market share repurchases announced in South Africa from 1999 to
2009. In comparison, Manconi, Peyer \& Vermaelen
(\protect\hyperlink{ref-manconi2014buybacks}{2014}) estimated that share
repurchases constituted approximately 58\% of total announcements in the
United States, 15\% in Canada, and 11\% in Japan over the same period,
indicative of a significant disparity in the adoption of share buybacks
across the world, despite their popularity in the United States.

Dividend payments and growth in dividends per share provides a return
cue and over the years studies on dividend signaling studies can be
categorized into academic and practitioner-oriented studies. Academic
studies, such as Fama \& French
(\protect\hyperlink{ref-fama1988permanent}{1988}), found a positive
correlation between increasing predictive power and longer forecast
horizons. However, subsequent studies like Ang \& Bekaert
(\protect\hyperlink{ref-ang2007stock}{2007}) found no evidence of
long-term predictability in stock returns when considering finite sample
influence. This suggests that dividend yield may not be a reliable
predictor of subsequent returns. One possible reason for this declining
predictive power is the increasing use of share buybacks as an
alternative means for capital distribution, which reduces the
contribution of dividend yield to total return
(\protect\hyperlink{ref-robertson2006}{Robertson \& Wright, 2006}).

On the other hand, practitioner-oriented literature focuses on the
long-term returns of systematic dividend portfolios. One popular
strategy is the ``Dogs of the Dow (DOD),'' which involves constructing a
portfolio of the top 10 highest-paying dividend stocks on the Dow Jones
Industrial Index at the beginning of the year based on the dividends
paid in the previous 12 months, therefore this entail deploying a high
yield strategy (\protect\hyperlink{ref-mcqueen1997does}{McQueen, Shields
\& Thorley, 1997}). Various studies have examined the DOD strategy or
similar high-yield dividend strategies in different time periods and
regions, consistently showing superior risk-adjusted returns compared to
the market index. Examples of such studies include Lemmon \& Nguyen
(\protect\hyperlink{ref-lemmon2015dividend}{2015}) in Hong Kong
Brzeszczyński \& Gajdka
(\protect\hyperlink{ref-brzeszczynski2007dividend}{2007}) in Poland,
Visscher \& Filbeck (\protect\hyperlink{ref-visscher2003dividend}{2003})
in Canada, Filbeck \& Visscher
(\protect\hyperlink{ref-filbeck1997}{1997}) in Britian, and Wang,
Larsen, Ainina, Akhbari \& Gressis
(\protect\hyperlink{ref-wang2011dogs}{2011}) in China. More recently,
Filbeck, Holzhauer \& Zhao
(\protect\hyperlink{ref-filbeck2017dividend}{2017}) investigated the
performance of DOD against a high-yield portfolio of Fortune Most
Desired Companies (MAC) compared to the Dow Jones Industrial Average and
the S\&P 500. The study found significantly higher risk-adjusted returns
for the DOD strategy.

\newpage

\hypertarget{methodology}{%
\section{Methodology}\label{methodology}}

\hypertarget{introduction-1}{%
\subsection{Introduction}\label{introduction-1}}

We employ dividend signals to rank our assets within our selected
universe to construct portfolios that offer higher risk adjusted return
than the market index. We consider a dividend yield ranking selecting
the top 20 stock, dividend growth per share and selecting the top 20
stock, an extension of the dividend portfolios by adding a price
momentum filter and a sustainability portfolio that aims penalizes stock
that have unsustainable dividend practices.

Optimizing an asset portfolio involves carefully calibrating the
trade-offs between risk and expected returns. In achieving the ultimate
goal of the study, we aim to investigate how different risk models can
provide significant cues in forming dividend strategies. To this end, we
employ Minimum Variance, Equal Risk Contribution, and Minimum
Volatility. Additionally, this study incorporates more refinements
models like Risk Efficiency and makes use of proprietary software-based
approaches, specifically drawing upon the Barras risk model.Unique to
the Barras model is the introduction of the Max Utility Operator, which
allows for a more sophisticated interpretation of risk by focusing not
only on the total risk but also the active risk associated with each
asset. This dual perspective enables the construction of a more
versatile covariance matrix, thereby enriching the portfolio
optimization process. The following sections analytically describes the
optimization problem and risk models used in the study.

\hypertarget{security-selection-methodology}{%
\section{Security Selection
Methodology}\label{security-selection-methodology}}

A widely used approach to evaluate dividend signals is to construct
subset portfolios and compare in sample performance. This methodology
does not provide parametric significance test, however, portfolio risk
and return measures are based on systematically constructed portfolios
and serve to provide valuable insights. Various such applications exist
in the literature. Damodaran
(\protect\hyperlink{ref-damodaran2004investment}{2004}) constructs top
decile portfolios based on trailing DY at the beginning of each year
from 1952 to 2001. For the last sample period (1991 -2001), it is found
that the highest dividend yielding portfolio outperformed the lowest by
about 3\%. Conover, Jensen \& Simpson
(\protect\hyperlink{ref-conover2016difference}{2016}) find that
portfolios constructed from high-dividend payers return over 1.5\% more
per year than non-dividend payers, in addition to having lower risk.

Following a similar approach we will rank stock within our selected
universe by dividend signals, namely dividend yield (DY) and dividend
growth per share (DGPS). First, we rebalance at the end of March and
September and construct fully invested, long only portfolios. On each re
balancing date, we take the top 100 stocks by market capitalization
(MC), and then select the top quintile (20 stocks) based on the our
signal scores. We then apply 25 basis trading costs to both buying and
selling of stocks, and we will then use total return values, adjust for
stock splits and other distorting effects on prices to calculate
portfolio returns. We also carefully apply back-dated adjustments to
dividends paid to accurately arrive at on-the-day dividends and actual
closing prices when calculating our Dividend Yield and Dividend Per
Share Growth measures.

We also apply at each re balancing on the risk models mentioned
previously. The optimization are constrained to have minimum and maximum
weight exposure of 0.5 and 1.5 times the equal weighted alternative.
With our quintile portfolios, this implies weights ranging between 2.5\%
and 7.5\%. For the Barra Max utility model with we use a risk aversion
parameter of Common Factor Risk Aversion ratio we'll use will be: 0.0075
\& asset specific R.A ratio of 1

Following this we will construct back-tests on the subset of dividend
signal portfolios.

The Standard portfolios considered as follows

The Standard Dividend Yield Portfolio: uses the 12 month mean trailing
dividend yield measure in its construction. - This avoids biasing to
stocks that experienced recent share price declines (negative momentum),
as would be done when considering on the day DY values; - This will be
treated as the vanilla DY signal portfolio.

The DPSG signal portfolio is constructed by considering the growth of
company dividends mentioned above on a 1, 3 and 5 year basis.

\begin{itemize}
\tightlist
\item
  For the three and five year measure, we only consider stocks that had
  positive share payment growth over the period considered.
\item
  E.g., if a stock had a DPS decrease in year 2, even if it has an
  increased dividend payment over three years
\item
  we set this value to zero.
\item
  This has the effect of rewarding consistency, but also reduces the
  sample set substantially if the period under consideration increases
\end{itemize}

Momentum Adjusted DY and DPSG portfolios extend both our DY and DPSG
portfolios by applying a momentum adjusted filter for each. We use the
following approach to make the adjustments:

\begin{itemize}
\tightlist
\item
  Step 1: Rank our sample (top 100 by MC) by risk-adjusted price
  momentum and consider the top half.
\item
  Note that we do not use the ``traditional'' definition of momentum (12
  - 1 month return as introduced by Jegadeesh and Titman), but rather
  use a risk-adjusted measure for momentum. Here we consider the 90 day
  moving average return series to the same 90 day standard deviation for
  each stock.
\item
  Step 2: Rank our sample by either the DY or DPSG measure, and pick the
  top 20 stocks.
\end{itemize}

Sustainability Adjusted DY and DPSG portfolios extend our DY and DPSG
signals by considering dividend payout ratios (DPR). DPR measures how
much of a company's profit is paid out in dividends. We construct this
signal by removing from the top 100 companies the 20 with the highest
DPR scores. The aim of this filter is simply to avoid the most
unsustainable stocks from a dividend payment perspective - thus
systematically avoiding stocks that are most likely to cut dividends in
the future, leading to a reactionary capital gain loss (as commonly
experienced in practice). - Step 1: Rank our sample (top 100 by MC) by
the payout ratio using normalized earnings and consider only the bottom
80 (lower DPR is more sustainable). This measure is calculated by
considering the fraction (percentage) of net income a firm pays to its
shareholders in dividends, calculated as: Total Common Cash Dividends /
Normalized Earnings. - Step 2: Rank our sample by either the DY or DPSG
measures, and pick the top 20 stocks for each.

For completeness we compare the performance of these constructed
portfolios to a standard value signal (PE) and a momentum signal,
constructed as a composite 60, 120 and 240 day risk-adjusted momentum
score for each stock. We next compare the absolute returns as well as
the risk-adjusted performance and drawdowns of each of these portfolios.
We then consider turnover and tracking error, before briefly showing the
sector exposure of some of the different strategies.

\hypertarget{portfolio-optimization}{%
\section{Portfolio Optimization}\label{portfolio-optimization}}

Portfolio optimization consists of determining a set of assets, and
their respective portfolio participation weights, which satisfy the
investor concerning the combination of risk-return trade-off. Markowitz
(\protect\hyperlink{ref-markowitz1959portfolio}{1959}) proposed the
Mean-Variance (MV) model in which the expected return 4.1 is given by
the a measure of the historical data of the stock's return. For our
study we geometrically chain return to measure true effects on portfolio
returns overtime. The risk is calculated by the variance of these
returns 4.2. The MV model treats returns of individual assets as random
variables and to adopt the value of expected return and variance in
order to quantify the return and investment risk, respectively
(\protect\hyperlink{ref-zhang2018portfolio}{Zhang, Li \& Guo, 2018}).
\begin{align}
\mu \quad = \quad w^TR \\ \notag
\sigma^2 \quad  = \quad w^T\sum \notag
w\end{align} The resulting objective function is to maximize return
given a certain level of risk and constraint: \begin{align}
Maximise \ w^TR \sum w<=\sigma^2\ and\ \sum_{i = 1}^{N}w{_i}=1 \notag
\end{align}

Linear constraints are generally included in MV portfolio optimization.
Optimization typically assume that portfolio weights sum to 1 and are
non negative. This defines a linear equality constraint on the
optimization. Another constraint typically used is no-short-selling
condition is a set of sign constraints or linear inequalities. This
reflects avoidance of unlimited liability investment often required in
institutional contexts.

This study will use a an extension of the MV that uses risk preferences
to determine optimum allocation of assets within a portfolio. Barra
definition of portfolio risk extends that from the Modern Portfolio
theory. It defines risk as the decomposition of the variance of returns.
Using Barra multifactor model, the return (r) of a portfolio can be
decomposed into both a common factor return (Xf) and asset specific
return (u) components as: \begin{align} 
r= X_{F} + u \notag
\end{align}

The multi-factor approach entails the creation of a factor covariance
matrix. This is a short term risk forecast that describes trade off of
each common factor within the model. This approach requires periodic
return calculation to these exposures. The upshot is that the
methodology provides an forecast of each assets specific risk.

The covariance matrix is defined as:

\begin{align}
 XFX^T + D \label{eq1} \notag
\end{align}

where X = n x k matrix of asset exposures to the factors. F = k x k
positive semi-definite factor covariance matrix, and D = n × n positive
semi-definite covariance matrix representing a forecast of asset
specific risk.

Expressing portfolio risk in decomposition allows for portfolio manager
to optimize portfolio from either a total risk perspective or an active
risk perspective. In total risk, portfolio holdings are only considered,
and the benchmark holdings are treated as irrelevant for optimization
purposes. Whereas in active risk, the tracking error in which the
difference between the portfolio holdings and those of the benchmark is
given primary consideration in the optimization problem.

\begin{align}
 Total\ Risk : h^T(\lambda_FXFX^T + \lambda_D)h\\ \quad \quad  \notag 
 Active \ Risk : (h-h_B)^T(\lambda_FXFX^T+\lambda_DD)(h-h_B) \notag
\end{align}

where, = \begin{align}
\lambda_f \quad = \quad common \ factor\ risk\ aversion\ parameter,\\ \lambda_d \quad  = \quad specific\ \notag risk\ aversion\ parameter,\\\ h \quad = \quad n×1 \ vector\ of\ managed\ portfolio’s\ holdings,\\ and \quad \notag\\h_B \quad =\quad vector \ normal\ (benchmark)\ portfolio’s\ holdings \notag 
\end{align}

The introduction of risk aversion parameters into Barra's portfolio
optimization is a form of a max utility operator that allows the
portfolio managers to incorporate a numeric representation of personal
risk preferences into the portfolio optimization process\footnote{see
  \url{https://www.sciencedirect.com/science/article/pii/S1057521921002556}
  for a detailed explanation on advantages of using maximum utility
  operators to efficiently factor investor risk preferences}. It also
provides the opportunity to quantify relative aversion to common factor
risk vis-à-vis specific risk. Consequently, these risk aversion
parameters are important tools that are available to assist the
portfolio manager in the construction of an optimal portfolio that is
consistent with their goals.

\hypertarget{equal-risk-contribution-erc}{%
\subsection{Equal Risk Contribution
(ERC)}\label{equal-risk-contribution-erc}}

Equal risk contribution is a return free approach that seeks to equalize
risk contributions from the universe of selected assets thus ensuring it
is fully diversified from a risk perspective. Let sigma measure
portfolio risk and C(x) defined to be the risk contribution of asset i.
If the portfolio risk is measured as by the variance of its return then
is;

\begin{align}
\sigma^2 = x^TQ{x}\\ and \ C_i(x)= x_i(Qx)_i \\ \notag
\quad where\ (Qx)_i =\sum_{i = 1}^{N}Q_ijxj \notag
\end{align}

\begin{align}
x^{ERC} \ satisfies \ C_i(x^{ERC}) = (R(x^{ERC})/N) \ for \ i = 1,...., N. \notag
\end{align}

From this we conclude that the variance and standard deviation measures
are the same for and when can then only the variance risk measure
appreciating that all results apply equally to standard deviation.

\hypertarget{minimum-volatility-minvol}{%
\subsection{Minimum Volatility
(MinVol)}\label{minimum-volatility-minvol}}

Minimum-variance similar to ERC do not require return forecasts, there
can be in some cases they may be more efficient than strategies that
trade off expected risk and return.

\hypertarget{tax-considerations}{%
\section{Tax considerations}\label{tax-considerations}}

Portfolio theory was developed in a perfect world without friction. In
practice, frictions need to be considered and in portfolio construction
this often entails considering the effect of taxes on income and capital
gains as they can erode returns and significantly alter risks and return
characteristics of shares. The contribution of dividends and capital
gains to total return can lead to varying tax inefficiencies for shares
as most jurisdictions imposed higher taxes than on capital gains.
Therefore shares with higher contribution of dividends will be less tax
efficient than those with a higher capital gains component and with
timing most jurisdictions tax dividends in the year that they are
receive\footnote{See Deloitte's tax guides and country highlights:
  \url{https://dits.deloitte.com/\#TaxGuides}}.

Jurisdictional laws can also affect the distribution of taxable returns
amongst shares depending on their class namely ordinary shares or
preferred shares. Preferred shares are viewed as a substitute for bonds
and income from preferred shares are often given tax at a lower rate
than those from dividends from ordinary shares.

We will not survey global tax regimes or incorporate all potential tax
complexities into the portfolio construction but assume a high level
commonalities exists amongst all jurisdictions this study uses. This is
a reasonable assumption considering the summary of taxes on dividends
and capital gains from major economies. For simplicity, we will assume a
basic tax regime includes the key elements of investment-related taxes
that are representative of what a typical taxable asset owner of a
global portfolio will contend with. The proposed methodology to employ
on the dividend portfolios use the following methodology.

\begin{align}
r_{a t}=p_d r_{p t}\left(1-t_d\right)+p_a r_{p t}\left(1-t_{c g}\right)
\end{align}

where r\_\{at\} the after tax return, p\_d= the proportion of r\_\{pt\}
attributed to dividend income, p\_a= the proportion of r\_\{pt\}
attributed to price appreciation, t\_d= the dividend tax rate and
t\_\{cg\}= the capital gains tax rate

\newpage

\hypertarget{data-source-and-stratification}{%
\section{Data Source and
Stratification}\label{data-source-and-stratification}}

The data used for this research is sourced from Bloomberg with the
sample period from 04/01/01--06/30/23. We collected daily price levels
for various indices and benchmarks, share prices for stock listed on the
Johanesburg Stock Exchange (JSE) to construct our own portfolios. To
capture the dynamic nature of financial markets in our stratification,
we segmented the data using proxies that reflect market cycles and
interest rate regimes. For volatility we used Chicago Board of Options
Exchange (CBOE) VIX Index for the US indexes and emerging market
indexes, V2X for Europe, FTSE IVUK for UK and SATITOP40 for SA
volatility proxies fo and data from the Federal Reserve Bank of the
United States. These metrics served as our stratifying variables,
allowing us to categorize our sample across different market and
interest rate cycles. For interest rate regimes, we use central bank
interest rate schedules from the geographies we study.

To calculate our excess returns, we geometrically chain the excess
returns for the different periods before annualizing. This produces
comparable cumulative annualized excess return (CAER) results, defined
as: \begin{align}
C A E R=\left[\prod_{t=1}^n\left(1+E R_t\right)\right]^{\frac{222}{n}}-1 \notag
\end{align}

Our rule to identifying volatility periods either high volatility
(Hi-vol) or low volatility (Lo-vol) is achieved by computing the top and
bottom quantile in standard deviation for our respective proxies. We
then pull the dates corresponding to the periods, and compute annualized
returns after geometrically chaining the monthly returns. The amount of
daily data for the respective interest rate cycles is large enough to
annualized, however, when the VIX, V2X or JALSH RV breach the top or
bottom quintile for less than 50 trading days, the period is excluded in
order to avoid annualizing small samples.

\hypertarget{when-do-dividend-strategies-work}{%
\section{When Do Dividend Strategies
Work}\label{when-do-dividend-strategies-work}}

The data presented in Table \ref{tab1} delineates the excess cumulative
returns of our globally traded dividend portfolios. On an aggregate
level, these portfolios yield a positive premium in comparison to their
corresponding market indices. Nevertheless, a nuanced examination
reveals a discernible variance in performance between the high yield
(HY) and dividend growth (DG) strategies. From the UK-centric proxies
for dividend strategies, the UK\_HY notably surpasses its counterparts,
delivering a cumulative return of 3.5 times the initial investment over
the sample period. In other regions, high yield strategies have
manifested returns exceeding 1x from the inception of the period. It's
pertinent to underscore, however, that these represent marginal gains
when contextualized within a 16-year investment horizon. In stark
contrast, dividend growth strategies have under performed, yielding
diminished excess returns since the onset of the period.

Upon assessing the cumulative returns, it becomes evident that there is
not a consistent indication that dividend strategies, irrespective of
their specific approach or geographical orientation, can consistently
procure a premium that, over time, translates into substantive value for
investors.

\begin{table}[H]
\centering
\begin{tabular}{rlrrrr}
  \hline
 & Regions & Start Date & Total Years & Median & Cumulative Excess Return \\ 
  \hline
1 & EM\_HY & 1199232000.00 & 16.00 & 0.83 & 0.85 \\ 
  2 & EU\_DG & 1199232000.00 & 16.00 & 0.76 & 0.82 \\ 
  3 & EU\_HY & 1199232000.00 & 16.00 & 1.08 & 1.13 \\ 
  4 & JP\_DG & 1199232000.00 & 16.00 & 0.70 & 0.70 \\ 
  5 & JP\_HY & 1199232000.00 & 16.00 & 0.99 & 1.26 \\ 
  6 & SA\_DG & 1199232000.00 & 16.00 & 0.46 & 0.40 \\ 
  7 & SA\_HY & 1199232000.00 & 16.00 & 0.56 & 0.40 \\ 
  8 & UK\_HY & 1199232000.00 & 16.00 & 1.71 & 3.52 \\ 
  9 & UK\_HY\_B & 1199232000.00 & 16.00 & 1.37 & 1.47 \\ 
  10 & US\_DG & 1199232000.00 & 16.00 & 0.76 & 0.74 \\ 
  11 & US\_HY & 1199232000.00 & 16.00 & 0.90 & 1.04 \\ 
  12 & W\_HY & 1199232000.00 & 16.00 & 1.07 & 1.20 \\ 
   \hline
\end{tabular}
\caption{Cumulative Excess Return \label{tab1}} 
\end{table}

By stratifying these samples according to distinct interest rate regimes
and equity market stability cycles, a more refined understanding emerges
regarding the efficacy of dividend signals. Initially, interest rates
are categorized into two distinct cycles: the ``cutting'' cycle and the
``hiking'' cycle. These cycles are defined by periods wherein sustained
rate changes (a minimum of three alterations) manifest at intervals of
at least every five quarters. Moreover, both implied and realized equity
market volatilities are leveraged to represent various epochs of market
stability. Subsequent to this stratification, we engage in the geometric
chaining of the excess returns across these varied periods, which are
then annualized. The resultant metric provides a comparative framework
for cumulative annualized returns.

Table \ref{tab2} delineates performance across diverse market cycles. An
immediate observation is that, on average, the various geographies
experience more periods characterized by market volatility than they do
periods of low volatility. In these heightened volatility epochs, the
annualized returns for most portfolios, be they HY or DG, tend to
outstrip those seen during low volatility phases. This observation
suggests that dividend portfolios might exhibit inherent defensive
qualities during phases of market instability. It is especially
noteworthy that the UK\_HY portfolio registers the most substantial
returns during both high and low volatility phases.

\begingroup\fontsize{12pt}{13pt}\selectfont
\begin{longtable}{llrr}
  \toprule
Index & Market Period & Months & annualized\_return \\ 
  \hline 
\endhead 
\hline 
{\footnotesize Continued on next page} 
\endfoot 
\endlastfoot 
 \midrule
SA\_DG & Low Vol Period &  82 & -0.13 \\ 
  SA\_HY & Low Vol Period &  82 & -0.09 \\ 
  SA\_HY & High Vol Period &  63 & -0.09 \\ 
  JP\_HY & Low Vol Period &  80 & -0.08 \\ 
  JP\_DG & Low Vol Period &  80 & -0.07 \\ 
  US\_DG & Low Vol Period &  80 & -0.05 \\ 
  JP\_DG & High Vol Period &  77 & -0.05 \\ 
  SA\_DG & High Vol Period &  63 & -0.05 \\ 
  US\_DG & High Vol Period &  77 & -0.05 \\ 
  EU\_DG & Low Vol Period &  86 & -0.03 \\ 
  EM\_HY & Low Vol Period &  80 & -0.03 \\ 
  JP\_HY & High Vol Period &  77 & -0.03 \\ 
  EU\_DG & High Vol Period &  93 & -0.03 \\ 
  EM\_HY & High Vol Period &  77 & -0.02 \\ 
  US\_HY & Low Vol Period &  80 & -0.01 \\ 
  US\_HY & High Vol Period &  77 & -0.00 \\ 
  EU\_HY & Low Vol Period &  86 & 0.01 \\ 
  EU\_HY & Low Vol Period &  86 & 0.01 \\ 
  EU\_HY & Low Vol Period &  86 & 0.01 \\ 
  EU\_HY & High Vol Period &  93 & 0.02 \\ 
  UK\_HY\_B & High Vol Period &  93 & 0.11 \\ 
  UK\_HY\_B & Low Vol Period &  86 & 0.14 \\ 
  UK\_HY & High Vol Period &  93 & 0.18 \\ 
  UK\_HY & Low Vol Period &  86 & 0.18 \\ 
   \bottomrule
\caption{Performance in Market Cycles\label{tab2}} 
\end{longtable}
\endgroup

Table \ref{tab3} presents the performance metrics of various dividend
portfolios across different interest rate regimes, encompassing Hiking,
Cutting, and Neutral phases. The Federal Reserve Funds Rate serves as a
representative metric for the interest rate regime in emerging markets,
given the recognition that interest rate shifts in the US influence risk
appetites, thus determining capital flows between advanced and emerging
economies. For other indices, the local central bank interest rate
cycles are employed to ascertain their corresponding interest rate
regimes. Japan stands as an anomaly among these economies; absent
distinct hiking or cutting cycles, its central bank largely maintained
constant rates. Consequently, we assess its performance exclusively
within the confines of a neutral interest rate cycle.

The high yield portfolios, with the exception of SA, typically register
superior annualized returns across all interest rate cycles. Several
portfolios, including EU\_DG, JP\_DG, JP\_HY, SA\_DG, and SA\_HY,
consistently report negative annualized returns regardless of the
prevailing interest rate regime. Meanwhile, the UK\_HY\_B index exhibits
a negative return in neutral phases, and the US\_DG portfolio
underperforms during both cutting and hiking phases. The index that
emerges with the highest annualized excess return across these cycles is
UK\_HY. Notably, the UK\_HY\_B portfolio demonstrates commendable excess
returns, realizing a 0.54\% gain during cutting cycles.

\begingroup\fontsize{12pt}{13pt}\selectfont
\begin{longtable}{llrr}
  \toprule
Index & Market Period & Months & annualized\_return \\ 
  \hline 
\endhead 
\hline 
{\footnotesize Continued on next page} 
\endfoot 
\endlastfoot 
 \midrule
SA\_HY & Hiking &  30 & -0.27 \\ 
  SA\_DG & Hiking &  30 & -0.23 \\ 
  EU\_DG & Hiking &  14 & -0.19 \\ 
  US\_DG & Cut &  12 & -0.14 \\ 
  SA\_HY & Cut &  24 & -0.11 \\ 
  JP\_HY & Neutral &  58 & -0.09 \\ 
  SA\_DG & Cut &  24 & -0.08 \\ 
  EU\_DG & Cut &  20 & -0.08 \\ 
  JP\_DG & Neutral &  58 & -0.06 \\ 
  EU\_DG & Neutral &  28 & -0.05 \\ 
  SA\_HY & Neutral &   8 & -0.05 \\ 
  US\_HY & Cut &  12 & -0.03 \\ 
  UK\_HY\_B & Neutral &  28 & -0.03 \\ 
  EU\_HY & Hiking &  14 & -0.03 \\ 
  US\_DG & Hiking &  22 & -0.03 \\ 
  SA\_DG & Neutral &   8 & -0.02 \\ 
  US\_HY & Hiking &  22 & 0.01 \\ 
  EM\_HY & Neutral &  28 & 0.02 \\ 
  EU\_HY & Cut &  20 & 0.02 \\ 
  US\_DG & Neutral &  28 & 0.02 \\ 
  US\_HY & Neutral &  28 & 0.03 \\ 
  EM\_HY & Cut &  12 & 0.04 \\ 
  EU\_HY & Neutral &  28 & 0.05 \\ 
  EM\_HY & Hiking &  22 & 0.07 \\ 
  UK\_HY\_B & Hiking &  18 & 0.11 \\ 
  UK\_HY & Neutral &  28 & 0.22 \\ 
  UK\_HY & Hiking &  18 & 0.44 \\ 
  UK\_HY\_B & Cut &  16 & 0.54 \\ 
  UK\_HY & Cut &  16 & 0.78 \\ 
   \bottomrule
\caption{Performance in Interest Rate Regimes\label{tab3}} 
\end{longtable}
\endgroup

\hypertarget{performance-consistentcy}{%
\subsection{Performance consistentcy}\label{performance-consistentcy}}

Figure \ref{fig1} illustrates the consistency in the performance of
dividend portfolios by employing the rolling information ratio. The
information ratio serves as a measure of a portfolio's performance
relative to a market benchmark. It is frequently used in the industry to
gauge a manager's proficiency in generating excess returns and the
consistency with which these returns are achieved. Thus, our objective
is to assess the capacity of our dividend portfolios to achieve such
excess returns.

We have adopted a rolling 60-month information ratio as a metric to
evaluate long-term performance consistency. This ratio is computed by
determining the rolling excess return of the index relative to its
benchmark and then dividing this by the volatility of those excess
returns.

\begin{figure}[H]

\includegraphics{ThesisWriteUp_files/figure-latex/unnamed-chunk-1-1} \hfill{}

\caption{Rolling 3 Year Returns \label{fig1}}\label{fig:unnamed-chunk-1}
\end{figure}

The findings presented here illuminate the sporadic performance of
dividend strategies, further highlighting the challenges inherent in
securing high returns in asset markets. In an ideal scenario, an
information ratio exceeding 0 is preferred, with most industry
strategies typically averaging around 0.3. However, Figure \ref{fig1}
demonstrates that, when assessed on a rolling 60-month basis, both
growth and high-yield dividend portfolios tend to underperform relative
to their benchmarks.Upon broadening our evaluation to encompass 24 and
36-month periods, the results appear even more volatile. Notably, SA
indices consistently demonstrate commendable performance, with the
SA\_HY standing out particularly. Within the domain of advanced markets,
the dividend portfolios of both the EU and Japan exhibit a modicum of
consistency in positive returns initially, only to later deteriorate.

It is imperative to note that despite their capacity to deliver decent
cumulative returns and exhibit defensive characteristics across
stratified periods, these indices, when gauged on a relative basis,
still leave much to be desired in terms of consistent performance.

\hypertarget{drawdowns}{%
\subsection{Drawdowns}\label{drawdowns}}

A trailing return provides insights into the performance of an
investment over a specified period, measured between two distinct dates.
As the name suggests, it essentially ``trails'' the investment from a
starting point to an end point, effectively capturing the point-to-point
returns. This tool offers a concise snapshot of, for instance, a mutual
fund's performance at a specific juncture in its trajectory.

The utility of the trailing return metric is multi-faceted. For one, it
can gauge returns over varying durations, such as year-to-date, over one
year, three years, and so forth. Furthermore, it is feasible to compute
trailing returns by referencing the current date back to the fund's very
inception. Such an analysis proves invaluable when attempting to discern
how varying inception dates might influence the performance of a fund.

Delving deeper into the realm of risk assessment, drawdowns provide a
comprehensive view of the risk attributes of the data series'
constituents. In the context of our study, the relevance of drawdowns is
underscored by their capacity to unveil latent correlations between
performance and drawdown. Essentially, by analyzing drawdowns, one can
ascertain the maximum potential loss an investment has experienced,
thereby providing insights into the inherent risks and vulnerabilities
associated with said investment.

\begin{figure}[H]

\includegraphics{ThesisWriteUp_files/figure-latex/Figure2-1} \hfill{}

\caption{Rolling 3 Year Returns \label{fig2}}\label{fig:Figure2}
\end{figure}

In this analysis, drawdowns are delineated as the disparity between the
peak and trough values of cumulative excess returns within a specified
time frame. When scrutinizing our dividend portfolios, a pattern of
similarity emerges both geographically and across varied strategies.The
SA portfolios are markedly conspicuous, displaying the most pronounced
drawdowns by value. This is closely followed by the UK High Yield (HY)
strategies. In contrast, portfolios associated with the EU and US
exhibit relatively milder drawdowns.

Yet, when the focus shifts from mere magnitude to the distribution or
dispersion of these drawdowns, the narrative undergoes a transformation.
The UK High Yield and Japan High Yield strategies are revealed to be
more volatile, being susceptible to significant fluctuations. In
juxtaposition, emerging markets, with South Africa as a case in point,
manifest a more stabilized profile, evidenced by diminished variation in
their drawdowns.

\hypertarget{wrapping-it-up}{%
\subsection{Wrapping it up}\label{wrapping-it-up}}

\begin{itemize}
\tightlist
\item
  I did a PCA which tries to link all these together. Partcilarly which
  risk attributes link the performance of dividend portfolios and how
  these can be accounted for by designing a partcilar portfolio.
\end{itemize}

In the extensive evaluation of globally dividend portfolios across
varied market cycles and interest rate regimes, it's evident that while
dividend strategies offer additional return offered by their income
componenet, their performance remains notably inconsistent. The
challenges in obtaining high returns in asset markets come to the fore,
with dividend strategies across various geographies and timelines
exhibiting uneven returns. The impact of interest rate cycles, plays a
decisive role in portfolio performance, a fact made salient by Japan's
unique neutral rate cycle. Adding another layer of depth to our
analysis, rolling informatio ratios offer periodic snapshots of
performance consistency, noting that most strategies fail to give
performance postive ratios but it's the drawdowns, defined by the
discrepancies between peak and trough values of cumulative excess
returns, that reveal the underlying risk attributes. Geographically,
while South African portfolios record pronounced drawdown values,
closely trailed by UK High Yield strategies, portfolios in regions like
the EU and US demonstrate relative stability.

\hypertarget{which-dividend-signals-matter-in-south-africa}{%
\section{Which Dividend Signals Matter in South
Africa?}\label{which-dividend-signals-matter-in-south-africa}}

We assessed dividend portfolio performance in different geographies and
\# Results and Analysis \# Discussion

\hypertarget{limitations-of-the-study}{%
\subsection{Limitations of the study}\label{limitations-of-the-study}}

Could not get constituent data for the indices that had superior
performance within the sample period. This would have been help in
constrcuting more representative indices and unpacking why it was that
dividend yield and or grow signals were more effective in that
portfolio.

\hypertarget{conclusion}{%
\section{Conclusion}\label{conclusion}}

\hypertarget{references}{%
\section{References}\label{references}}

\hypertarget{refs}{}
\begin{CSLReferences}{1}{0}
\leavevmode\vadjust pre{\hypertarget{ref-al2018revisiting}{}}%
Al-Najjar, B. \& Kilincarslan, E. 2018. Revisiting firm-specific
determinants of dividend policy: Evidence from turkey. \emph{Economic
issues}. 23(1):3--34.

\leavevmode\vadjust pre{\hypertarget{ref-ang2007stock}{}}%
Ang, A. \& Bekaert, G. 2007. Stock return predictability: Is it there?
\emph{The Review of Financial Studies}. 20(3):651--707.

\leavevmode\vadjust pre{\hypertarget{ref-baker1999corporate}{}}%
Baker, H.K. \& Powell, G.E. 1999. How corporate managers view dividend
policy. \emph{Quarterly Journal of Business and Economics}. 17--35.

\leavevmode\vadjust pre{\hypertarget{ref-bhattacharyya2007dividend}{}}%
Bhattacharyya, N. 2007. Dividend policy: A review. \emph{Managerial
Finance}. 33(1):4--13.

\leavevmode\vadjust pre{\hypertarget{ref-black1996dividend}{}}%
Black, F. 1996. The dividend puzzle. \emph{Journal of Portfolio
Management}. 8.

\leavevmode\vadjust pre{\hypertarget{ref-brzeszczynski2007dividend}{}}%
Brzeszczyński, J. \& Gajdka, J. 2007. Dividend-driven trading
strategies: Evidence from the warsaw stock exchange. \emph{International
Advances in Economic Research}. 13:285--300.

\leavevmode\vadjust pre{\hypertarget{ref-conover2016difference}{}}%
Conover, C.M., Jensen, G.R. \& Simpson, M.W. 2016. What difference do
dividends make? \emph{Financial Analysts Journal}. 72(6):28--40.

\leavevmode\vadjust pre{\hypertarget{ref-cornell2014dividend}{}}%
Cornell, B. 2014. Dividend-price ratios and stock returns: International
evidence. \emph{Journal of Portfolio management}. 40(2):122.

\leavevmode\vadjust pre{\hypertarget{ref-damodaran2004investment}{}}%
Damodaran, A. 2004. \emph{Investment fables: Exposing the myths of"
can't miss" investment strategies}. FT Press.

\leavevmode\vadjust pre{\hypertarget{ref-deangelo2006irrelevance}{}}%
DeAngelo, H. \& DeAngelo, L. 2006. The irrelevance of the MM dividend
irrelevance theorem. \emph{Journal of financial economics}.
79(2):293--315.

\leavevmode\vadjust pre{\hypertarget{ref-fama1988permanent}{}}%
Fama, E.F. \& French, K.R. 1988. Permanent and temporary components of
stock prices. \emph{Journal of political Economy}. 96(2):246--273.

\leavevmode\vadjust pre{\hypertarget{ref-filbeck1997}{}}%
Filbeck, G. \& Visscher, S. 1997. Dividend yield strategies in the
british stock market. \emph{The European Journal of Finance}.
3(4):277--289.

\leavevmode\vadjust pre{\hypertarget{ref-filbeck2017dividend}{}}%
Filbeck, G., Holzhauer, H.M. \& Zhao, X. 2017. Dividend-yield
strategies: A new breed of dogs. \emph{The Journal of Investing}.
26(2):26--47.

\leavevmode\vadjust pre{\hypertarget{ref-gordon1962}{}}%
Gordon, M.J. 1962. The savings investment and valuation of a
corporation. \emph{The Review of Economics and Statistics}. 37--51.

\leavevmode\vadjust pre{\hypertarget{ref-gordon1963optimal}{}}%
Gordon, M.J. 1963. Optimal investment and financing policy. \emph{The
Journal of finance}. 18(2):264--272.

\leavevmode\vadjust pre{\hypertarget{ref-hussainey2011dividend}{}}%
Hussainey, K., Mgbame, C.O. \& Chijoke-Mgbame, A.M. 2011. Dividend
policy and share price volatility: UK evidence. \emph{The Journal of
risk finance}. 12(1):57--68.

\leavevmode\vadjust pre{\hypertarget{ref-jensen1976theory}{}}%
Jensen, M.C. \& Meckling, W.H. 1976. Theory of the firm: Managerial
behavior, agency costs and ownership structure. \emph{Journal of
financial economics}. 3(4):305--360.

\leavevmode\vadjust pre{\hypertarget{ref-10.2307ux2f3694818}{}}%
Koch, A.S. \& Sun, A.X. 2004. Dividend changes and the persistence of
past earnings changes. \emph{The Journal of Finance}. 59(5):2093--2116.

\leavevmode\vadjust pre{\hypertarget{ref-lemmon2015dividend}{}}%
Lemmon, M.L. \& Nguyen, T. 2015. Dividend yields and stock returns in
hong kong. \emph{Managerial Finance}. 41(2):164--181.

\leavevmode\vadjust pre{\hypertarget{ref-lintner1956distribution}{}}%
Lintner, J. 1956. Distribution of incomes of corporations among
dividends, retained earnings, and taxes. \emph{The American economic
review}. 46(2):97--113.

\leavevmode\vadjust pre{\hypertarget{ref-maio2015dividend}{}}%
Maio, P. \& Santa-Clara, P. 2015. Dividend yields, dividend growth, and
return predictability in the cross section of stocks. \emph{Journal of
Financial and Quantitative Analysis}. 50(1-2):33--60.

\leavevmode\vadjust pre{\hypertarget{ref-manconi2014buybacks}{}}%
Manconi, A., Peyer, U. \& Vermaelen, T. 2014. Buybacks around the world.
\emph{European Corporate Governance Institute (ECGI)-Finance Working
Paper}. 436.

\leavevmode\vadjust pre{\hypertarget{ref-markowitz1959portfolio}{}}%
Markowitz, H.M. 1959. Portfolio selection, 1952{]}: Portfolio selection.
\emph{Journal of Finance}.

\leavevmode\vadjust pre{\hypertarget{ref-masum2014dividend}{}}%
Masum, A. 2014. Dividend policy and its impact on stock price--a study
on commercial banks listed in dhaka stock exchange. \emph{Global
disclosure of Economics and Business}. 3(1).

\leavevmode\vadjust pre{\hypertarget{ref-mcqueen1997does}{}}%
McQueen, G., Shields, K. \& Thorley, S.R. 1997. Does the {``dow-10
investment strategy''} beat the dow statistically and economically?
\emph{Financial Analysts Journal}. 53(4):66--72.

\leavevmode\vadjust pre{\hypertarget{ref-miller1985dividend}{}}%
Miller, M.H. \& Rock, K. 1985. Dividend policy under asymmetric
information. \emph{The Journal of finance}. 40(4):1031--1051.

\leavevmode\vadjust pre{\hypertarget{ref-rangvid2014dividend}{}}%
Rangvid, J., Schmeling, M. \& Schrimpf, A. n.d. Dividend predictability
around the world. \emph{Journal of Financial and Quantitative Analysis}.
49(5-6):1255--1277.

\leavevmode\vadjust pre{\hypertarget{ref-robertson2006}{}}%
Robertson, D. \& Wright, S. 2006. Dividends, total cash flow to
shareholders, and predictive return regressions. \emph{Review of
Economics and Statistics}. 88(1):91--99.

\leavevmode\vadjust pre{\hypertarget{ref-suwanna2012impacts}{}}%
Suwanna, T. 2012. Impacts of dividend announcement on stock return.
\emph{Procedia-Social and Behavioral Sciences}. 40:721--725.

\leavevmode\vadjust pre{\hypertarget{ref-van2013advanced}{}}%
Van Deventer, D.R., Imai, K. \& Mesler, M. 2013. \emph{Advanced
financial risk management: Tools and techniques for integrated credit
risk and interest rate risk management}. John Wiley \& Sons.

\leavevmode\vadjust pre{\hypertarget{ref-vijayakumar2010effect}{}}%
Vijayakumar, A. 2010. Effect of financial performance on share prices in
the indian corporate sector: An empirical study. \emph{Management and
Labour Studies}. 35(3):369--381.

\leavevmode\vadjust pre{\hypertarget{ref-visscher2003dividend}{}}%
Visscher, S. \& Filbeck, G. 2003. Dividend-yield strategies in the
canadian stock market. \emph{Financial Analysts Journal}. 59(1):99--106.

\leavevmode\vadjust pre{\hypertarget{ref-wang2011dogs}{}}%
Wang, C., Larsen, J.E., Ainina, M.F., Akhbari, M.L. \& Gressis, N. 2011.
The dogs of the dow in china. \emph{International Journal of Business
and Social Science}. 2(18).

\leavevmode\vadjust pre{\hypertarget{ref-wesson2014market}{}}%
Wesson, N., Muller, C. \& Ward, M. 2014. Market underreaction to open
market share repurchases on the JSE. \emph{South African Journal of
Business Management}. 45(4):59--69.

\leavevmode\vadjust pre{\hypertarget{ref-zhang2018portfolio}{}}%
Zhang, Y., Li, X. \& Guo, S. 2018. Portfolio selection problems with
markowitz's mean--variance framework: A review of literature.
\emph{Fuzzy Optimization and Decision Making}. 17:125--158.

\end{CSLReferences}

\hypertarget{appendix}{%
\section{Appendix}\label{appendix}}

\bibliography{Tex/ref}





\end{document}
