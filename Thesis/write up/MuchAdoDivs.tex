\documentclass[11pt,preprint, authoryear]{elsarticle}

\usepackage{lmodern}
%%%% My spacing
\usepackage{setspace}
\setstretch{1}
\DeclareMathSizes{12}{14}{10}{10}

% Wrap around which gives all figures included the [H] command, or places it "here". This can be tedious to code in Rmarkdown.
\usepackage{float}
\let\origfigure\figure
\let\endorigfigure\endfigure
\renewenvironment{figure}[1][2] {
    \expandafter\origfigure\expandafter[H]
} {
    \endorigfigure
}

\let\origtable\table
\let\endorigtable\endtable
\renewenvironment{table}[1][2] {
    \expandafter\origtable\expandafter[H]
} {
    \endorigtable
}


\usepackage{ifxetex,ifluatex}
\usepackage{fixltx2e} % provides \textsubscript
\ifnum 0\ifxetex 1\fi\ifluatex 1\fi=0 % if pdftex
  \usepackage[T1]{fontenc}
  \usepackage[utf8]{inputenc}
\else % if luatex or xelatex
  \ifxetex
    \usepackage{mathspec}
    \usepackage{xltxtra,xunicode}
  \else
    \usepackage{fontspec}
  \fi
  \defaultfontfeatures{Mapping=tex-text,Scale=MatchLowercase}
  \newcommand{\euro}{€}
\fi

\usepackage{amssymb, amsmath, amsthm, amsfonts}

\def\bibsection{\section*{References}} %%% Make "References" appear before bibliography


\usepackage[round]{natbib}

\usepackage{longtable}
\usepackage[margin=2.3cm,bottom=2cm,top=2.5cm, includefoot]{geometry}
\usepackage{fancyhdr}
\usepackage[bottom, hang, flushmargin]{footmisc}
\usepackage{graphicx}
\numberwithin{equation}{section}
\numberwithin{figure}{section}
\numberwithin{table}{section}
\setlength{\parindent}{0cm}
\setlength{\parskip}{1.3ex plus 0.5ex minus 0.3ex}
\usepackage{textcomp}
\renewcommand{\headrulewidth}{0.2pt}
\renewcommand{\footrulewidth}{0.3pt}

\usepackage{array}
\newcolumntype{x}[1]{>{\centering\arraybackslash\hspace{0pt}}p{#1}}

%%%%  Remove the "preprint submitted to" part. Don't worry about this either, it just looks better without it:
\makeatletter
\def\ps@pprintTitle{%
  \let\@oddhead\@empty
  \let\@evenhead\@empty
  \let\@oddfoot\@empty
  \let\@evenfoot\@oddfoot
}
\makeatother

 \def\tightlist{} % This allows for subbullets!

\usepackage{hyperref}
\hypersetup{breaklinks=true,
            bookmarks=true,
            colorlinks=true,
            citecolor=blue,
            urlcolor=blue,
            linkcolor=blue,
            pdfborder={0 0 0}}


% The following packages allow huxtable to work:
\usepackage{siunitx}
\usepackage{multirow}
\usepackage{hhline}
\usepackage{calc}
\usepackage{tabularx}
\usepackage{booktabs}
\usepackage{caption}


\newenvironment{columns}[1][]{}{}

\newenvironment{column}[1]{\begin{minipage}{#1}\ignorespaces}{%
\end{minipage}
\ifhmode\unskip\fi
\aftergroup\useignorespacesandallpars}

\def\useignorespacesandallpars#1\ignorespaces\fi{%
#1\fi\ignorespacesandallpars}

\makeatletter
\def\ignorespacesandallpars{%
  \@ifnextchar\par
    {\expandafter\ignorespacesandallpars\@gobble}%
    {}%
}
\makeatother

\newenvironment{CSLReferences}[2]{%
}

\urlstyle{same}  % don't use monospace font for urls
\setlength{\parindent}{0pt}
\setlength{\parskip}{6pt plus 2pt minus 1pt}
\setlength{\emergencystretch}{3em}  % prevent overfull lines
\setcounter{secnumdepth}{5}

%%% Use protect on footnotes to avoid problems with footnotes in titles
\let\rmarkdownfootnote\footnote%
\def\footnote{\protect\rmarkdownfootnote}
\IfFileExists{upquote.sty}{\usepackage{upquote}}{}

%%% Include extra packages specified by user
\usepackage{booktabs}
\usepackage{longtable}
\usepackage{array}
\usepackage{multirow}
\usepackage{wrapfig}
\usepackage{float}
\usepackage{colortbl}
\usepackage{pdflscape}
\usepackage{tabu}
\usepackage{threeparttable}
\usepackage{threeparttablex}
\usepackage[normalem]{ulem}
\usepackage{makecell}
\usepackage{xcolor}

%%% Hard setting column skips for reports - this ensures greater consistency and control over the length settings in the document.
%% page layout
%% paragraphs
\setlength{\baselineskip}{12pt plus 0pt minus 0pt}
\setlength{\parskip}{12pt plus 0pt minus 0pt}
\setlength{\parindent}{0pt plus 0pt minus 0pt}
%% floats
\setlength{\floatsep}{12pt plus 0 pt minus 0pt}
\setlength{\textfloatsep}{20pt plus 0pt minus 0pt}
\setlength{\intextsep}{14pt plus 0pt minus 0pt}
\setlength{\dbltextfloatsep}{20pt plus 0pt minus 0pt}
\setlength{\dblfloatsep}{14pt plus 0pt minus 0pt}
%% maths
\setlength{\abovedisplayskip}{12pt plus 0pt minus 0pt}
\setlength{\belowdisplayskip}{12pt plus 0pt minus 0pt}
%% lists
\setlength{\topsep}{10pt plus 0pt minus 0pt}
\setlength{\partopsep}{3pt plus 0pt minus 0pt}
\setlength{\itemsep}{5pt plus 0pt minus 0pt}
\setlength{\labelsep}{8mm plus 0mm minus 0mm}
\setlength{\parsep}{\the\parskip}
\setlength{\listparindent}{\the\parindent}
%% verbatim
\setlength{\fboxsep}{5pt plus 0pt minus 0pt}



\begin{document}



\begin{frontmatter}  %

\title{Much Ado About Dividends}

% Set to FALSE if wanting to remove title (for submission)




\author[Add1]{Gabriel Rambanapasi}
\ead{gabriel.rams44@gmail.com}





\address[Add1]{Stellenbosch University, Cape Town, South Africa}

\cortext[cor]{Corresponding author: Gabriel Rambanapasi}


\vspace{1cm}





\vspace{0.5cm}

\end{frontmatter}

\setcounter{footnote}{0}



%________________________
% Header and Footers
%%%%%%%%%%%%%%%%%%%%%%%%%%%%%%%%%
\pagestyle{fancy}
\chead{}
\rhead{}
\lfoot{}
\rfoot{\footnotesize Page \thepage}
\lhead{}
%\rfoot{\footnotesize Page \thepage } % "e.g. Page 2"
\cfoot{}

%\setlength\headheight{30pt}
%%%%%%%%%%%%%%%%%%%%%%%%%%%%%%%%%
%________________________

\headsep 35pt % So that header does not go over title




\hypertarget{introduction}{%
\section*{Introduction}\label{introduction}}
\addcontentsline{toc}{section}{Introduction}

\hypertarget{literature-review}{%
\section{Literature Review}\label{literature-review}}

\hypertarget{introduction-1}{%
\subsection{Introduction}\label{introduction-1}}

The debate surrounding the relevance of dividends in firm valuation has
been a longstanding topic in finance literature. Pioneered Miller \&
Modigliani (\protect\hyperlink{ref-miller}{1961}), with the irrelevance
of dividends theory (MM theory) posits that, in a perfect market,
investors should be indifferent to receiving dividends as firm value is
derived from earnings resulting from investment policy. However, critics
have challenged this school of thought, by considering investor
preferences and information asymmetries. Their findings offers insights
into the significance of dividends in conveying information to investors
about company prospects. Empirically,we find studies that conform to
literature pertaining to the bird in the hand theory, signalling theory
and agency cost theory. We find that high dividend yield and dividend
growth ratios could be used to proxy value for our dividend paying
companies, showing predictive power of dividend metrics on stock
returns. By synthesizing and analyzing these studies, we gain a deeper
understanding of the role of dividends in firm valuation and investor
decision-making processes.

\hypertarget{dividend-theories}{%
\subsection{\texorpdfstring{Dividend Theories
\label{theory}}{Dividend Theories }}\label{dividend-theories}}

The debate surrounding the irrelevance of dividends pioneered by Miller
\& Modigliani (\protect\hyperlink{ref-miller}{1961}), gave insight into
firm valuation. He posited that in a perfect market, one that is free
from market frictions and transaction costs, investors should be
indifferent to receiving dividends, as firm value is derived from
earnings which result from investment policy. They reinforced the
dividend irrelevance theorem by arguing that if the dividend practice
adopted by any firm corresponds to the dividend preference of its
shareholders each firm would attract its clientele based on its dividend
policy practice. Moreover given their assumptions, a change in dividend
policy will not materially affect any firm valuation because with the
existence of several competing firms, a firm that changes dividend
policy may not act by the preference of some shareholders to simply
induce a movement across firms as investors try to align with firms
whose dividend practice corresponds with their dividend preference.
Consequently, in the long run, equilibrium in terms of choice of
investment and dividend preference will be attained and shareholders
valuation of the firm will not be different from those of firms with
different dividend policy.

Miller \& Modigliani (\protect\hyperlink{ref-miller}{1961}) shortfalls
emanates from unrealistic assumptions resulting in opposing arguments.
Firstly, the bird in the hand theory, a common argument that is used to
make a case for investor preferences for dividend payment, Walter
(\protect\hyperlink{ref-walter1963dividend}{1963}) suggests that
investors prefer dividends to capital gains because the nature payment
of dividends effectively guarantees income towards investors against
some probability of receiving capital gains at some point during an
investment holding period. This implies that investors prefer to receive
dividends now so that they can reinvest and earn a further return.
Buttressing this, Lintner
(\protect\hyperlink{ref-lintner1956distribution}{1956}) argued that
dividend are desired because it helps to reduce the level of information
asymmetry, a firm that pays dividend assures investors that the firm is
performing well. Moreover, Gordon
(\protect\hyperlink{ref-gordon1963optimal}{1963}) saw dividend as
preferred to capital gains because dividend payment reduces risks
associated with investments because it is more certain. Therefore, the
major implication of the bird in the hand theory sees risk in the
reinvestment of company profits through keeping it as retained earning.
Consequently, investors expect a higher expected return with the payment
of dividend which ultimately raises the costs of capital from investors.

Secondly, Lintner
(\protect\hyperlink{ref-lintner1956distribution}{1956}) suggested that
information asymmetry between shareholders and management, brings into
focus management decisions and their impact on firm prospects. Cognizant
of this, managers of dividend paying companies are more willing to raise
rather than reduce dividend levels, and this is construed to mean that
given dividend payment history decreases are associated with negative
signals while dividend increases signal positive news signal positive
news. More formally, Bhattacharyya
(\protect\hyperlink{ref-bhattacharyya2007dividend}{1979}) presents a
signaling model where the liquidation of the firm is related to the
actual dividend paid and any change in dividend alters the liquidation
value of the firm. The liquidation value represents the amount of money
shareholders would receive if the company were to be dissolved, of which
dividends can affect this value by depleting the firm's cash reserves by
distributing earnings to shareholders. The distribution the firm reduces
its financial cushion during financial hardships. Consequently, dividend
payments can influence investors' perceptions of the firm's risk profile
and future prospects.Changes in dividend under the signalling model
convey important information about a firm's outlook. for the quality of
firms assets.

Lastly, Jensen \& Meckling
(\protect\hyperlink{ref-jensen1976theory}{1976}) argued for agency
costs, managerial behavior and ownership structure in advocating for the
relevance of dividend payments. The separation of control and ownership
gives rise to a principle-agent problem because managers have the
responsibility of acting in the best interest of the owners, however,
there are possibilities for conflicts between the managers and
shareholders. A high level of retained may motivate managers to pursue
decisions that promote their own self interests, therefore shareholders
mimimize the amount of retained earnings on hand available to managers
to mitigate the risk of acting out of self interest. Jensen \& Meckling
(\protect\hyperlink{ref-jensen1976theory}{1976}) proposes a free cash
flow model that states ``when a firm has financed all its positive net
present value investments, it should distribute all its free cash flow
as dividends''. This prescription should reduce agency costs. The
insights gained from these opposing theories to Miller \& Modigliani
(\protect\hyperlink{ref-miller}{1961}) do make thought interesting
arguments to the issuance of dividends by companies.

\hypertarget{empirical-studies-and-their-return-predictive-signal}{%
\subsection{Empirical studies and their return predictive
signal}\label{empirical-studies-and-their-return-predictive-signal}}

Empirical studies we asesss are concerned with finding factors that
influence dividend policy, thus relating them underlying theories in
section \ref{theory}. Baker \& Wurgler
(\protect\hyperlink{ref-baker2006investor}{2006}) show that investor
sentiment is a significant determinant of dividend policy. Their study
suggests that waves of investor sentiment have differential effects on
stock returns, particularly for stock that are difficult to arbitrage
and have valuations that are highly subjective. In reaction, management
may adjust their dividend policy in response to prevailing market
sentiment, potentially paying dividends to signal confidence during
periods of high sentiment and conserving cash during times of market
uncertainty. With regard to category of stock and performance, they also
find that during periods of low sentiment, returns are high for small
size, young firms, unprofitable stock, non dividend paying, extreme
growth and distressed stocks. During periods of high sentiment this
group of stock experiences low returns.

Grullon, Larkin \& Michaely
(\protect\hyperlink{ref-grullon2019dividend}{2019}) analyse managerial
decisions to distribute cash flows in industries with varying levels of
competitiveness. Although this study was conducted in the manufacturing
sector, its insight indicates that firms operating in competitive
industries tend to have lower payout ratios.Thus firms in highly
competitive environments prioritize investment and innovation over
dividend payments, moreover whose overall characteristics suggest that
lower agency costs and less likely to be a target of predation. In other
words, dividends act as a tool to reduce agency costs. Interestingly,
the authors allude to disciplinary forces that follow Jensen \& Meckling
(\protect\hyperlink{ref-jensen1976theory}{1976}) free cash flow theory,
that states, managers in competitive industries payout excess cash and
with the idea that corporate payouts are the ``outcome'' of external
factors. Denis \& Osobov (\protect\hyperlink{ref-denis2008firms}{2008})
provide cross-country evidence on the determinants of dividend policy,
revealing that larger, more profitable firms with higher retained
earnings are more likely to pay dividends across US, Canada, UK,
Germany, France, and Japan. In each country, aggregate dividend payments
concentrated among the largest, most profitable firms. Outside of the US
there is little evidence of a systematic positive relation between
relative prices of dividend paying and non-paying firms and the
propensity to pay dividends. These reconcile with Jensen \& Meckling
(\protect\hyperlink{ref-jensen1976theory}{1976}) \& Lintner
(\protect\hyperlink{ref-lintner1956distribution}{1956}) as these studies
posit that firms with strong financial performance and stability are
more inclined to pay dividends to signal quality and mitigate agency
conflicts.

Baker \& Wurgler (\protect\hyperlink{ref-baker2006investor}{2006}),
Grullon \emph{et al.}
(\protect\hyperlink{ref-grullon2019dividend}{2019}) \& Denis \& Osobov
(\protect\hyperlink{ref-denis2008firms}{2008}) show that dividend paying
companies usually comprise of larger more profitable firms. An intuitive
proxy for why companies with high dividend yields might outperform non
dividend paying companies, is that dividend yield proxies for the value
factor (\protect\hyperlink{ref-basu1977investment}{Basu, 1977}).
Consider the equation below to understand this arguemnet:

\begin{align*}
DY = \frac{EPS}{\text{Price}} \times \text{Payout Ratio}
\end{align*}

Assuming a constant payout ratio, dividend yield (DY) would simply
become a function of changes in earnings yield. From the equation,
holding earnings per share (EPS) constant, price has a inverse relation
to DY. This implies that studies identifying return predictive signal DY
verify the existence of the value-signal \footnote{Value as an
  investment factor dates back to Basu (1977), who used the
  price-earnings ratio of companies to compare stock performance. Since
  then, many studies have confirmed the existence of a value premium,
  where cheap stocks outperform their more expensive counterparts over
  time.}. Notably, Cornell
(\protect\hyperlink{ref-cornell2014dividend}{2014}) studied the
predictive power of dividend-price ratios using US data. Their findings
revealed that higher dividend-price ratios are associated with higher
future stock returns, suggesting stock-return predictability based on
dividend-price ratios. This implies that investors may be able to
leverage dividend-price ratios as an informational tool to make informed
investment decisions and potentially earn above-average returns.
Similarly, Conover, Jensen \& Simpson
(\protect\hyperlink{ref-conover2016difference}{2016}) explored the
investment benefits of dividend-paying stocks by analyzing dividend
yield and its correlation with stock returns. The study uncovered that
high-dividend-paying stocks tend to exhibit lower risk and higher
returns compared to non-dividend-paying stocks. Specifically,
high-dividend payers outperformed non-dividend payers by over 1.5\% per
year on average. Other studies used growth in dividends to capture
second order characteristics of dividend payers, Chen
(\protect\hyperlink{ref-chen2009reversal}{2009}) delved into the
predictive power of dividend growth for future stock returns, focusing
on historical data from the prewar period of the early 1900s. The study
identified strong predictive power in dividend growth during this
historical period, suggesting that changes in dividend growth rates can
serve as valuable signals for predicting future stock returns. However,
the predictive power of dividend growth appeared to diminish in the
postwar years, highlighting the importance of historical context and
market dynamics in assessing the effectiveness of dividend-based
investment strategies.

\newpage

\hypertarget{data-and-methodology}{%
\section*{Data and Methodology}\label{data-and-methodology}}
\addcontentsline{toc}{section}{Data and Methodology}

\hypertarget{data-for-globally-traded-dividend-indexes}{%
\subsection{\texorpdfstring{Data for Globally Traded Dividend
Indexes\label{int}}{Data for Globally Traded Dividend Indexes}}\label{data-for-globally-traded-dividend-indexes}}

We utilize historical price data for globally traded dividend indexes
spanning from January 1, 2003, to January 1, 2023, constituting a
20-year sample period. These indexes are constructed in gross total
return and denominated in the same currency, the full list of names and
tickers are given in \ref{tabdes}. We MSCI ,Standard and Poor's (S\&P)
and FTSE High Yield and Dividend Growth constructed indexes. S\&P
Dividend Aristocrats indexes: these indexes measure the performance of
companies that have a history of consistently increasing dividends on an
equal weighted basis and years of required dividend growth varies across
countries. MSCI High Dividend Yield indexes: indexes are designed to
measure the performance of companies with high dividend yields. The
indexes exclude companies that do not exhibit dividend sustainability,
persistence and quality. Constituents are first screened for
deteriorating fundamentals, thus attempting to explicitly avoid
associated `yield traps'. Constituents are then weighted by size. FTSE
Dividend Plus indexes (UK and SA only): the constituents of the Divi
Plus indexes are chosen based on the one-year dividend per share
forecast. Constituent weights are determined by forward dividend yields.
The UK index comprises of 50 constituents whereas the SA index comprises
of 30 constituents.

Our metric of interest are annualized excess returns of individual
indexes to their respective benchmarks following Bacon
(\protect\hyperlink{ref-bacon2023practical}{2023}). To acheive this, we
scale observations to an annual scale by raising the compound return to
the number of periods in a year, following this we take the root of
total observations: \begin{align*}
\operatorname{Annualized Return} & = \sqrt[n]{\operatorname{prod}\left(1+R_a\right)^{\operatorname{scale}}} - 1
\end{align*} where \(R_a\) is the return on the asset, \(scale\) is the
number of periods in a year, \(n\) is the total number of periods.
Following this, we simply take the difference of annualized return to
get our excess return: \begin{align*}
E R_a = R_{a} - R_{b}
\end{align*} where \(R_{a}\) are annualized return on the asset and
\(R_{b}\) are annualized return of the benchmark. From our excess return
metrics, we calculate second moments and third moments to describe
distributional properties. Moreover we make other transformations to our
international portfolio sample to get a nuanced perspective on
performance during different time periods. For interest rate regime
cycles we use interest rate schedules from central banks of country of
domicile for the index. That is, to proxy periods of high and low
interest rates we stratify our sample interest rate hiking and cutting
cycles where periods of sustained changes rate changes occur at least
every five quarters. Likewise, we use proxies for volatility such as the
VIX index in the United States, V2X in Europe and JALSH RV in South
Africa that consider implied and realized equity market volatilities to
proxy for different periods of market stability. After stratification,
we geometrically chain the excess returns for the different periods
before annualizing. To stratify amount of daily data for the respective
interest rate cycles has to be large enough to annualized, however, if
our proxies for volatility breach the top or bottom quintile for less
than 50 trading days, the period is excluded in order to avoid
annualizing small frequencies.

\hypertarget{local-dividend-portfolio-construction}{%
\subsection{Local Dividend Portfolio
Construction}\label{local-dividend-portfolio-construction}}

We use historical daily price data in table \ref{tabdes} from January 1
2003 to July 7 2023, a roughly 20 year sample period for equity listed
in the Johannesburg Stock Exchange and retrieve dividend yield ratios,
dividend cover ratio, price momentum, price to earnings ratio, dividend
growth per share from Bloomberg to construct dividend portfolios.

We construct 4 factor portfolios namely; Divi1, Divi2, Divi3 and Divi4
using ranked signals on fundamental and statistical factors . Divi1 is
created through the conditional signal that adds 15\% or 35\% of
dividend coverage or price momentum or both if it ranks in the bottom
quantile in its own factor ranking to 2/3 DY(Fwd\_3) and 1/3 DY(Fwd\_9).
Divi2 ranks based on the signal 67.7\% DY(Fwd\_3) and 33.3\% DY(Fwd\_9).
Divi3 ranks based on the signal Price to Earnings Ratio and Divi4 ranks
based on the signal Dividend Per Share Growth\_1Y = 40\% (DPS\_Growth),
DPS\_Growth\_3Y = 30\%, Fwd\_3 = 20\% and Fwd\_9 = 10\%. Ranking score
ranges from 0 to 100. We feed ranked signals in our optimizer, which
does a aplha transformation on the signal. Our optimizer makes
modifications to the mean variance portfolio that considers risk
preferences of investors
(\protect\hyperlink{ref-markowitz1959portfolio}{Markowitz, 1959}). We
define risk of \(n\) assets as \(\sigma^2\) based on individual asset
returns \(R_a\). We decompose returns into common factor \(X_f\) and
specific return \(u\). Our covariance matrix is then defined as: \[
\begin{aligned}
XFX^T + D
\end{aligned}
\] where: \begin{align*}
X & : n \times k \text{ matrix of asset exposures to the factors,} \\ F & : k \times k \text{ positive semi-definite factor covariance matrix, and} \\ D & : n \times n \text{ positive semi-definite covariance matrix representing a forecast of asset-specific risk.}
\end{align*} For optimization purposes, our \(\sigma^2\) is considered
in two forms: total risk, where only portfolio holdings are considered
(benchmark holdings are irrelevant for the optimization process), and
active risk, which takes into account the difference between portfolio
holdings and benchmark holdings defined as: \begin{align*}
\text{Total Risk:} & \quad h^T(\lambda_F X F X^T + \lambda_D D) h \\
\text{Active Risk:} & \quad (h - h_B)^T(\lambda_F X F X^T + \lambda_D D)(h - h_B)
\end{align*} where: \begin{align*}
\lambda_F & : \text{common factor risk aversion parameter,} \\
\lambda_D & : \text{ specific risk aversion parameter,} \\
h & : n \times 1 \text{ vector of managed portfolio's holdings, and} \\
h_B & : n \times 1 \text{ vector of normal (benchmark) portfolio's holdings}
\end{align*}

The following accounts for practical considerations in constructing our
portfolios. We use the Capped SWIX as our benchmark and \(\lambda_F\)
and \(\lambda_D\) are set to 0.0075 and 1, respectively, with active
risk have an upper limit of 5\%. For our box constraints, sector
exposure can deviate from a range of 10\% relative to the Capped SWIX
and individual asset exposure is limited to 15\%. Our portfolio turnover
is limited to 10\% and we exclude the property sector from our
portfolios. \newpage

\hypertarget{results}{%
\section*{Results}\label{results}}
\addcontentsline{toc}{section}{Results}

\hypertarget{globally-traded-portfolios-performance-results}{%
\subsection{Globally traded portfolios performance
results}\label{globally-traded-portfolios-performance-results}}

Figure \ref{fig1} shows the cumulative monthly excess returns of the
selected indexes relative to the universe benchmark from which
constituents are chosen. The indexes start on different dates of
inception, therefore cumulative excess returns for the period are not
comparable across all indexes for the respective plot duration. Note the
blue indexes are constructed using DY ratios, while the red lines
constitute DPSG. The figure below is not entirely suggestive of a clear
and consistent out performance of dividend strategies over the entire
considered period, whether using DY or DPSG, globally. From \ref{fig1}
we notice that two indices being SPDAEET and SPDAUDT over the investment
horizon return multiples in excess of 1 suggesting that some benefit
over holding the parent index.

\begin{figure}[H]

{\centering \includegraphics{MuchAdoDivs_files/figure-latex/unnamed-chunk-1-1} 

}

\caption{Full Sample Cumulative Returns \label{fig1}}\label{fig:unnamed-chunk-1}
\end{figure}

The table below shows greater detail to full sample performance.
Firstly, SPDAEET and SPDAUDT represent the indexes that gave positive
excess returns 1.6\% and 0.78\%, respectively. Coupled to this, their
respective tracking error were the amongst the lowest at 5.4\% and
6.8\%. Amongst the indexes that over the period return cumulative
returns of less than 1, their respective annualized excess return were
negative and tracking error were high, namely for the TJDIVD and
SPSADAZT which are domiciled in SA at 20\% individually. Developed
market indexes over the period had tracking error within a range of
5.6\% to 8.6\%. Interestingly, SPDAJXT had a tracking error of 15\%.
\begingroup\fontsize{12pt}{13pt}\selectfont

\begin{longtable}{lllll}
  \toprule
index & Country & Cumulative Return & Ann. Excess Return & Tracking Error \\ 
  \hline 
\endhead 
\hline 
{\footnotesize Continued on next page} 
\endfoot 
\endlastfoot 
 \midrule
SPDAUDT & UK & 1.3 & 1.6\% & 5.6\% \\ 
  SPDAEET & EU & 1.1 & 0.78\% & 6.8\% \\ 
  M2WDHDVD...12 & UK & 0.9 & -0.39\% & 6.4\% \\ 
  M2GBDY & UK & 0.77 & -1.5\% & 8.6\% \\ 
  SPJXDAJT & JP & 0.69 & -1.2\% & 15\% \\ 
  TJDIVD & SA & 0.54 & -2.4\% & 20\% \\ 
  SPSADAZT & SA & 0.4 & -3.8\% & 20\% \\ 
  FUDP & US & 0.2 & -8.4\% & 7.9\% \\ 
   \bottomrule
\caption{Full Sample Period Absolute and Relative Return of Dividend Indexes} 
\end{longtable}
\endgroup

Once we stratify excess return by time period we see different results.
The table below shows stratified performance according to high market
volatility. TJDIVD, SPSADAZT and SPDAEET exhibited strong excess
positive annualized excess returns at 24.5\%,18.9\% and 14.4\%
respectively. These represent DPGS signal and a DY signal. Their
respective tracking error were the amongst the lowest at 5.4\% and
6.8\%. Other indices with negative annual excess returns were in were
indices domicile in developed markets.\\
\begingroup\fontsize{12pt}{13pt}\selectfont

\begin{longtable}{lllll}
  \toprule
Index & Country & Ann. Excess Return & Tracking Error & Signal \\ 
  \hline 
\endhead 
\hline 
{\footnotesize Continued on next page} 
\endfoot 
\endlastfoot 
 \midrule
SPDAEET & EU &  24.5\% & 2.6\% & DGPS \\ 
  TJDIVD & SA &  18.9\% & 3.3\% & DY \\ 
  SPSADAZT & SA &  14.4\% & 4.9\% & DGPS \\ 
  SPJXDAJT & JP &  -0.7\% & 2.1\% & DY \\ 
  FUDP & US &  -2.8\% & 2.2\% & DY \\ 
  SPDAUDT & UK &  -3.5\% & 1.6\% & DY \\ 
  M2WDHDVD...12 & UK &  -8.5\% & 2.1\% & DY \\ 
  M2GBDY & UK & -12.6\% & 3.4\% & DY \\ 
   \bottomrule
\caption{Dividend Indexes Performance During High Volatility Periods} 
\end{longtable}
\endgroup

A similar result is noticable when we consider low volatility periods.
TJDIVD, SPSADAZT make up the top performing at 27.37\% and 5.84\%,
albeit their difference in performance has greater disparity than in the
high volatility market cycle. These represent DY signal and DGPS signal.
Their respective tracking error were the amongst the lowest at 5.4\% and
6.8\%. Other indices with negative annual excess returns were in were
indices domiciled in developed markets.

\begingroup\fontsize{12pt}{13pt}\selectfont
\begin{longtable}{lllll}
  \toprule
Index & Country & Ann. Excess Return & Tracking Error & Signal \\ 
  \hline 
\endhead 
\hline 
{\footnotesize Continued on next page} 
\endfoot 
\endlastfoot 
 \midrule
TJDIVD & SA &  27.37\% & 4.04\% & DY \\ 
  SPSADAZT & SA &   5.84\% & 3.55\% & DGPS \\ 
  SPDAUDT & UK &   0.08\% & 1.11\% & DY \\ 
  M2GBDY & UK &  -2.11\% & 1.06\% & DY \\ 
  M2WDHDVD...12 & UK &  -4.04\% & 0.64\% & DY \\ 
  FUDP & US & -14.22\% & 1.55\% & DY \\ 
  SPDAEET & EU & -15.73\% & 1.64\% & DGPS \\ 
  SPJXDAJT & JP & -28.43\% & 3.90\% & DY \\ 
   \bottomrule
\caption{Dividend Indexes Performance During Low Volatility Periods} 
\end{longtable}
\endgroup

Next we consider interest rate hiking and cutting regimes with countries
of domicile for the selected indexes. The table below shows stratified
performance according to interest rate cutting. Firstly, here we notice
a tight positive perfromance amongst a larger group of indexes, with a
tight range range from 1.54\% to 5.27\%. Amongst these indexes, there
domiciled in developed markets constructed using DY signals. Amongst the
negative excess return series, the range is similarly tight at -0.61\%
to -3.19\%. Amongst them, there is a mix of DGPS and DY, of either
deveoping markets and emerging market (SA).

\begingroup\fontsize{12pt}{13pt}\selectfont
\begin{longtable}{lllll}
  \toprule
Index & Country & Ann. Excess Return & Tracking Error & Signal \\ 
  \hline 
\endhead 
\hline 
{\footnotesize Continued on next page} 
\endfoot 
\endlastfoot 
 \midrule
SPJXDAJT & JP &  5.27\% & 1.4\% & DY \\ 
  M2WDHDVD...12 & UK &  4.39\% & 1.2\% & DY \\ 
  SPDAUDT & UK &  2.19\% & 2.1\% & DY \\ 
  SPDAEET & EU &  1.54\% & 1.3\% & DGPS \\ 
  M2GBDY & UK & -0.61\% & 2.3\% & DY \\ 
  TJDIVD & SA & -2.40\% & 3.3\% & DY \\ 
  FUDP & US & -2.84\% & 2.2\% & DY \\ 
  SPSADAZT & SA & -3.19\% & 3.5\% & DGPS \\ 
   \bottomrule
\caption{Dividend Indexes Performance During Cutting Interest Rate Periods} 
\end{longtable}
\endgroup

The table below shows stratified performance according to interest rate
hiking cycles. Again, there is a tight positive performance amongst a
larger group of indexes, ranging from 1.03\% to 3.83\%. However, there
is a variation in country of domcile in developed markets and with 2
indexes constructed form DY and 2 indexes constructed from DGPS. Amongst
the negative excess return series, the range is similarly tight at
-0.61\% to -3.19\%. Amongst them, there is a mix of DGPS and DY, of
either deveoping markets and emerging market (SA).
\begingroup\fontsize{12pt}{13pt}\selectfont

\begin{longtable}{lllll}
  \toprule
Index & Country & Ann. Excess Return & Tracking Error & Signal \\ 
  \hline 
\endhead 
\hline 
{\footnotesize Continued on next page} 
\endfoot 
\endlastfoot 
 \midrule
M2WDHDVD...12 & UK &   3.83\% & 1.2\% & DY \\ 
  SPDAUDT & UK &   2.12\% & 1.1\% & DY \\ 
  SPSADAZT & SA &   1.54\% & 4.4\% & DGPS \\ 
  SPDAEET & EU &   1.08\% & 2.5\% & DGPS \\ 
  M2GBDY & UK &  -1.74\% & 1.8\% & DY \\ 
  TJDIVD & SA &  -3.87\% & 3.8\% & DY \\ 
  SPJXDAJT & JP &  -8.43\% & 3.5\% & DY \\ 
  FUDP & US & -15.70\% & 2.4\% & DY \\ 
   \bottomrule
\caption{Dividend Indexes Performance During Hiking Interest Rate Periods} 
\end{longtable}
\endgroup

\hypertarget{locally-constrcucted-portfolio-results}{%
\subsection{Locally Constrcucted portfolio
results}\label{locally-constrcucted-portfolio-results}}

\hypertarget{discussion}{%
\section*{Discussion}\label{discussion}}
\addcontentsline{toc}{section}{Discussion}

\hypertarget{conclusion}{%
\section*{Conclusion}\label{conclusion}}
\addcontentsline{toc}{section}{Conclusion}

\hypertarget{appendix}{%
\section*{Appendix}\label{appendix}}
\addcontentsline{toc}{section}{Appendix}

\begingroup\fontsize{8pt}{9pt}\selectfont
\begin{longtable}{ll}
  \toprule
Ticker & Name \\ 
  \hline 
\endhead 
\hline 
{\footnotesize Continued on next page} 
\endfoot 
\endlastfoot 
 \midrule
FUDP & FTSE UK Dividend+ Index \\ 
  M2EFDY & MSCI EM HY Gross Total Return USD Index \\ 
  M2GBDY & MSCI UK HY Gross Total Return USD Index \\ 
  M2JPDY & MSCI Japan HY Gross Total Return USD \\ 
  M2USADVD & MSCI USA HY Gross Total Return USD Index \\ 
  M2WDHDVD & MSCI World HY Gross Total Return Total Return USD Index \\ 
  SPDAEET & S\&P EU 350 Dividends Aristocrats Total Return Index \\ 
  SPJXDAJT & S\&P/JPX Dividend Aristocrats Total Return Index \\ 
  SPDAUDT & S\&P 500 Dividend Aristocrats Total Return Index \\ 
  SPSADAZT & S\&P South Africa Dividend Aristocrats Index ZAR Gross TR \\ 
  TJDIVD & FTSE/JSE Dividend+ Index Total Return Index \\ 
  M2EUGDY & MSCI Europe Ex UK HYGross Total Return USD Index \\ 
  TUKXG & FTSE 100 Total Return Index GBP \\ 
  GDUEEGF & MSCI Daily TR Gross EM USD \\ 
  GDDUUK & MSCI UK Gross Total Return USD Index \\ 
  TPXDDVD & Topix Total Return Index JPY \\ 
  GDDUUS & MSCI Daily TR Gross USA USD \\ 
  GDDUWI & MSCI Daily TR Gross World USD \\ 
  SPTR350E & S\&P Europe 350 Gross Total Return Index \\ 
  SPXT & S\&P 500 Total Return Index \\ 
  SPXT & S\&P 500 Total Return Index \\ 
  JALSH & FTSE/JSE Africa All Share Index \\ 
  JALSH & FTSE/JSE Africa All Share Index \\ 
  GDDUE15X & MSCI Daily TR Gross Europe Ex UK USD \\ 
   \bottomrule
\caption{Index Names \label{tabdes}} 
\end{longtable}
\endgroup

\newpage

\hypertarget{references}{%
\section*{References}\label{references}}
\addcontentsline{toc}{section}{References}

\hypertarget{refs}{}
\begin{CSLReferences}{1}{0}
\leavevmode\vadjust pre{\hypertarget{ref-bacon2023practical}{}}%
Bacon, C.R. 2023. \emph{Practical portfolio performance measurement and
attribution}. John Wiley \& Sons.

\leavevmode\vadjust pre{\hypertarget{ref-baker2006investor}{}}%
Baker, M. \& Wurgler, J. 2006. Investor sentiment and the cross-section
of stock returns. \emph{The journal of Finance}. 61(4):1645--1680.

\leavevmode\vadjust pre{\hypertarget{ref-basu1977investment}{}}%
Basu, S. 1977. Investment performance of common stocks in relation to
their price-earnings ratios: A test of the efficient market hypothesis.
\emph{The journal of Finance}. 32(3):663--682.

\leavevmode\vadjust pre{\hypertarget{ref-bhattacharyya2007dividend}{}}%
Bhattacharyya, N. 1979. Dividend policy: A review. \emph{Managerial
Finance}. 33(1):4--13.

\leavevmode\vadjust pre{\hypertarget{ref-chen2009reversal}{}}%
Chen, L. 2009. On the reversal of return and dividend growth
predictability: A tale of two periods. \emph{Journal of Financial
Economics}. 92(1):128--151.

\leavevmode\vadjust pre{\hypertarget{ref-conover2016difference}{}}%
Conover, C.M., Jensen, G.R. \& Simpson, M.W. 2016. What difference do
dividends make? \emph{Financial Analysts Journal}. 72(6):28--40.

\leavevmode\vadjust pre{\hypertarget{ref-cornell2014dividend}{}}%
Cornell, B. 2014. Dividend-price ratios and stock returns: International
evidence. \emph{Journal of Portfolio management}. 40(2):122.

\leavevmode\vadjust pre{\hypertarget{ref-denis2008firms}{}}%
Denis, D.J. \& Osobov, I. 2008. Why do firms pay dividends?
International evidence on the determinants of dividend policy.
\emph{Journal of Financial economics}. 89(1):62--82.

\leavevmode\vadjust pre{\hypertarget{ref-gordon1963optimal}{}}%
Gordon, M.J. 1963. Optimal investment and financing policy. \emph{The
Journal of finance}. 18(2):264--272.

\leavevmode\vadjust pre{\hypertarget{ref-grullon2019dividend}{}}%
Grullon, G., Larkin, Y. \& Michaely, R. 2019. Dividend policy and
product market competition. \emph{Available at SSRN 972221}.

\leavevmode\vadjust pre{\hypertarget{ref-jensen1976theory}{}}%
Jensen, M.C. \& Meckling, W.H. 1976. Theory of the firm: Managerial
behavior, agency costs and ownership structure. \emph{Journal of
financial economics}. 3(4):305--360.

\leavevmode\vadjust pre{\hypertarget{ref-lintner1956distribution}{}}%
Lintner, J. 1956. Distribution of incomes of corporations among
dividends, retained earnings, and taxes. \emph{The American economic
review}. 46(2):97--113.

\leavevmode\vadjust pre{\hypertarget{ref-markowitz1959portfolio}{}}%
Markowitz, H.M. 1959. Portfolio selection, 1952{]}: Portfolio selection.
\emph{Journal of Finance}.

\leavevmode\vadjust pre{\hypertarget{ref-miller}{}}%
Miller, M.H. \& Modigliani, F. 1961. Dividend policy, growth, and the
valuation of shares. \emph{The Journal of Business}. 34(4):411--433.

\leavevmode\vadjust pre{\hypertarget{ref-walter1963dividend}{}}%
Walter, J.E. 1963. Dividend policy: Its influence on the value of the
enterprise. \emph{The Journal of finance}. 18(2):280--291.

\end{CSLReferences}

\bibliography{Tex/ref}





\end{document}
