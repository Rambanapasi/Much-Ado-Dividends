% Options for packages loaded elsewhere
\PassOptionsToPackage{unicode}{hyperref}
\PassOptionsToPackage{hyphens}{url}
%
\documentclass[
]{article}
\usepackage{amsmath,amssymb}
\usepackage{lmodern}
\usepackage{iftex}
\ifPDFTeX
  \usepackage[T1]{fontenc}
  \usepackage[utf8]{inputenc}
  \usepackage{textcomp} % provide euro and other symbols
\else % if luatex or xetex
  \usepackage{unicode-math}
  \defaultfontfeatures{Scale=MatchLowercase}
  \defaultfontfeatures[\rmfamily]{Ligatures=TeX,Scale=1}
\fi
% Use upquote if available, for straight quotes in verbatim environments
\IfFileExists{upquote.sty}{\usepackage{upquote}}{}
\IfFileExists{microtype.sty}{% use microtype if available
  \usepackage[]{microtype}
  \UseMicrotypeSet[protrusion]{basicmath} % disable protrusion for tt fonts
}{}
\makeatletter
\@ifundefined{KOMAClassName}{% if non-KOMA class
  \IfFileExists{parskip.sty}{%
    \usepackage{parskip}
  }{% else
    \setlength{\parindent}{0pt}
    \setlength{\parskip}{6pt plus 2pt minus 1pt}}
}{% if KOMA class
  \KOMAoptions{parskip=half}}
\makeatother
\usepackage{xcolor}
\usepackage[margin=1in]{geometry}
\usepackage{graphicx}
\makeatletter
\def\maxwidth{\ifdim\Gin@nat@width>\linewidth\linewidth\else\Gin@nat@width\fi}
\def\maxheight{\ifdim\Gin@nat@height>\textheight\textheight\else\Gin@nat@height\fi}
\makeatother
% Scale images if necessary, so that they will not overflow the page
% margins by default, and it is still possible to overwrite the defaults
% using explicit options in \includegraphics[width, height, ...]{}
\setkeys{Gin}{width=\maxwidth,height=\maxheight,keepaspectratio}
% Set default figure placement to htbp
\makeatletter
\def\fps@figure{htbp}
\makeatother
\setlength{\emergencystretch}{3em} % prevent overfull lines
\providecommand{\tightlist}{%
  \setlength{\itemsep}{0pt}\setlength{\parskip}{0pt}}
\setcounter{secnumdepth}{-\maxdimen} % remove section numbering
\ifLuaTeX
  \usepackage{selnolig}  % disable illegal ligatures
\fi
\IfFileExists{bookmark.sty}{\usepackage{bookmark}}{\usepackage{hyperref}}
\IfFileExists{xurl.sty}{\usepackage{xurl}}{} % add URL line breaks if available
\urlstyle{same} % disable monospaced font for URLs
\hypersetup{
  pdftitle={Literature Review},
  pdfauthor={Gabriel Rambanapasi},
  hidelinks,
  pdfcreator={LaTeX via pandoc}}

\title{Literature Review}
\author{Gabriel Rambanapasi}
\date{2023-07-07}

\begin{document}
\maketitle

\hypertarget{what-constitutes-a-dividend}{%
\section{What constitutes a
dividend?}\label{what-constitutes-a-dividend}}

Dividends are a form of cash payout that companies distribute to their
shareholders. They can take various forms, such as cash, stock,
liquidating, scrip, or property dividends. However, cash dividends are
the most common type. The decision to issue dividends is typically made
by the board of directors, taking into account the company's operating
needs for a given financial year. When a dividend is announced, it has
an impact on the financial statements of a company. It creates a
liability in the form of current liabilities and decreases shareholder
equity, specifically retained earnings, on the balance sheet. In other
words, the company incurs an obligation to pay out the dividend, and the
value of the company retained by shareholders decreases accordingly.

Cash dividends, although widely used, are not as tax-efficient as other
types of dividends, such as share buybacks. Share buybacks have gained
popularity in advanced economies, particularly in the United States,
where they reached a high of \$437 billion in 2018. Surprisingly, their
adoption has been relatively slow in South Africa. According to a study
by Wesson, Muller, and Ward in 2014, there were only 195 open market
share repurchases announced in South Africa from 1999 to 2009. In
comparison, Manconi, Peyer, and Vermaelen estimated that share
repurchases constituted approximately 58\% of total announcements in the
United States, 15\% in Canada, and 11\% in Japan over the same period.
This indicates a significant disparity in the adoption of share buybacks
between South Africa and these other countries. However

\hypertarget{why-do-firms-pay-dividends}{%
\subsection{Why do firms pay
dividends}\label{why-do-firms-pay-dividends}}

The issue of dividend policy and its impact on shareholder wealth has
sparked debates and opposing arguments in financial literature. On one
hand, the dividend puzzle suggests that dividends reduce equity value
and make investors worse off. According to this view, dividends can be
seen as a reward for investors who bear the risk associated with their
investments. Additionally, dividends can be considered a return on
investment rather than relying solely on capital gains. Various
literature has emerged trying to solve the puzzel, either supporting
irrelevance or relevance in dividend payments.The Modigliani-Miller (MM)
theory opines that dividends are irrelevant, it argues that shareholders
are indifferent to dividend payments, implying that there is no optimal
dividend policy. They contend that all dividend policies are equally
good and thus payments of dividends could easily be reinvested in shares
and make no difference to share holder wealth. However, the MM theorem
fails to consider real-world market imperfections that may give
relevance to dividend payments. Global assets market face multiple
constraints or imperfections namely flotation costs, transaction costs
(e.g., taxes and flotation costs), information asymmetry, and
principal-agent problems.

Tax preferences play a role in the argument for dividend relevance.
Different investors may be attracted to different stocks based on their
tax treatments, thus investors choose stocks based on their individual
investment needs. However, supporters of the MM theorem argue that
changes in dividend policy should not significantly impact stock prices
due to the substitution effect. According to this effect, allocation
decisions of firms occur almost simultaneously, resulting in a net zero
effect on prices. Flotation costs refer to the opportunity costs
incurred by a firm when paying dividends. By distributing dividends,
companies forego opportunities to expand their operations using retained
earnings. In a world without flotation costs, as suggested by the MM
theorem, management would be indifferent between issuing dividends and
borrowing from the market. However, in reality, external financing comes
at a higher cost, leading to trade-offs in dividend policy decisions.

Information asymmetry between shareholders and managers is another
factor that gives relevance to dividend payments. Investors rely on
dividend announcements to assess a company's stock price. Dividend
signaling conveys information about the company's quality. Investors
compare dividend announcements to historical levels while considering
company fundamentals. However, there is a risk of manipulation by
management, making the dividend signal imperfect for determining share
prices. Extending the argument on information asymmetry leads to the
principal-agent problem, where management and shareholders may have
differing goals for the use of retained earnings, leading to conflicts.
The free cash flow hypothesis suggests that dividend payments force
management to raise capital from external sources, which increases
borrowing costs and scrutiny from capital markets. This, in turn,
reduces management's ability to make sub optimal investments.

\hypertarget{when-would-dividends-matter}{%
\section{When Would Dividends
Matter}\label{when-would-dividends-matter}}

We now step away from the why onto the when - thus studying whether
dividend payments as a signal can have informative return predictive
value (irrespective of whether it is due to a value- or cash-flow
management proxy, or through a direct preference premium

\end{document}
