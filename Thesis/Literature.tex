% Options for packages loaded elsewhere
\PassOptionsToPackage{unicode}{hyperref}
\PassOptionsToPackage{hyphens}{url}
%
\documentclass[
]{article}
\usepackage{amsmath,amssymb}
\usepackage{lmodern}
\usepackage{iftex}
\ifPDFTeX
  \usepackage[T1]{fontenc}
  \usepackage[utf8]{inputenc}
  \usepackage{textcomp} % provide euro and other symbols
\else % if luatex or xetex
  \usepackage{unicode-math}
  \defaultfontfeatures{Scale=MatchLowercase}
  \defaultfontfeatures[\rmfamily]{Ligatures=TeX,Scale=1}
\fi
% Use upquote if available, for straight quotes in verbatim environments
\IfFileExists{upquote.sty}{\usepackage{upquote}}{}
\IfFileExists{microtype.sty}{% use microtype if available
  \usepackage[]{microtype}
  \UseMicrotypeSet[protrusion]{basicmath} % disable protrusion for tt fonts
}{}
\makeatletter
\@ifundefined{KOMAClassName}{% if non-KOMA class
  \IfFileExists{parskip.sty}{%
    \usepackage{parskip}
  }{% else
    \setlength{\parindent}{0pt}
    \setlength{\parskip}{6pt plus 2pt minus 1pt}}
}{% if KOMA class
  \KOMAoptions{parskip=half}}
\makeatother
\usepackage{xcolor}
\setlength{\emergencystretch}{3em} % prevent overfull lines
\providecommand{\tightlist}{%
  \setlength{\itemsep}{0pt}\setlength{\parskip}{0pt}}
\setcounter{secnumdepth}{-\maxdimen} % remove section numbering
\ifLuaTeX
  \usepackage{selnolig}  % disable illegal ligatures
\fi
\IfFileExists{bookmark.sty}{\usepackage{bookmark}}{\usepackage{hyperref}}
\IfFileExists{xurl.sty}{\usepackage{xurl}}{} % add URL line breaks if available
\urlstyle{same} % disable monospaced font for URLs
\hypersetup{
  pdftitle={Literature Review},
  pdfauthor={Gabriel Rambanapasi},
  hidelinks,
  pdfcreator={LaTeX via pandoc}}

\title{Literature Review}
\author{Gabriel Rambanapasi}
\date{2023-07-11}

\begin{document}
\maketitle

\hypertarget{dividends-constitution-and-impliaction-on-stakeholder-and-firms}{%
\section{Dividends constitution and Impliaction on Stakeholder and
Firms}\label{dividends-constitution-and-impliaction-on-stakeholder-and-firms}}

Dividends are cash payouts that companies use to distribute capital to
their shareholders. They can take various forms, such as cash, stock,
liquidating, scrip, or property dividends. However, cash dividends are
the most common type. The decision to issue dividends is typically made
by the board of directors, taking into account the company's operating
needs for a given financial year. When a dividend is announced, it has
an impact on the financial statements of a company. It creates a
liability in the form of current liabilities and decreases shareholder
equity, specifically retained earnings, on the balance sheet. In other
words, the company incurs an obligation to pay out the dividend, and the
value of the company retained by shareholders decreases accordingly.

\begin{itemize}
\tightlist
\item
  give examples of the decrease in shareholder value when dividends are
  announced
\item
  Include the argumemt that there are a proxy for value
\end{itemize}

Cash dividends, although widely used, are not as tax-efficient as other
types of capital distributions, such as share buybacks. Share buybacks
have gained popularity in advanced economies, particularly in the United
States, where they reached a high of \$437 billion in 2018.
Surprisingly, their adoption has been relatively slow in South Africa.
According to a study by Wesson, Muller, and Ward in 2014, there were
only 195 open market share repurchases announced in South Africa from
1999 to 2009. In comparison, Manconi, Peyer, and Vermaelen estimated
that share repurchases constituted approximately 58\% of total
announcements in the United States, 15\% in Canada, and 11\% in Japan
over the same period. This indicates a significant disparity in the
adoption of share buybacks across the world, despite their popularity in
the United States.

\begin{itemize}
\tightlist
\item
  update argument on sharebuybacks and why you just mentioned them
  instead of others
\end{itemize}

Where else are share buybacks less popular versus dividends Are
dividends still relevant

\hypertarget{why-do-firms-pay-dividends}{%
\subsection{Why do firms pay
dividends}\label{why-do-firms-pay-dividends}}

The issue of dividend policy and its impact on shareholder wealth has
sparked debates and opposing arguments in financial literature. On one
hand, the dividend puzzle suggests that dividends reduce equity value
and make investors worse off. According to this view, dividends can be
seen as a reward for investors who bear the risk associated with their
investments. Additionally, dividends can be considered a return on
investment rather than relying solely on capital gains. Various
literature has emerged trying to solve the puzzel, either supporting
irrelevance or relevance in dividend payments.The Modigliani-Miller (MM)
theory opines that dividends are irrelevant, it argues that shareholders
are indifferent to dividend payments, implying that there is no optimal
dividend policy. They contend that all dividend policies are equally
good and thus payments of dividends could easily be reinvested in shares
and make no difference to share holder wealth. However, the MM theorem
fails to consider real-world market imperfections that may give
relevance to dividend payments. Global assets market face multiple
constraints or imperfections namely flotation costs, transaction costs
(e.g., taxes and flotation costs), information asymmetry, and
principal-agent problems.

Tax preferences play a role in the argument for dividend relevance.
Different investors may be attracted to different stocks based on their
tax treatments, thus investors choose stocks based on their individual
investment needs. However, supporters of the MM theorem argue that
changes in dividend policy should not significantly impact stock prices
due to the substitution effect. According to this effect, allocation
decisions of firms occur almost simultaneously, resulting in a net zero
effect on prices. Flotation costs refer to the opportunity costs
incurred by a firm when paying dividends. By distributing dividends,
companies forego opportunities to expand their operations using retained
earnings. In a world without flotation costs, as suggested by the MM
theorem, management would be indifferent between issuing dividends and
borrowing from the market. However, in reality, external financing comes
at a higher cost, leading to trade-offs in dividend policy decisions.

Information asymmetry between shareholders and managers is another
factor that gives relevance to dividend payments. Investors rely on
dividend announcements to assess a company's stock price. Dividend
signaling conveys information about the company's quality. Investors
compare dividend announcements to historical levels while considering
company fundamentals. However, there is a risk of manipulation by
management, making the dividend signal imperfect for determining share
prices. Extending the argument on information asymmetry leads to the
principal-agent problem, where management and shareholders may have
differing goals for the use of retained earnings, leading to conflicts.
The free cash flow hypothesis suggests that dividend payments force
management to raise capital from external sources, which increases
borrowing costs and scrutiny from capital markets. This, in turn,
reduces management's ability to make sub optimal investments.

\hypertarget{dividend-portfolios-to-signal-returns}{%
\section{Dividend Portfolios To Signal
Returns}\label{dividend-portfolios-to-signal-returns}}

Studies on dividend signaling for returns can be categorized into
academic return signaling studies and practitioner-oriented long-term
return studies. Academic studies, such as Fama and French (1988),
initially found a positive correlation between increasing predictive
power and longer forecast horizons. However, subsequent studies like
Bekaert and Ang (2001) and Ang and Bekaert (2006) found no evidence of
long-term predictability in stock returns when considering finite sample
influence. This suggests that dividend yield may not be a reliable
predictor of subsequent returns. One possible reason for this declining
predictive power is the increasing use of share buybacks as an
alternative means for capital distribution, which reduces the
contribution of dividend yield to total return (Robertson and Wright,
2006).

On the other hand, practitioner-oriented literature focuses on the
long-term returns of systematic dividend portfolios. One popular
strategy is the ``Dogs of the Dow (DOD),'' which involves creating a
portfolio of the top 10 highest-paying dividend stocks on the Dow Jones
Industrial Index at the beginning of the year based on the dividends
paid in the previous 12 months. The portfolio is held for 12 months, and
the process is repeated annually. Various studies have examined the DOD
strategy or similar high-yield dividend strategies in different time
periods and regions, consistently showing superior risk-adjusted returns
compared to the market index. Examples of such studies include Da Silva
(2001) in Latin America, Alles and Sheng (2008) in Australia, Visscher
and Filbeck (2003) in Canada, Kotkamp and Otte (2001) in Germany, and
Wang et al.~(2011) in China. More recently, Filbeck (2017) investigated
the performance of DOD against a high-yield portfolio of Fortune Most
Desired Companies (MAC) compared to the Dow Jones Industrial Average and
the S\&P 500. The study found significantly higher risk-adjusted returns
for the DOD strategy.

\begin{itemize}
\tightlist
\item
  Give a summary of the indexes and methodology used in all the papers
  listed.
\item
  Look at papers that investigate dividend signalling by sector, this
  moves it away from just bunching all sectors into one portfolio
\item
\end{itemize}

\end{document}
